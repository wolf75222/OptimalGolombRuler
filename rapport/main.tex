\documentclass[a4paper,12pt]{report}

% Citation en début de chapitre
\usepackage{epigraph}

% Encodages et langue
\usepackage[utf8]{inputenc} % Encodage UTF-8
\usepackage[T1]{fontenc}
\usepackage[french]{babel}    % Langue du document
\usepackage{csquotes}          % Requis pour biblatex avec babel

\usepackage{biblatex}
\addbibresource{references.bib}
\usepackage{libertine}         % Added for Libertine font
\usepackage{setspace}          % Added for \setstretch
\usepackage{tabularx}

% Mise en page
\usepackage[margin=2.5cm, headheight=15pt]{geometry} % Marges du document

% Python 
\usepackage{pythonhighlight}

\usepackage{fancybox} % Pour créer des cadres élégants


% Inclusion d'images
\usepackage{graphicx}          % Pour inclure des images

% Ajout du package float pour le positionnement [H]
\usepackage{float}

% Bibliothèques pour les mathématiques
\usepackage{amsmath} % Mathématiques avancées
\usepackage{amssymb} % Symboles mathématiques
\usepackage{amsfonts} % Polices mathématiques
\usepackage{amsthm} % Définitions et théorèmes
\usepackage{algorithm}
\usepackage{algpseudocode}
\usepackage{tikz}
\usepackage{pgfplots}
\pgfplotsset{compat=1.18}
\usepackage{amsmath, amsthm}
% Configuration des définitions et théorèmes
\newtheorem{definition}{Définition}[chapter] % Définitions numérotées par chapitre
\newtheorem{theorem}{Théorème}[chapter] % Théorèmes numérotés par chapitre
\newtheorem{example}{Exemple}[chapter] % Exemples numérotés par chapitre
\newtheorem{remark}{Remarque}[chapter] % Remarques numérotées par chapitre


\usepackage{xcolor}
\definecolor{grey}{rgb}{0.5, 0.5, 0.5} % Définit la couleur grey

% Couleurs personnalisées pour la page de garde
\definecolor{burgundy}{HTML}{8e5e2d} % Couleur principale #4b0e20
\definecolor{gold}{HTML}{4c1207} % Couleur secondaire #b29959

% Configuration des couleurs pour les sections, sous-sections, etc.
\usepackage{titlesec}
\usepackage{subcaption}
\usepackage{svg}
\usepackage{wrapfig}


% Personnalisation des titres de section en burgundy
\titleformat{\section}
  {\Large\bfseries\color{burgundy}}
  {\thesection}
  {1em}
  {}

\titleformat{\subsection}
  {\large\bfseries\color{burgundy}}
  {\thesubsection}
  {1em}
  {}

\titleformat{\subsubsection}
  {\normalsize\bfseries\color{burgundy}}
  {\thesubsubsection}
  {1em}
  {}

% Hyperliens
\usepackage{hyperref}
\hypersetup{
    colorlinks=true,            % Colore les liens
    linkcolor=burgundy,         % Couleur des liens internes (ex. table des matières) - burgundy
    citecolor=gold,             % Couleur des références bibliographiques - gold
    filecolor=burgundy,         % Couleur des liens vers des fichiers locaux - burgundy
    urlcolor=gold               % Couleur des URL - gold
}




\usepackage{enumitem}
\usepackage{booktabs}
\usepackage{multirow}
\usepackage{siunitx}
\usepackage{subcaption}

% En-têtes et pieds de page
\usepackage{fancyhdr}          % Pour personnaliser les en-têtes et pieds de page
\usepackage{lastpage}          % Pour référencer la dernière page

% Configuration du package fancyhdr
\pagestyle{fancy}
\fancyhf{} % Efface tous les champs de l'en-tête et du pied de page

% En-tête
\lhead{\textbf{\color{burgundy}CHPS0701 : Algorithmique et Programmation Parallèle}} % En-tête à gauche
\rhead{\textbf{\color{burgundy}M1 CHPS}} % En-tête à droite

% Pied de page
\lfoot{\textbf{\color{burgundy}Romain Despoullains}} % Pied de page à gauche
\rfoot{\textbf{\color{burgundy}Page \thepage\ \color{burgundy}/ \color{burgundy}\pageref{LastPage}}} % Pied de page à droite

% --- Personnalisation des lignes d'en-tête et de pied de page ---
\renewcommand{\headrulewidth}{0.4pt} % Épaisseur de la ligne sous l'en-tête
\renewcommand{\footrulewidth}{0.4pt} % Épaisseur de la ligne au-dessus du pied de page

% Redéfinition de \headrule et \footrule pour changer leur couleur
% On utilise \fancy@setcolor pour que la couleur soit correctement gérée par fancyhdr
\makeatletter % Nécessaire pour utiliser des commandes avec @

% Pour la ligne d'en-tête
\renewcommand{\headrule}{%
  \if@fancyplain\let\headrulewidth\plainheadrulewidth\fi
  {\color{burgundy} % Définit la couleur de la règle
  \hrule\@width\headwidth\@height\headrulewidth\vskip-\headrulewidth}
  \if@fancyplain\let\headrulewidth\plainheadrulewidth\fi % Rétablit la valeur originale (important)
}

% Pour la ligne de pied de page
\renewcommand{\footrule}{%
  \if@fancyplain\let\footrulewidth\plainfootrulewidth\fi
  \vskip-\footruleskip\vskip-\footrulewidth
  {\color{burgundy} % Définit la couleur de la règle
  \hrule\@width\headwidth\@height\footrulewidth\vskip\footruleskip} % Utilise \headwidth pour la largeur
  \if@fancyplain\let\footrulewidth\plainfootrulewidth\fi % Rétablit la valeur originale
}
\makeatother
% Ajouter les lignes sous l'en-tête et au-dessus du pied de page
\renewcommand{\headrulewidth}{0.4pt} % Ligne sous l'en-tête
\renewcommand{\footrulewidth}{0.4pt} % Ligne au-dessus du pied de page

% Inclusion de code source (si nécessaire)
\usepackage{listings}      % Inclusion de code source
\usepackage{color}         % Couleurs


\definecolor{lightgray}{gray}{0.95}
\definecolor{darkgray}{gray}{0.4}
\definecolor{purple}{rgb}{0.58,0,0.82}

\lstdefinelanguage{bash}{
  keywords={cd, ls, mkdir, rm, mv, cp, sudo, apt, grep, find, echo, cat, git, docker, cargo},
  sensitive=true,
  comment=[l]{\#},
  morecomment=[l]{\#},
  morestring=[b]",
}

\lstset{
  language=bash,
  basicstyle=\small\ttfamily,
  backgroundcolor=\color{lightgray},
  commentstyle=\color{darkgray}\itshape,
  keywordstyle=\color{purple}\bfseries,
  stringstyle=\color{blue},
  breaklines=true,
  breakatwhitespace=true,
  frame=single,
  showstringspaces=false,
  tabsize=2,
  captionpos=b,
  extendedchars=true,
  literate=
    {├}{{\char"251C}}1
    {─}{{\char"2500}}1
    {└}{{\char"2514}}1
    {│}{{\char"2502}}1
}


% Définition des couleurs pour la coloration syntaxique
\definecolor{dkgreen}{rgb}{0,0.6,0}
\definecolor{gray}{rgb}{0.5,0.5,0.5}
\definecolor{mauve}{rgb}{0.58,0,0.82}

% Configuration du package listings
\definecolor{maroon}{cmyk}{0, 0.87, 0.68, 0.32}
\definecolor{halfgray}{gray}{0.55}
\definecolor{ipython_frame}{RGB}{207, 207, 207}
\definecolor{ipython_bg}{RGB}{247, 247, 247}
\definecolor{ipython_red}{RGB}{186, 33, 33}
\definecolor{ipython_green}{RGB}{0, 128, 0}
\definecolor{ipython_cyan}{RGB}{64, 128, 128}
\definecolor{ipython_purple}{RGB}{170, 34, 255}

\usepackage{listings}
\lstset{
    breaklines=true,
    %
    extendedchars=true,
    literate=
    {á}{{\'a}}1 {é}{{\'e}}1 {í}{{\'i}}1 {ó}{{\'o}}1 {ú}{{\'u}}1
    {Á}{{\'A}}1 {É}{{\'E}}1 {Í}{{\'I}}1 {Ó}{{\'O}}1 {Ú}{{\'U}}1
    {à}{{\`a}}1 {è}{{\`e}}1 {ì}{{\`i}}1 {ò}{{\`o}}1 {ù}{{\`u}}1
    {À}{{\`A}}1 {È}{{\'E}}1 {Ì}{{\`I}}1 {Ò}{{\`O}}1 {Ù}{{\`U}}1
    {ä}{{\"a}}1 {ë}{{\"e}}1 {ï}{{\"i}}1 {ö}{{\"o}}1 {ü}{{\"u}}1
    {Ä}{{\"A}}1 {Ë}{{\"E}}1 {Ï}{{\"I}}1 {Ö}{{\"O}}1 {Ü}{{\"U}}1
    {â}{{\^a}}1 {ê}{{\^e}}1 {î}{{\^i}}1 {ô}{{\^o}}1 {û}{{\^u}}1
    {Â}{{\^A}}1 {Ê}{{\^E}}1 {Î}{{\^I}}1 {Ô}{{\^O}}1 {Û}{{\^U}}1
    {œ}{{\oe}}1 {Œ}{{\OE}}1 {æ}{{\ae}}1 {Æ}{{\AE}}1 {ß}{{\ss}}1
    {ç}{{\c c}}1 {Ç}{{\c C}}1 {ø}{{\o}}1 {å}{{\r a}}1 {Å}{{\r A}}1
    {€}{{\EUR}}1 {£}{{\pounds}}1,
    %
    literate=
    *{+}{{{\color{ipython_purple}+}}}1
    {-}{{{\color{ipython_purple}-}}}1
    {*}{{{\color{ipython_purple}$^\ast$}}}1
    {/}{{{\color{ipython_purple}/}}}1
    {^}{{{\color{ipython_purple}\^{}}}}1
    {?}{{{\color{ipython_purple}?}}}1
    {!}{{{\color{ipython_purple}!}}}1
    {\%}{{{\color{ipython_purple}\%}}}1
    {<}{{{\color{ipython_purple}<}}}1
    {>}{{{\color{ipython_purple}>}}}1
    {|}{{{\color{ipython_purple}|}}}1
    {\&}{{{\color{ipython_purple}\&}}}1
    {~}{{{\color{ipython_purple}~}}}1
    %
    {==}{{{\color{ipython_purple}==}}}2
    {<=}{{{\color{ipython_purple}<=}}}2
    {>=}{{{\color{ipython_purple}>=}}}2
    %
    {+=}{{{+=}}}2
    {-=}{{{-=}}}2
    {*=}{{{$^\ast$=}}}2
    {/=}{{{/=}}}2,
    %
%   identifierstyle=\color{red}\ttfamily,
    commentstyle=\color{ipython_cyan}\ttfamily,
    stringstyle=\color{ipython_red}\ttfamily,
    keepspaces=true,
    showspaces=false,
    showstringspaces=false,
    %
    rulecolor=\color{ipython_frame},
    frame=single,
    frameround={t}{t}{t}{t},
    framexleftmargin=6mm,
    numbers=left,
    numberstyle=\tiny\color{halfgray},
    %
    %
    backgroundcolor=\color{ipython_bg},
    %   extendedchars=true,
    basicstyle=\scriptsize\ttfamily,
    keywordstyle=\color{ipython_green}\ttfamily,
    escapechar=@,
}


% Inclusion de PlantUML pour les schémas basiques
\usepackage{tikz}
\usetikzlibrary{shapes.geometric, arrows, positioning, calc, patterns, decorations.pathreplacing, matrix, fit, backgrounds}

% Command for horizontal rule on title page
\newcommand{\HRule}{\rule{\linewidth}{0.5mm}}

% Informations du document
\title{Cours}
\author{Université de Reims Champagne Ardennes \\ Despoullains Romain}
\date{\\today}

\usepackage{tcolorbox} % Pour les boîtes stylisées
\tcbuselibrary{breakable} % Pour permettre les boîtes sur plusieurs pages

\tcbset {
  base/.style={
    arc=0mm, 
    bottomtitle=0.5mm,
    boxrule=0mm,
    colbacktitle=black!10!white, 
    coltitle=black, 
    fonttitle=\bfseries, 
    left=2.5mm,
    leftrule=1mm,
    right=3.5mm,
    title={#1},
    toptitle=0.75mm, 
  }
}

% Définition des couleurs
\definecolor{brandblue}{rgb}{0.34, 0.7, 1}
\definecolor{brandgreen}{rgb}{0.0, 0.8, 0.4}
\definecolor{brandorange}{rgb}{1.0, 0.6, 0.0}
\definecolor{brandred}{rgb}{0.9, 0.1, 0.2}

\newtcolorbox{methode}[1]{
  colframe=brandblue, 
  base={#1}
}

\newtcolorbox{defi}[1]{
  colframe=brandgreen,
  base={#1}
}

\newtcolorbox{exemple}[1]{
  colframe=brandorange,
  base={#1}
}

\newtcolorbox{important}[1]{
  colframe=brandred,
  base={#1}
}

% Couleurs supplémentaires pour les encadrés
\definecolor{brandpurple}{rgb}{0.5, 0.0, 0.5}
\definecolor{brandteal}{rgb}{0.0, 0.5, 0.5}

% Encadré pour les résultats importants
\newtcolorbox{result}{
  colframe=brandpurple,
  colback=brandpurple!5,
  arc=2mm,
  boxrule=1pt,
  left=3mm,
  right=3mm,
  top=2mm,
  bottom=2mm,
  fonttitle=\bfseries,
  title={Résultat clé}
}

% Encadré pour les recommandations
\newtcolorbox{recommendation}{
  colframe=brandteal,
  colback=brandteal!5,
  arc=2mm,
  boxrule=1pt,
  left=3mm,
  right=3mm,
  top=2mm,
  bottom=2mm,
  fonttitle=\bfseries,
  title={Recommandation}
}

% Packages for background image and graphics
\usepackage{eso-pic}
\usepackage{graphicx}
\usepackage{tikz}
%-----------------------------------------------------------------
% Command to place a background image at 15% opacity
%-----------------------------------------------------------------
\newcommand\BackgroundPic{%
  \begin{tikzpicture}[remember picture,overlay]
    % The node has "opacity=0.15" to set the entire image at 15% visibility.
    \node[opacity=0.20] at (current page.center){%
      \includegraphics[
        width=\paperwidth,       % fill horizontally
        height=\paperheight,     % fill vertically
        keepaspectratio          % preserve aspect ratio
      ]{assets/back.png}    % <-- cover_background.png
    };
  \end{tikzpicture}%
}

% Configuration des chapitres avec fncychap (style Bjornstrup personnalisé)
\usepackage[Bjornstrup]{fncychap}

% Commandes de couleur personnalisées pour fncychap
\newcommand{\colortitlechap}{\color{gold}} % couleur du titre de chapitre en gold
\newcommand{\colornumberchap}{\color{gold}} % couleur du numéro de chapitre en gold
\newcommand{\colorbackchap}{\colorbox{burgundy}} % couleur de fond des règles en burgundy

\makeatletter

\renewcommand{\DOCH}{
    \settowidth{\py}{\CNoV\thechapter}
    \addtolength{\py}{-10pt}% 
    \fboxsep=0pt%
    \colorbackchap{\rule{0pt}{40pt}\parbox[b]{\textwidth}{\hfill}}%
    \kern-\py\raise20pt%
    \hbox{\colornumberchap\CNoV\thechapter}\\%
}

\renewcommand{\DOTI}[1]{%
    \nointerlineskip\raggedright%
    \fboxsep=\myhi%
    \vskip-1ex%
    \colorbackchap{\parbox[t]{\mylen}{\CTV\FmTi{\color{gold}#1}}}\par\nobreak%
    \vskip 40\p@%
}

\renewcommand{\DOTIS}[1]{%
    \fboxsep=0pt%
    \colorbackchap{\rule{0pt}{40pt}\parbox[b]{\textwidth}{\hfill}}\\%
    \nointerlineskip\raggedright%
    \fboxsep=\myhi%
    \colorbackchap{\parbox[t]{\mylen}{\CTV\FmTi{\color{gold}#1}}}\par\nobreak%
    \vskip 40\p@%
}

\makeatother

\author{DESPOULLAINS ROMAIN}

\begin{document}

% Title page with background image
\begin{titlepage}
    \BackgroundPic % Add the background image

    % === LOGOS EN HAUT ===

    \noindent

    \begin{minipage}[t]{0.45\textwidth}
        \raggedright
        \includegraphics[width=4cm]{assets/logo-urca.png} 
    \end{minipage}%
    \hfill
    \begin{minipage}[t]{0.45\textwidth}
        \raggedleft
        \includegraphics[width=5.3cm]{assets/logo.jpg}
    \end{minipage}    

      \vspace*{2.5cm}

    % === TITRE PRINCIPAL AVEC BARRES HARMONISÉES ===
    \centering
    {\color{gold}\rule{0.6\textwidth}{1pt}}\\[0.5cm]
    {\fontsize{22}{26}\selectfont \bfseries \color{burgundy} Rapport Projet}\\[0.5cm]
    {\color{gold}\rule{0.6\textwidth}{1pt}}\\[1.5cm]% === SOUS-TITRES DE PROJET ===
    \begin{minipage}{0.85\textwidth}
        \centering
        {\Large \textbf{Recherche de Règles de Golomb Optimales}}\\[0.3cm]
        {\large Approches Séquentielle, OpenMP et MPI Hybride}\\[0.5cm]
    \end{minipage}

    \vspace*{7cm}

    % === INFORMATIONS PERSONNELLES ET ACADÉMIQUES ===
    \begin{minipage}{0.6\textwidth}
        \centering        \begin{tabular}{r l}
            \textbf{Auteur} & : \textit{M. Despoullains} \\
        \end{tabular}
    \end{minipage}

    \vspace*{2cm}    % === FILIÈRE ET ANNÉE ===

    {\footnotesize \textcolor{gray}{Université de Reims Champagne-Ardenne}}\\[0.2cm]
    {\footnotesize \textcolor{gray}{CHPS0701 : Algorithmique et Programmation Parallèle}}\\[0.2cm]
    {\footnotesize \textcolor{gray}{2025 - 2026}}

    \vspace*{1cm}
\end{titlepage}
\clearpage % Ensures that content following the title page starts on a new page

% === RÉSUMÉ EN FRANÇAIS ===
\chapter*{Résumé}
\addcontentsline{toc}{chapter}{Résumé}

Ce rapport présente l'implémentation et l'optimisation d'algorithmes de recherche de \textbf{règles de Golomb optimales}, un problème combinatoire NP-difficile aux applications variées (radioastronomie, codes correcteurs d'erreurs, théorie de l'information). \\

Nous développons une approche par \textit{backtracking} avec élagage \textit{branch-and-bound}, optimisée selon les principes CSAPP (déroulage de boucles, structures alignées en cache, opérations bit-à-bit). L'algorithme est ensuite parallélisé via \textbf{OpenMP} (mémoire partagée) et une architecture \textbf{hybride MPI+OpenMP} (mémoire distribuée avec topologie hypercube). \\

Les résultats démontrent des accélérations significatives sur des configurations allant jusqu'à 192 threads et 32 processus MPI, validés par comparaison aux solutions optimales connues ($n \leq 14$). \\

\textbf{Mots-clés :} Règles de Golomb, backtracking, branch-and-bound, OpenMP, MPI, calcul haute performance, optimisation combinatoire. \\

% === ABSTRACT EN ANGLAIS ===
\chapter*{Abstract}
\addcontentsline{toc}{chapter}{Abstract}

This report presents the implementation and optimization of search algorithms for \textbf{optimal Golomb rulers}, an NP-hard combinatorial problem with various applications (radio astronomy, error-correcting codes, information theory). \\

We develop a \textit{backtracking} approach with \textit{branch-and-bound} pruning, optimized according to CSAPP principles (loop unrolling, cache-aligned structures, bitwise operations). The algorithm is then parallelized using \textbf{OpenMP} (shared memory) and a \textbf{hybrid MPI+OpenMP} architecture (distributed memory with hypercube topology). \\

Results demonstrate significant speedups on configurations up to 192 threads and 32 MPI processes, validated against known optimal solutions ($n \leq 14$). \\

\textbf{Keywords:} Golomb rulers, backtracking, branch-and-bound, OpenMP, MPI, high-performance computing, combinatorial optimization. \\

% === TABLE DES MATIÈRES ===
\tableofcontents
\clearpage

% === LISTE DES FIGURES ===
\listoffigures
\addcontentsline{toc}{chapter}{Liste des figures}
\clearpage

% === LISTE DES TABLEAUX ===
\listoftables
\addcontentsline{toc}{chapter}{Liste des tableaux}
\clearpage

% =============================================================================
% INCLUSION DES CHAPITRES
% =============================================================================

% Chapitre 1 : Introduction
\chapter{Introduction}

\epigraph{\textit{``The most incomprehensible thing about the world is that it is comprehensible.''}}{--- Albert Einstein}

\section{Contexte}

L'optimisation combinatoire constitue un domaine central de l'informatique théorique et appliquée, caractérisé par la recherche de solutions optimales dans des espaces de configurations discrets et souvent exponentiels. Parmi les problèmes emblématiques de ce domaine, le \textbf{problème des règles de Golomb} occupe une place singulière de par sa simplicité apparente et sa difficulté intrinsèque.

Une \textit{règle de Golomb} est un ensemble de marques positionnées sur une règle graduée tel qu'aucune paire de marques ne mesure la même distance. Formellement, pour un ensemble de $n$ marques $\{m_0, m_1, \ldots, m_{n-1}\}$ avec $m_0 = 0 < m_1 < \cdots < m_{n-1}$, la propriété de Golomb exige que :
\[
\forall (i,j) \neq (k,l), \quad m_j - m_i \neq m_l - m_k
\]

Le problème de la \textbf{règle de Golomb optimale} (OGR, \textit{Optimal Golomb Ruler}) consiste à trouver, pour un nombre de marques $n$ donné, la règle de longueur minimale $m_{n-1}$. Ce problème appartient à la classe des problèmes NP-difficiles : aucun algorithme polynomial n'est connu pour le résoudre, et la complexité croît de manière exponentielle avec $n$.

Les règles de Golomb trouvent des applications concrètes dans plusieurs domaines :
\begin{itemize}
    \item \textbf{Radioastronomie} : positionnement optimal des antennes dans les interféromètres pour maximiser les lignes de base distinctes ;
    \item \textbf{Théorie de l'information} : conception de codes correcteurs d'erreurs et de séquences à faible autocorrélation ;
    \item \textbf{Télécommunications} : allocation de fréquences sans interférence dans les réseaux à accès multiple.
\end{itemize}

\subsection{Intérêt du calcul haute performance}

La nature combinatoire du problème OGR rend les approches exhaustives impraticables au-delà de quelques marques : l'espace de recherche croît exponentiellement. À titre d'exemple, la recherche de la règle optimale à 13 marques requiert l'exploration de milliards de configurations.

Le \textbf{calcul haute performance} (HPC, \textit{High-Performance Computing}) offre une réponse à ce défi en permettant :
\begin{itemize}
    \item L'exploitation du \textbf{parallélisme à mémoire partagée} via des frameworks comme OpenMP, permettant d'utiliser efficacement les architectures multicœurs modernes ;
    \item La \textbf{distribution du calcul} sur plusieurs nœuds via MPI (\textit{Message Passing Interface}), élargissant considérablement les ressources computationnelles disponibles ;
    \item L'application de techniques d'\textbf{optimisation bas niveau} (vectorisation, alignement mémoire, déroulage de boucles) pour maximiser l'efficacité de chaque thread.
\end{itemize}

Ce projet s'inscrit dans cette démarche : exploiter les paradigmes de programmation parallèle pour repousser les limites de ce qui est calculable dans un temps raisonnable.

\section{Objectifs du projet}

Ce projet poursuit plusieurs objectifs complémentaires, articulés autour de trois axes principaux :

\subsection{Trouver les règles de Golomb optimales}

L'objectif premier est de développer un algorithme de recherche exhaustive capable de trouver les règles de Golomb optimales pour un nombre de marques donné. Cet algorithme doit :
\begin{itemize}
    \item Garantir l'\textbf{optimalité} de la solution trouvée (la règle de longueur minimale) ;
    \item Être \textbf{correct}, c'est-à-dire ne jamais manquer la solution optimale ;
    \item Utiliser des techniques d'\textbf{élagage} efficaces (\textit{branch-and-bound}) pour réduire drastiquement l'espace de recherche.
\end{itemize}

\subsection{Accélérer la recherche via le parallélisme}

Le second objectif consiste à paralléliser l'algorithme selon deux paradigmes complémentaires :

\begin{enumerate}
    \item \textbf{OpenMP} (mémoire partagée) : distribution du travail entre les threads d'un même nœud, avec synchronisation légère et partage de la borne supérieure courante ;
    \item \textbf{MPI+OpenMP} (architecture hybride) : distribution des sous-problèmes entre processus MPI communiquant selon une topologie en \textit{hypercube}, chaque processus utilisant OpenMP en interne.
\end{enumerate}

Cette approche hybride vise à combiner les avantages des deux paradigmes : la faible latence d'OpenMP pour la synchronisation intra-nœud et la scalabilité de MPI pour la distribution inter-nœuds.

\subsection{Produire des benchmarks rigoureux}

Le troisième objectif est d'évaluer rigoureusement les performances des différentes implémentations à travers :
\begin{itemize}
    \item Des \textbf{mesures de temps d'exécution} sur des configurations variées ($n = 10$ à $n = 14$ marques) ;
    \item Le calcul de métriques de \textbf{scalabilité} : speedup et efficacité parallèle ;
    \item Des expérimentations sur le \textbf{supercalculateur Romeo} de l'Université de Reims, avec des configurations allant jusqu'à 192 threads et 32 processus MPI ;
    \item Une \textbf{comparaison} des différentes versions de l'algorithme (séquentielle, OpenMP, MPI hybride) et des optimisations successives.
\end{itemize}

\section{Contributions et organisation du rapport}

\subsection{Contributions}

Les principales contributions de ce travail sont les suivantes :

\begin{enumerate}
    \item \textbf{Implémentation d'un algorithme de backtracking optimisé} : utilisation d'une structure \texttt{BitSet128} permettant la détection de collisions en $O(1)$ via des opérations de décalage binaire, au lieu de $O(n)$ pour l'approche naïve ;

    \item \textbf{Six versions OpenMP} avec des niveaux d'optimisation croissants : de l'approche itérative de base avec déroulage de boucles jusqu'à la version V5 utilisant des opérations \texttt{uint64\_t} directes ;

    \item \textbf{Trois versions MPI+OpenMP} exploitant différentes stratégies de communication : topologie hypercube avec réduction logarithmique et \texttt{MPI\_Allreduce} standard ;

    \item \textbf{Validation expérimentale} sur architectures x86 et ARM, avec des résultats reproduisant les règles optimales connues jusqu'à $n = 14$.
\end{enumerate}

\subsection{Organisation du rapport}

Ce rapport est structuré en onze chapitres, suivis d'annexes techniques :

\begin{itemize}
    \item Le \textbf{Chapitre 2} présente les \textit{préliminaires} nécessaires : complexité algorithmique, paradigmes de programmation parallèle (OpenMP, MPI), métriques de performance ;

    \item Le \textbf{Chapitre 3} formalise le \textit{problème des règles de Golomb} : définitions, propriétés, critère d'optimalité et résultats connus ;

    \item Le \textbf{Chapitre 4} détaille l'\textit{approche algorithmique séquentielle} : backtracking, branch-and-bound, élagage et optimisation BitSet128 ;

    \item Le \textbf{Chapitre 5} aborde l'\textit{implémentation C++} et les aspects d'ingénierie logicielle : structures de données, organisation du code, compilation ;

    \item Le \textbf{Chapitre 6} présente la \textit{parallélisation OpenMP} avec les six versions successives et leurs optimisations ;

    \item Le \textbf{Chapitre 7} traite de la \textit{parallélisation hybride MPI+OpenMP}, de la topologie hypercube et de l'équilibrage de charge ;

    \item Le \textbf{Chapitre 8} décrit le \textit{protocole expérimental} et la méthodologie de benchmarking sur le cluster Romeo ;

    \item Le \textbf{Chapitre 9} présente les \textit{résultats} obtenus, leur analyse selon les principes CSAPP, et les leçons du profilage ;

    \item Le \textbf{Chapitre 10} discute des \textit{limites et perspectives} : évolution des versions, limites actuelles, pistes d'amélioration ;

    \item Le \textbf{Chapitre 11} \textit{conclut} ce rapport en résumant les apports, les recommandations pour une V2.0, et les réflexions personnelles.
\end{itemize}

Les \textbf{Annexes} fournissent les détails techniques : structures de données, commandes de compilation, format CSV des benchmarks, et tables de solutions optimales de référence.


% Chapitre 2 : Préliminaires
\chapter{Préliminaires}
\label{chap:preliminaires}

\epigraph{\textit{``The purpose of computing is insight, not numbers.''}}{--- Richard Hamming}

Ce chapitre introduit les concepts fondamentaux nécessaires à la compréhension des techniques employées dans ce rapport. Nous présentons les notions de complexité algorithmique, les paradigmes de programmation parallèle (OpenMP et MPI), ainsi que les métriques de performance utilisées pour évaluer nos implémentations.

% =============================================================================
\section{Complexité algorithmique}
\label{sec:prelim:complexite}
% =============================================================================

\subsection{Notations asymptotiques}

L'analyse de complexité permet d'évaluer le comportement d'un algorithme en fonction de la taille de l'entrée. Nous utilisons les notations asymptotiques standard :

\begin{definition}[Notation $O$ (grand O)]
Soit $f, g : \mathbb{N} \to \mathbb{R}^+$. On dit que $f(n) = O(g(n))$ s'il existe des constantes $c > 0$ et $n_0$ telles que :
\[
\forall n \geq n_0 : f(n) \leq c \cdot g(n)
\]
\end{definition}

\begin{definition}[Notation $\Omega$ (grand Omega)]
$f(n) = \Omega(g(n))$ si $g(n) = O(f(n))$, c'est-à-dire si $f$ croît au moins aussi vite que $g$.
\end{definition}

\begin{definition}[Notation $\Theta$ (grand Theta)]
$f(n) = \Theta(g(n))$ si $f(n) = O(g(n))$ et $f(n) = \Omega(g(n))$, c'est-à-dire si $f$ et $g$ ont le même ordre de croissance.
\end{definition}

\subsection{Classes de complexité courantes}

\begin{table}[H]
\centering
\begin{tabular}{lll}
\toprule
\textbf{Classe} & \textbf{Nom} & \textbf{Exemple} \\
\midrule
$O(1)$ & Constante & Accès tableau, opérations bit-à-bit \\
$O(\log n)$ & Logarithmique & Recherche dichotomique \\
$O(n)$ & Linéaire & Parcours de tableau \\
$O(n \log n)$ & Quasi-linéaire & Tri fusion, tri rapide (moyen) \\
$O(n^2)$ & Quadratique & Tri par insertion, produit matriciel naïf \\
$O(2^n)$ & Exponentielle & Backtracking exhaustif \\
$O(n!)$ & Factorielle & Génération de permutations \\
\bottomrule
\end{tabular}
\caption{Classes de complexité courantes}
\label{tab:complexity_classes}
\end{table}

\subsection{Problèmes NP-difficiles}

\begin{definition}[Classe NP]
La classe \textbf{NP} (Nondeterministic Polynomial time) contient les problèmes de décision dont une solution candidate peut être \textit{vérifiée} en temps polynomial.
\end{definition}

\begin{definition}[Problème NP-difficile]
Un problème est \textbf{NP-difficile} s'il est au moins aussi difficile que tout problème dans NP. Formellement, tout problème de NP peut être réduit polynomialement à ce problème.
\end{definition}

Le problème des règles de Golomb optimales est NP-difficile : aucun algorithme polynomial n'est connu pour le résoudre, et la vérification qu'une règle est optimale nécessite une exploration exhaustive de l'espace de recherche.

% =============================================================================
\section{Programmation parallèle : concepts fondamentaux}
\label{sec:prelim:parallele}
% =============================================================================

\subsection{Motivation}

La loi de Moore, qui prédisait un doublement de la densité des transistors tous les 18 mois, a atteint ses limites physiques. Depuis le milieu des années 2000, l'augmentation des performances passe principalement par le parallélisme :
\begin{itemize}
    \item \textbf{Multicœur} : plusieurs unités de calcul sur une même puce
    \item \textbf{SIMD} : une instruction, plusieurs données (vectorisation)
    \item \textbf{Clusters} : plusieurs machines interconnectées
\end{itemize}

\subsection{Modèles de parallélisme}

\paragraph{Mémoire partagée.}
Tous les threads accèdent à un espace mémoire commun. La synchronisation se fait via des mécanismes comme les mutex, les variables atomiques ou les barrières. Ce modèle est utilisé au sein d'un nœud de calcul.

\paragraph{Mémoire distribuée.}
Chaque processus possède son propre espace mémoire. La communication se fait par passage de messages. Ce modèle est utilisé entre nœuds d'un cluster.

\paragraph{Modèle hybride.}
Combinaison des deux : mémoire partagée au sein de chaque nœud (OpenMP), mémoire distribuée entre nœuds (MPI). C'est le modèle dominant en HPC.

\begin{figure}[H]
\centering
\begin{tikzpicture}[scale=0.8]
    % Noeud 1
    \draw[thick, burgundy] (0,0) rectangle (5,4);
    \node[above] at (2.5,4) {\textbf{Nœud 1}};
    \draw[fill=gold!30] (0.5,0.5) rectangle (2,1.5);
    \node at (1.25,1) {Core 0};
    \draw[fill=gold!30] (2.5,0.5) rectangle (4,1.5);
    \node at (3.25,1) {Core 1};
    \draw[fill=gold!30] (0.5,2) rectangle (2,3);
    \node at (1.25,2.5) {Core 2};
    \draw[fill=gold!30] (2.5,2) rectangle (4,3);
    \node at (3.25,2.5) {Core 3};
    \node at (2.5,3.5) {\small Mémoire partagée};

    % Noeud 2
    \draw[thick, burgundy] (7,0) rectangle (12,4);
    \node[above] at (9.5,4) {\textbf{Nœud 2}};
    \draw[fill=gold!30] (7.5,0.5) rectangle (9,1.5);
    \node at (8.25,1) {Core 0};
    \draw[fill=gold!30] (9.5,0.5) rectangle (11,1.5);
    \node at (10.25,1) {Core 1};
    \draw[fill=gold!30] (7.5,2) rectangle (9,3);
    \node at (8.25,2.5) {Core 2};
    \draw[fill=gold!30] (9.5,2) rectangle (11,3);
    \node at (10.25,2.5) {Core 3};
    \node at (9.5,3.5) {\small Mémoire partagée};

    % Réseau
    \draw[<->, thick, dashed] (5,2) -- (7,2);
    \node[above] at (6,2.2) {\small MPI};
    \node[below] at (6,1.8) {\small (réseau)};

    % Labels
    \node[below] at (2.5,-0.3) {\small OpenMP};
    \node[below] at (9.5,-0.3) {\small OpenMP};
\end{tikzpicture}
\caption{Architecture hybride MPI+OpenMP}
\label{fig:hybrid_arch}
\end{figure}

% =============================================================================
\section{OpenMP : parallélisme à mémoire partagée}
\label{sec:prelim:openmp}
% =============================================================================

\subsection{Présentation}

\textbf{OpenMP} (Open Multi-Processing) est une API de programmation parallèle pour les architectures à mémoire partagée. Elle se compose de :
\begin{itemize}
    \item \textbf{Directives de compilation} : pragmas \texttt{\#pragma omp}
    \item \textbf{Fonctions de bibliothèque} : \texttt{omp\_get\_thread\_num()}, etc.
    \item \textbf{Variables d'environnement} : \texttt{OMP\_NUM\_THREADS}, etc.
\end{itemize}

\subsection{Modèle fork-join}

OpenMP utilise le modèle \textbf{fork-join} :
\begin{enumerate}
    \item Le programme démarre avec un thread \textit{master}
    \item À l'entrée d'une région parallèle, le master \textit{fork} : création de threads
    \item Les threads exécutent le travail en parallèle
    \item À la sortie, les threads se synchronisent (\textit{join}) et seul le master continue
\end{enumerate}

\begin{figure}[H]
\centering
\begin{tikzpicture}[scale=0.9]
    % Timeline
    \draw[thick, ->] (0,0) -- (14,0) node[right] {temps};

    % Master thread
    \draw[thick, burgundy] (0,0.5) -- (2,0.5);
    \draw[thick, burgundy] (10,0.5) -- (14,0.5);
    \node[left] at (0,0.5) {Master};

    % Fork
    \draw[thick, burgundy, dashed] (2,0.5) -- (3,2);
    \draw[thick, burgundy, dashed] (2,0.5) -- (3,1);
    \draw[thick, burgundy, dashed] (2,0.5) -- (3,0);
    \draw[thick, burgundy, dashed] (2,0.5) -- (3,-1);

    % Parallel region
    \draw[thick, gold] (3,2) -- (9,2);
    \draw[thick, gold] (3,1) -- (9,1);
    \draw[thick, gold] (3,0) -- (9,0);
    \draw[thick, gold] (3,-1) -- (9,-1);
    \node at (6,2.5) {\small Thread 0};
    \node at (6,1.5) {\small Thread 1};
    \node at (6,-0.5) {\small Thread 2};
    \node at (6,-1.5) {\small Thread 3};

    % Join
    \draw[thick, burgundy, dashed] (9,2) -- (10,0.5);
    \draw[thick, burgundy, dashed] (9,1) -- (10,0.5);
    \draw[thick, burgundy, dashed] (9,0) -- (10,0.5);
    \draw[thick, burgundy, dashed] (9,-1) -- (10,0.5);

    % Labels
    \node[above] at (2,0.7) {\small fork};
    \node[above] at (10,0.7) {\small join};
    \draw[decorate, decoration={brace, amplitude=5pt}] (3,-1.8) -- (9,-1.8);
    \node[below] at (6,-2.3) {Région parallèle};
\end{tikzpicture}
\caption{Modèle fork-join d'OpenMP}
\label{fig:prelim:fork_join}
\end{figure}

\subsection{Directives principales}

\begin{lstlisting}[language=C++, caption={Directives OpenMP fondamentales}]
// Région parallèle de base
#pragma omp parallel
{
    int tid = omp_get_thread_num();
    // Code exécuté par chaque thread
}

// Parallélisation de boucle
#pragma omp parallel for schedule(dynamic, 1)
for (int i = 0; i < n; ++i) {
    process(i);
}

// Section critique (exclusion mutuelle)
#pragma omp critical
{
    shared_variable += local_result;
}

// Réduction
int sum = 0;
#pragma omp parallel for reduction(+:sum)
for (int i = 0; i < n; ++i) {
    sum += array[i];
}
\end{lstlisting}

\subsection{Ordonnancement des boucles}

La clause \texttt{schedule} contrôle la distribution des itérations :

\begin{table}[H]
\centering
\begin{tabular}{lp{8cm}}
\toprule
\textbf{Schedule} & \textbf{Description} \\
\midrule
\texttt{static} & Division égale des itérations, assignation fixe \\
\texttt{dynamic} & Assignation à la demande (équilibrage de charge) \\
\texttt{guided} & Blocs de taille décroissante \\
\texttt{auto} & Choix laissé au compilateur/runtime \\
\bottomrule
\end{tabular}
\caption{Politiques d'ordonnancement OpenMP}
\label{tab:schedule}
\end{table}

Pour notre problème avec des sous-arbres de tailles très variables, \texttt{schedule(dynamic, 1)} est optimal : chaque thread prend une nouvelle tâche dès qu'il termine la précédente.

% =============================================================================
\section{MPI : parallélisme à mémoire distribuée}
\label{sec:prelim:mpi}
% =============================================================================

\subsection{Présentation}

\textbf{MPI} (Message Passing Interface) est le standard de facto pour la programmation parallèle à mémoire distribuée. Chaque processus MPI possède :
\begin{itemize}
    \item Son propre espace mémoire (pas de partage direct)
    \item Un rang unique dans un \textit{communicateur}
    \item La capacité d'envoyer/recevoir des messages
\end{itemize}

\subsection{Communications point-à-point}

Les communications de base sont les envois et réceptions entre deux processus :

\begin{lstlisting}[language=C++, caption={Communications MPI point-à-point}]
int rank, size;
MPI_Comm_rank(MPI_COMM_WORLD, &rank);
MPI_Comm_size(MPI_COMM_WORLD, &size);

if (rank == 0) {
    int data = 42;
    MPI_Send(&data, 1, MPI_INT, 1, 0, MPI_COMM_WORLD);
} else if (rank == 1) {
    int data;
    MPI_Recv(&data, 1, MPI_INT, 0, 0, MPI_COMM_WORLD, MPI_STATUS_IGNORE);
}
\end{lstlisting}

\subsection{Communications collectives}

Les opérations collectives impliquent tous les processus d'un communicateur :

\begin{table}[H]
\centering
\begin{tabular}{lp{8cm}}
\toprule
\textbf{Opération} & \textbf{Description} \\
\midrule
\texttt{MPI\_Bcast} & Diffusion d'un processus vers tous \\
\texttt{MPI\_Scatter} & Distribution de données (1 vers N) \\
\texttt{MPI\_Gather} & Collecte de données (N vers 1) \\
\texttt{MPI\_Reduce} & Réduction avec opération (somme, min, max...) \\
\texttt{MPI\_Allreduce} & Réduction avec résultat sur tous les processus \\
\texttt{MPI\_Barrier} & Synchronisation globale \\
\bottomrule
\end{tabular}
\caption{Opérations collectives MPI}
\label{tab:mpi_collective}
\end{table}

Pour notre problème, \texttt{MPI\_Allreduce} avec l'opération \texttt{MPI\_MIN} permet de propager efficacement la meilleure borne connue à tous les processus.

\subsection{Topologie hypercube}

Une topologie \textbf{hypercube} de dimension $d$ connecte $2^d$ nœuds. Chaque nœud est identifié par un nombre binaire de $d$ bits, et deux nœuds sont voisins si leurs identifiants diffèrent d'exactement un bit.

\begin{figure}[H]
\centering
\begin{tikzpicture}[scale=1.2]
    % Hypercube 3D
    \node[circle, draw, fill=gold!30] (000) at (0,0) {000};
    \node[circle, draw, fill=gold!30] (001) at (2,0) {001};
    \node[circle, draw, fill=gold!30] (010) at (0,2) {010};
    \node[circle, draw, fill=gold!30] (011) at (2,2) {011};
    \node[circle, draw, fill=gold!30] (100) at (1,0.7) {100};
    \node[circle, draw, fill=gold!30] (101) at (3,0.7) {101};
    \node[circle, draw, fill=gold!30] (110) at (1,2.7) {110};
    \node[circle, draw, fill=gold!30] (111) at (3,2.7) {111};

    % Arêtes
    \draw[thick] (000) -- (001) -- (011) -- (010) -- (000);
    \draw[thick] (100) -- (101) -- (111) -- (110) -- (100);
    \draw[thick] (000) -- (100);
    \draw[thick] (001) -- (101);
    \draw[thick] (010) -- (110);
    \draw[thick] (011) -- (111);
\end{tikzpicture}
\caption{Topologie hypercube de dimension 3 (8 nœuds)}
\label{fig:prelim:hypercube}
\end{figure}

Propriétés de l'hypercube :
\begin{itemize}
    \item \textbf{Diamètre} : $d$ (distance maximale entre deux nœuds)
    \item \textbf{Degré} : $d$ (nombre de voisins par nœud)
    \item \textbf{Diffusion} : $O(\log P)$ étapes pour atteindre tous les nœuds
\end{itemize}

% =============================================================================
\section{Métriques de performance parallèle}
\label{sec:prelim:metriques}
% =============================================================================

\subsection{Speedup}

\begin{definition}[Speedup]
Le \textbf{speedup} $S(p)$ mesure l'accélération obtenue avec $p$ processeurs par rapport à l'exécution séquentielle :
\begin{equation}
S(p) = \frac{T_1}{T_p}
\end{equation}
où $T_1$ est le temps séquentiel et $T_p$ le temps avec $p$ processeurs.
\end{definition}

Le speedup idéal (linéaire) est $S(p) = p$. En pratique, on observe $S(p) < p$ à cause des overheads de parallélisation.

\subsection{Efficacité}

\begin{definition}[Efficacité]
L'\textbf{efficacité} $E(p)$ mesure l'utilisation des ressources :
\begin{equation}
E(p) = \frac{S(p)}{p} = \frac{T_1}{p \cdot T_p}
\end{equation}
\end{definition}

Une efficacité de 100\% correspond au speedup linéaire idéal. En pratique, on vise $E(p) > 80\%$.

\subsection{Loi d'Amdahl}

\begin{theorem}[Loi d'Amdahl]
Si une fraction $f$ du code est intrinsèquement séquentielle, le speedup maximal est borné par :
\begin{equation}
S_{max} = \lim_{p \to \infty} S(p) = \frac{1}{f}
\end{equation}
\end{theorem}

\begin{example}
Si 5\% du code est séquentiel ($f = 0.05$), le speedup maximal est $1/0.05 = 20$, quelle que soit le nombre de processeurs.
\end{example}

Cette loi souligne l'importance de paralléliser la plus grande partie possible du code.

\subsection{Loi de Gustafson}

La loi de Gustafson offre une perspective plus optimiste en considérant que la taille du problème augmente avec les ressources :

\begin{theorem}[Loi de Gustafson]
Si on augmente la taille du problème proportionnellement aux ressources, le speedup devient :
\begin{equation}
S(p) = p - f(p-1)
\end{equation}
\end{theorem}

Pour notre problème de recherche de règles de Golomb, augmenter $n$ augmente l'espace de recherche de manière exponentielle, ce qui offre plus de parallélisme (\textit{weak scaling}).

% =============================================================================
\section{Optimisation de code : principes fondamentaux}
\label{sec:prelim:optim}
% =============================================================================

\subsection{Hiérarchie mémoire}

Les processeurs modernes utilisent une hiérarchie de caches pour masquer la latence de la mémoire principale :

\begin{table}[H]
\centering
\begin{tabular}{lccc}
\toprule
\textbf{Niveau} & \textbf{Taille typique} & \textbf{Latence} & \textbf{Bande passante} \\
\midrule
Registres & $\sim$1 Ko & $<$1 ns & $\sim$TB/s \\
Cache L1 & 32-64 Ko & 1-2 ns & $\sim$1 TB/s \\
Cache L2 & 256 Ko - 1 Mo & 3-10 ns & $\sim$500 GB/s \\
Cache L3 & 8-64 Mo & 10-40 ns & $\sim$200 GB/s \\
RAM & 16-256 Go & 50-100 ns & $\sim$50 GB/s \\
\bottomrule
\end{tabular}
\caption{Hiérarchie mémoire typique (AMD EPYC)}
\label{tab:memory_hierarchy}
\end{table}

\subsection{Localité des données}

\begin{definition}[Localité temporelle]
Un programme a une bonne \textbf{localité temporelle} s'il réutilise les données récemment accédées.
\end{definition}

\begin{definition}[Localité spatiale]
Un programme a une bonne \textbf{localité spatiale} s'il accède à des données proches en mémoire.
\end{definition}

Ces principes guident la conception de structures de données efficaces. Notre structure \texttt{BitSet128} (16 octets) est conçue pour tenir entièrement dans les registres CPU.

\subsection{Prédiction de branchement}

Les processeurs modernes utilisent la \textbf{prédiction de branchement} pour anticiper le résultat des conditions. Une mauvaise prédiction coûte 10-20 cycles.

Stratégies d'optimisation :
\begin{itemize}
    \item Éviter les branches dans les boucles critiques
    \item Utiliser des opérations sans branchement (\textit{branchless})
    \item Aider le compilateur avec \texttt{[[likely]]} et \texttt{[[unlikely]]}
\end{itemize}

\begin{lstlisting}[language=C++, caption={Utilisation des attributs de prédiction}]
if (condition) [[unlikely]] {
    // Code rarement exécuté
    handleError();
}
\end{lstlisting}


% Chapitre 3 : Problème des règles de Golomb
\chapter{Problème des règles de Golomb}

\epigraph{\textit{``Mathematics is the art of giving the same name to different things.''}}{--- Henri Poincaré}

\section{Définition formelle, notations et contraintes}

\subsection{Définition d'une règle de Golomb}

\begin{definition}[Règle de Golomb]
Une \textbf{règle de Golomb} d'ordre $n$ est un ensemble de $n$ entiers positifs distincts $G = \{m_0, m_1, \ldots, m_{n-1}\}$ appelés \textit{marques}, avec la convention $m_0 = 0 < m_1 < m_2 < \cdots < m_{n-1}$, tel que toutes les différences $m_j - m_i$ pour $i < j$ sont distinctes.
\end{definition}

Formellement, la \textbf{propriété de Golomb} s'exprime :
\begin{equation}
\forall (i,j,k,l) \in \{0, \ldots, n-1\}^4 \text{ avec } i < j \text{ et } k < l : \quad (i,j) \neq (k,l) \Rightarrow m_j - m_i \neq m_l - m_k
\end{equation}

\subsection{Notations}

Nous adoptons les notations suivantes tout au long de ce rapport :

\begin{table}[H]
\centering
\begin{tabular}{cl}
\toprule
\textbf{Notation} & \textbf{Signification} \\
\midrule
$n$ & Ordre de la règle (nombre de marques) \\
$G_n$ & Règle de Golomb d'ordre $n$ \\
$m_i$ & Position de la $i$-ème marque ($m_0 = 0$) \\
$L(G_n)$ & Longueur de la règle $G_n$, définie par $L(G_n) = m_{n-1}$ \\
$D(G_n)$ & Ensemble des différences : $D(G_n) = \{m_j - m_i : 0 \leq i < j \leq n-1\}$ \\
$OGR(n)$ & Règle de Golomb optimale d'ordre $n$ \\
$L^*(n)$ & Longueur minimale d'une règle de Golomb d'ordre $n$ \\
\bottomrule
\end{tabular}
\caption{Notations utilisées pour les règles de Golomb}
\label{tab:notations}
\end{table}

\subsection{Représentation graphique}

Une règle de Golomb peut être visualisée comme une règle graduée où seules certaines positions sont marquées. La figure \ref{fig:golomb_example} illustre la règle optimale d'ordre 4 : $G_4 = \{0, 1, 4, 6\}$.

\begin{figure}[H]
\centering
\begin{tikzpicture}[scale=0.8]
    % Règle principale
    \draw[thick] (0,0) -- (12,0);

    % Graduations
    \foreach \x in {0,1,...,12} {
        \draw (\x,-0.1) -- (\x,0.1);
        \node[below] at (\x,-0.2) {\tiny \x};
    }

    % Marques de Golomb (positions 0, 1, 4, 6)
    \foreach \x/\label in {0/m_0, 2/m_1, 8/m_2, 12/m_3} {
        \draw[fill=burgundy, burgundy] (\x,-0.3) -- (\x,0.5);
        \draw[fill=burgundy] (\x,0.5) circle (0.15);
    }

    % Annotations des positions
    \node[above] at (0,0.7) {$0$};
    \node[above] at (2,0.7) {$1$};
    \node[above] at (8,0.7) {$4$};
    \node[above] at (12,0.7) {$6$};

    % Différences (arcs)
    \draw[<->, gold, thick] (0,-0.8) -- (2,-0.8) node[midway, below] {\small 1};
    \draw[<->, gold, thick] (2,-1.4) -- (8,-1.4) node[midway, below] {\small 3};
    \draw[<->, gold, thick] (8,-0.8) -- (12,-0.8) node[midway, below] {\small 2};
    \draw[<->, burgundy!70, thick] (0,-2.0) -- (8,-2.0) node[midway, below] {\small 4};
    \draw[<->, burgundy!70, thick] (2,-2.6) -- (12,-2.6) node[midway, below] {\small 5};
    \draw[<->, burgundy!50, thick] (0,-3.2) -- (12,-3.2) node[midway, below] {\small 6};
\end{tikzpicture}
\caption{Règle de Golomb optimale d'ordre 4 : $G_4 = \{0, 1, 4, 6\}$. Les 6 différences (1, 2, 3, 4, 5, 6) sont toutes distinctes.}
\label{fig:golomb_example}
\end{figure}

\subsection{Nombre de différences}

Pour une règle d'ordre $n$, le nombre de paires de marques distinctes est :
\begin{equation}
|D(G_n)| = \binom{n}{2} = \frac{n(n-1)}{2}
\end{equation}

Ce nombre correspond au nombre de différences à vérifier pour s'assurer de l'unicité. Par exemple :
\begin{itemize}
    \item $n = 4$ marques $\Rightarrow$ 6 différences
    \item $n = 7$ marques $\Rightarrow$ 21 différences
    \item $n = 13$ marques $\Rightarrow$ 78 différences
\end{itemize}

\subsection{Contraintes structurelles}

\begin{theorem}[Borne inférieure triviale]
Pour toute règle de Golomb d'ordre $n$, la longueur vérifie :
\begin{equation}
L(G_n) \geq \frac{n(n-1)}{2}
\end{equation}
\end{theorem}

\begin{proof}
Les $\binom{n}{2}$ différences doivent être distinctes et positives. La plus petite configuration possible serait que ces différences soient exactement $\{1, 2, 3, \ldots, \binom{n}{2}\}$. La plus grande différence, qui est la longueur de la règle, est donc au moins $\binom{n}{2}$.
\end{proof}

\begin{remark}
Une règle atteignant cette borne inférieure est dite \textbf{parfaite}. Les règles parfaites sont rares : elles n'existent que pour $n \in \{1, 2, 3, 4\}$.
\end{remark}

\section{Critère d'optimalité}

\subsection{Définition de l'optimalité}

\begin{definition}[Règle de Golomb optimale]
Une règle de Golomb $G_n^*$ d'ordre $n$ est dite \textbf{optimale} si sa longueur est minimale parmi toutes les règles de Golomb du même ordre :
\begin{equation}
L(G_n^*) = L^*(n) = \min\{L(G_n) : G_n \text{ est une règle de Golomb d'ordre } n\}
\end{equation}
\end{definition}

Le problème de recherche de la règle de Golomb optimale (\textit{Optimal Golomb Ruler}, OGR) est un problème d'optimisation combinatoire :

\begin{important}{Problème OGR}
\textbf{Entrée :} Un entier $n \geq 2$. \\
\textbf{Sortie :} Une règle de Golomb $G_n^* = \{0, m_1, \ldots, m_{n-1}\}$ de longueur minimale $L^*(n)$.
\end{important}

\subsection{Complexité du problème}

Le problème OGR est \textbf{NP-difficile}. Cela implique qu'aucun algorithme polynomial n'est connu pour le résoudre. En pratique, la recherche exhaustive de l'optimum requiert un temps exponentiel en $n$.

L'espace de recherche peut être estimé comme suit : pour une règle d'ordre $n$ avec longueur maximale $L_{max}$, le nombre de configurations possibles est de l'ordre de $\binom{L_{max}}{n-1}$, qui croît exponentiellement.

\begin{figure}[H]
\centering
\begin{tikzpicture}[scale=0.9]
    \begin{axis}[
        xlabel={Ordre $n$},
        ylabel={Temps relatif (échelle log)},
        ymode=log,
        grid=major,
        width=10cm,
        height=6cm,
        legend pos=north west,
        xtick={4,5,6,7,8,9,10,11,12,13},
    ]
    \addplot[mark=*, burgundy, thick] coordinates {
        (4, 0.001)
        (5, 0.002)
        (6, 0.01)
        (7, 0.05)
        (8, 0.3)
        (9, 2)
        (10, 15)
        (11, 150)
        (12, 1500)
        (13, 20000)
    };
    \addlegendentry{Temps de calcul}
    \end{axis}
\end{tikzpicture}
\caption{Croissance exponentielle du temps de calcul en fonction de l'ordre $n$}
\label{fig:complexity}
\end{figure}

\subsection{Équivalence et symétrie}

Deux règles de Golomb sont dites \textbf{équivalentes} si l'une peut être obtenue à partir de l'autre par :
\begin{itemize}
    \item \textbf{Translation} : ajout d'une constante à toutes les marques (éliminée par convention $m_0 = 0$) ;
    \item \textbf{Réflexion} : transformation $m_i \mapsto L - m_{n-1-i}$, qui inverse l'ordre des marques.
\end{itemize}

\begin{example}
Les règles $\{0, 1, 4, 6\}$ et $\{0, 2, 5, 6\}$ sont équivalentes par réflexion :
\[
\{0, 1, 4, 6\} \xrightarrow{\text{réflexion}} \{6-6, 6-4, 6-1, 6-0\} = \{0, 2, 5, 6\}
\]
\end{example}

Pour éviter les redondances, on impose souvent la convention $m_1 < L - m_{n-2}$, ce qui sélectionne une forme canonique parmi les deux équivalentes.

\section{Vérification d'une solution}

\subsection{Algorithme de vérification}

La vérification qu'un ensemble de marques forme bien une règle de Golomb consiste à s'assurer que toutes les différences sont distinctes. L'algorithme \ref{alg:verify} présente cette procédure.

\begin{algorithm}[H]
\caption{Vérification d'une règle de Golomb}
\label{alg:verify}
\begin{algorithmic}[1]
\Require Ensemble de marques $M = \{m_0, m_1, \ldots, m_{n-1}\}$ trié
\Ensure \texttt{true} si $M$ est une règle de Golomb, \texttt{false} sinon
\State $seen \gets \emptyset$ \Comment{Ensemble des différences vues}
\For{$i \gets 0$ \textbf{to} $n-2$}
    \For{$j \gets i+1$ \textbf{to} $n-1$}
        \State $d \gets m_j - m_i$
        \If{$d \in seen$}
            \State \Return \texttt{false} \Comment{Différence en double}
        \EndIf
        \State $seen \gets seen \cup \{d\}$
    \EndFor
\EndFor
\State \Return \texttt{true}
\end{algorithmic}
\end{algorithm}

\subsection{Complexité de la vérification}

\begin{itemize}
    \item \textbf{Approche naïve} (avec tableau ou ensemble) : $O(n^2)$ comparaisons, chaque insertion/recherche dans un ensemble haché étant $O(1)$ en moyenne.
    \item \textbf{Approche par bitset} : en utilisant un vecteur de bits indexé par les différences, la vérification reste $O(n^2)$ mais avec des opérations très rapides (accès mémoire contigus).
\end{itemize}

\subsection{Implémentation avec bitset}

Dans notre implémentation, nous utilisons un \texttt{bitset} pour représenter les différences déjà vues. Cette structure permet des opérations en $O(1)$ pour le test et le marquage :

\begin{lstlisting}[language=C++, caption={Vérification avec bitset en C++}]
static inline bool isValid(const std::vector<int>& marks) {
    std::bitset<MAX_DIFF> seen;
    const size_t size = marks.size();

    for (size_t i = 0; i < size; ++i) {
        const int mi = marks[i];
        for (size_t j = i + 1; j < size; ++j) {
            const int d = marks[j] - mi;
            if (d >= MAX_DIFF) return false;
            if (seen[d]) return false;  // Collision!
            seen.set(d);
        }
    }
    return true;
}
\end{lstlisting}

\subsection{Optimisation par décalage de bits}

Une optimisation clé de notre implémentation consiste à calculer \textit{toutes} les nouvelles différences en une seule opération de décalage (\textit{shift}). Cette technique, détaillée au chapitre 4, permet de passer d'une complexité $O(k)$ par marque ajoutée à $O(1)$ :

\begin{figure}[H]
\centering
\begin{tikzpicture}[scale=0.7]
    % Bitset des marques inversées
    \node[anchor=east] at (-0.5, 2) {Marques inversées :};
    \draw (0,1.5) rectangle (12,2.5);
    \foreach \x/\val in {0/1, 2/1, 5/1, 11/0} {
        \node at (\x+0.5, 2) {\val};
    }
    \foreach \x in {0,1,...,11} {
        \draw (\x,1.5) -- (\x,2.5);
    }
    \node[below] at (0.5, 1.4) {\tiny 0};
    \node[below] at (2.5, 1.4) {\tiny 2};
    \node[below] at (5.5, 1.4) {\tiny 5};

    % Flèche
    \draw[->, thick] (6, 1) -- (6, 0);
    \node[right] at (6.2, 0.5) {shift de 3};

    % Bitset après décalage
    \node[anchor=east] at (-0.5, -0.5) {Après shift (offset=3) :};
    \draw (0,-1) rectangle (12,0);
    \foreach \x/\val in {3/1, 5/1, 8/1, 11/0} {
        \node at (\x+0.5, -0.5) {\val};
    }
    \foreach \x in {0,1,...,11} {
        \draw (\x,-1) -- (\x,0);
    }
    \node[below] at (3.5, -1.1) {\tiny 3};
    \node[below] at (5.5, -1.1) {\tiny 5};
    \node[below] at (8.5, -1.1) {\tiny 8};

    % Explication
    \node[align=left, anchor=west] at (13, 0.5) {Nouvelles différences\\calculées en $O(1)$};
\end{tikzpicture}
\caption{Calcul des différences par décalage de bits}
\label{fig:bitshift}
\end{figure}

\section{Résultats optimaux connus}

\subsection{Historique et méthodes de découverte}

Les règles de Golomb optimales ont été découvertes progressivement :
\begin{itemize}
    \item Les petites valeurs ($n \leq 11$) ont été trouvées par des méthodes exhaustives classiques ;
    \item Les valeurs plus grandes ($n \geq 12$) ont nécessité des projets de calcul distribué comme \textit{distributed.net} et des années de calcul ;
    \item La règle optimale pour $n = 27$ (longueur 553) a été prouvée en 2014 après plus de 6 ans de calcul distribué.
\end{itemize}

\subsection{Table des résultats optimaux}

Le tableau \ref{tab:optimal_results} présente les règles de Golomb optimales connues jusqu'à $n = 14$, qui servent de référence pour valider notre implémentation.

\begin{table}[H]
\centering
\begin{tabular}{ccc}
\toprule
\textbf{Ordre $n$} & \textbf{Longueur $L^*(n)$} & \textbf{Règle optimale} \\
\midrule
2 & 1 & $\{0, 1\}$ \\
3 & 3 & $\{0, 1, 3\}$ \\
4 & 6 & $\{0, 1, 4, 6\}$ \\
5 & 11 & $\{0, 1, 4, 9, 11\}$ \\
6 & 17 & $\{0, 1, 4, 10, 12, 17\}$ \\
7 & 25 & $\{0, 1, 4, 10, 18, 23, 25\}$ \\
8 & 34 & $\{0, 1, 4, 9, 15, 22, 32, 34\}$ \\
9 & 44 & $\{0, 1, 5, 12, 25, 27, 35, 41, 44\}$ \\
10 & 55 & $\{0, 1, 6, 10, 23, 26, 34, 41, 53, 55\}$ \\
11 & 72 & $\{0, 1, 4, 13, 28, 33, 47, 54, 64, 70, 72\}$ \\
12 & 85 & $\{0, 2, 6, 24, 29, 40, 43, 55, 68, 75, 76, 85\}$ \\
13 & 106 & $\{0, 2, 5, 25, 37, 43, 59, 70, 85, 89, 98, 99, 106\}$ \\
14 & 127 & $\{0, 4, 6, 20, 35, 52, 59, 77, 78, 86, 89, 99, 122, 127\}$ \\
\bottomrule
\end{tabular}
\caption{Règles de Golomb optimales pour $n = 2$ à $14$}
\label{tab:optimal_results}
\end{table}

\subsection{Observations sur les résultats}

\begin{enumerate}
    \item \textbf{Règles parfaites} : Pour $n \leq 4$, $L^*(n) = \binom{n}{2}$ (borne inférieure atteinte). Au-delà, les règles ne sont plus parfaites.

    \item \textbf{Croissance} : La longueur optimale croît approximativement comme $O(n^2)$, mais le coefficient multiplicatif augmente avec $n$.

    \item \textbf{Non-unicité} : Il peut exister plusieurs règles optimales distinctes (non équivalentes) pour un même ordre. Par exemple, pour $n = 11$, il existe plusieurs règles de longueur 72.
\end{enumerate}

\begin{figure}[H]
\centering
\begin{tikzpicture}[scale=0.8]
    \begin{axis}[
        xlabel={Ordre $n$},
        ylabel={Longueur optimale $L^*(n)$},
        grid=major,
        width=12cm,
        height=7cm,
        legend pos=north west,
        xtick={2,3,4,5,6,7,8,9,10,11,12,13,14},
    ]
    % Longueur optimale
    \addplot[mark=*, burgundy, thick] coordinates {
        (2, 1) (3, 3) (4, 6) (5, 11) (6, 17) (7, 25)
        (8, 34) (9, 44) (10, 55) (11, 72) (12, 85) (13, 106) (14, 127)
    };
    \addlegendentry{$L^*(n)$}

    % Borne inférieure
    \addplot[mark=square*, gold, dashed, thick] coordinates {
        (2, 1) (3, 3) (4, 6) (5, 10) (6, 15) (7, 21)
        (8, 28) (9, 36) (10, 45) (11, 55) (12, 66) (13, 78) (14, 91)
    };
    \addlegendentry{$\binom{n}{2}$ (borne inf.)}
    \end{axis}
\end{tikzpicture}
\caption{Comparaison entre longueur optimale et borne inférieure théorique}
\label{fig:optimal_vs_bound}
\end{figure}

\subsection{Validation de l'implémentation}

Ces résultats servent de \textit{ground truth} pour valider notre implémentation. Une implémentation correcte doit :
\begin{enumerate}
    \item Trouver exactement la longueur optimale $L^*(n)$ pour chaque ordre testé ;
    \item Retourner une règle valide (toutes les différences distinctes) ;
    \item La règle retournée peut différer de celles listées (équivalence ou autre optimal).
\end{enumerate}

\begin{defi}{Critère de validation}
L'implémentation est considérée \textbf{correcte} si et seulement si, pour tout $n$ testé :
\begin{enumerate}
    \item $L_{trouvée}(n) = L^*(n)$
    \item \texttt{isValid}$(G_{trouvée}) = $ \texttt{true}
\end{enumerate}
\end{defi}


% Chapitre 4 : Approche algorithmique séquentielle
\chapter{Approche algorithmique séquentielle}

\epigraph{\textit{``The art of programming is the art of organizing complexity.''}}{--- Edsger W. Dijkstra}

\section{Backtracking et Branch-and-Bound : principe général}

\subsection{Le backtracking}

Le \textbf{backtracking} (retour sur trace) est une technique algorithmique de recherche exhaustive qui explore systématiquement l'espace des solutions possibles en construisant incrémentalement des solutions partielles. Lorsqu'une solution partielle ne peut plus mener à une solution valide, l'algorithme \textit{revient en arrière} pour explorer d'autres branches.

\begin{definition}[Backtracking]
Le backtracking construit un arbre de recherche où :
\begin{itemize}
    \item Chaque \textbf{nœud} représente une solution partielle (un préfixe de règle de Golomb) ;
    \item Chaque \textbf{arête} représente l'ajout d'une nouvelle marque ;
    \item Les \textbf{feuilles} sont soit des solutions complètes, soit des impasses.
\end{itemize}
\end{definition}

Pour le problème OGR, une solution partielle est un ensemble de marques $\{0, m_1, \ldots, m_k\}$ avec $k < n$ qui satisfait déjà la propriété de Golomb (toutes les différences distinctes).

\begin{figure}[H]
\centering
\begin{tikzpicture}[
    level distance=1.8cm,
    sibling distance=3cm,
    edge from parent/.style={draw, -latex},
    every node/.style={draw, rounded corners, minimum width=1.5cm, minimum height=0.7cm, align=center, font=\small},
    level 1/.style={sibling distance=6cm},
    level 2/.style={sibling distance=2.5cm},
    level 3/.style={sibling distance=1.5cm},
    pruned/.style={draw=red!60, fill=red!10, dashed},
    solution/.style={draw=green!60!black, fill=green!20},
]
    \node {$\{0\}$}
        child { node {$\{0,1\}$}
            child { node {$\{0,1,3\}$}
                child { node[solution] {$\{0,1,3,7\}$} edge from parent node[left, draw=none, font=\tiny] {+7} }
                child { node[pruned] {...} edge from parent[dashed, red] }
                edge from parent node[left, draw=none, font=\tiny] {+3}
            }
            child { node {$\{0,1,4\}$}
                child { node[solution] {$\{0,1,4,6\}$} edge from parent node[left, draw=none, font=\tiny] {+6} }
                edge from parent node[right, draw=none, font=\tiny] {+4}
            }
            edge from parent node[left, draw=none, font=\tiny] {+1}
        }
        child { node {$\{0,2\}$}
            child { node {$\{0,2,5\}$}
                child { node[pruned] {...} edge from parent[dashed, red] }
                edge from parent node[left, draw=none, font=\tiny] {+5}
            }
            child { node[pruned] {$\{0,2,6\}$} edge from parent[dashed, red] node[right, draw=none, font=\tiny] {+6} }
            edge from parent node[right, draw=none, font=\tiny] {+2}
        };
\end{tikzpicture}
\caption{Arbre de recherche par backtracking pour $n=4$. Les nœuds rouges sont élagués, les verts sont des solutions.}
\label{fig:backtrack_tree}
\end{figure}

\subsection{Le Branch-and-Bound}

Le \textbf{branch-and-bound} (séparation et évaluation) enrichit le backtracking en ajoutant des \textbf{bornes} permettant d'élaguer (pruning) les branches qui ne peuvent pas mener à une solution meilleure que la meilleure solution connue.

\begin{important}{Principe du Branch-and-Bound}
À chaque nœud de l'arbre de recherche :
\begin{enumerate}
    \item Calculer une \textbf{borne inférieure} $LB$ de la meilleure solution atteignable depuis ce nœud ;
    \item Si $LB \geq$ meilleure solution connue, \textbf{élaguer} cette branche ;
    \item Sinon, \textbf{explorer} les sous-branches (branching).
\end{enumerate}
\end{important}

L'efficacité du branch-and-bound dépend de deux facteurs :
\begin{itemize}
    \item La \textbf{qualité des bornes} : plus elles sont serrées, plus l'élagage est efficace ;
    \item L'\textbf{ordre d'exploration} : trouver rapidement de bonnes solutions améliore les bornes.
\end{itemize}

\subsection{Application au problème OGR}

Pour la recherche de règles de Golomb optimales, l'algorithme procède ainsi :

\begin{enumerate}
    \item Initialiser avec les marques $\{0\}$ et une borne supérieure $bestLen$ (longueur maximale acceptable) ;
    \item Pour chaque position candidate $next > m_{k-1}$ :
    \begin{enumerate}
        \item Vérifier que $next$ ne crée pas de conflit (différence déjà utilisée) ;
        \item Vérifier que la borne inférieure ne dépasse pas $bestLen$ ;
        \item Si valide : ajouter $next$ aux marques et continuer récursivement ;
    \end{enumerate}
    \item Si $k = n$ : on a une solution complète, mettre à jour $bestLen$ si meilleure ;
    \item Sinon : backtrack (retirer $next$ et essayer la position suivante).
\end{enumerate}

\section{Stratégies de pruning}

L'efficacité de la recherche repose sur l'identification rapide des branches non prometteuses. Plusieurs stratégies de pruning sont implémentées.

\subsection{Coupe sur la borne supérieure}

La coupe la plus simple : si la position candidate $next$ atteint ou dépasse la meilleure longueur connue, elle est rejetée.

\begin{algorithm}[H]
\caption{Coupe sur la borne supérieure}
\begin{algorithmic}[1]
\State $upperBound \gets bestLen - 1$ \Comment{On veut strictement mieux}
\For{$next \gets lastMark + 1$ \textbf{to} $upperBound$}
    \State \Comment{Explorer seulement si $next < bestLen$}
\EndFor
\end{algorithmic}
\end{algorithm}

Cette coupe est triviale mais essentielle : elle garantit que toute solution trouvée améliore strictement la meilleure connue.

\subsection{Borne inférieure de Golomb}

Une borne plus sophistiquée utilise le nombre de marques restantes à placer :

\begin{theorem}[Borne inférieure dynamique]
Soit une solution partielle avec $k$ marques et dernière marque à la position $L_k$. S'il reste $r = n - k$ marques à placer, alors la longueur finale minimale est :
\begin{equation}
L_{min} = L_k + \frac{r(r+1)}{2}
\end{equation}
\end{theorem}

\begin{proof}
Les $r$ marques restantes créeront au moins $r$ nouvelles différences (entre chaque nouvelle marque et la dernière). Ces différences minimales sont $1, 2, \ldots, r$, donc l'extension minimale est $1 + 2 + \cdots + r = \frac{r(r+1)}{2}$.
\end{proof}

\begin{figure}[H]
\centering
\begin{tikzpicture}[scale=0.6]
    % Règle actuelle
    \draw[thick] (0,0) -- (16,0);
    \foreach \x in {0,2,4,6,8,10,12,14,16} {
        \draw (\x,-0.1) -- (\x,0.1);
    }

    % Marques existantes
    \foreach \x in {0,2,8} {
        \draw[fill=burgundy, burgundy, thick] (\x,-0.3) -- (\x,0.6);
        \draw[fill=burgundy] (\x,0.6) circle (0.12);
    }

    % Extension minimale
    \draw[<->, gold, thick] (8,-0.8) -- (9,-0.8) node[midway, below] {\small +1};
    \draw[<->, gold, thick] (9,-1.4) -- (11,-1.4) node[midway, below] {\small +2};
    \draw[<->, gold, thick] (11,-0.8) -- (14,-0.8) node[midway, below] {\small +3};

    % Marques futures (pointillées)
    \foreach \x in {9,11,14} {
        \draw[fill=gold!50, gold, thick, dashed] (\x,-0.3) -- (\x,0.6);
        \draw[fill=gold!50, dashed] (\x,0.6) circle (0.12);
    }

    % Annotations
    \node[above] at (0, 0.8) {$0$};
    \node[above] at (2, 0.8) {$m_1$};
    \node[above] at (8, 0.8) {$L_k$};
    \node[above, gold] at (14, 0.8) {$L_{min}$};

    % Formule
    \node[align=center] at (8, -3) {$r = 3$ marques restantes $\Rightarrow$ extension minimale = $\frac{3 \times 4}{2} = 6$};
\end{tikzpicture}
\caption{Illustration de la borne inférieure dynamique avec $r=3$ marques restantes}
\label{fig:lower_bound}
\end{figure}

L'implémentation dans notre code :

\begin{lstlisting}[language=C++, caption={Pruning par borne inférieure}]
const int r = n - numMarks;  // marques restantes
const int minAdditionalLength = (r * (r + 1)) / 2;

if (lastMark + minAdditionalLength >= state.bestLen) {
    // Elaguer: impossible d'améliorer bestLen
    continue;
}
\end{lstlisting}

\subsection{Borne supérieure dynamique}

On peut également borner la position maximale pour la prochaine marque :

\begin{equation}
next_{max} = bestLen - 1 - \frac{(r-1) \cdot r}{2}
\end{equation}

Cette borne anticipe que les $(r-1)$ marques après $next$ devront au minimum ajouter $\frac{(r-1)r}{2}$ à la longueur.

\begin{lstlisting}[language=C++, caption={Calcul de la borne supérieure pour next}]
const int max_remaining = ((r - 1) * r) / 2;
const int max_pos = state.bestLen - max_remaining - 1;

for (int pos = min_pos; pos <= max_pos; ++pos) {
    // Explorer uniquement les positions viables
}
\end{lstlisting}

\subsection{Comparaison des stratégies}

\begin{table}[H]
\centering
\begin{tabular}{lcc}
\toprule
\textbf{Stratégie} & \textbf{Complexité calcul} & \textbf{Efficacité élagage} \\
\midrule
$next < bestLen$ & $O(1)$ & Faible \\
Borne inférieure Golomb & $O(1)$ & Moyenne à forte \\
Borne supérieure dynamique & $O(1)$ & Moyenne \\
Combinaison & $O(1)$ & Forte \\
\bottomrule
\end{tabular}
\caption{Comparaison des stratégies de pruning}
\label{tab:pruning_strategies}
\end{table}

\section{Réduction de symétrie}

\subsection{Principe de la symétrie}

Comme vu au chapitre précédent, chaque règle de Golomb possède une \textbf{règle miroir} équivalente obtenue par réflexion. Explorer les deux est redondant et double le temps de calcul.

\begin{example}
$\{0, 1, 4, 6\}$ et $\{0, 2, 5, 6\}$ sont miroirs :
\begin{align*}
\{0, 1, 4, 6\} &\xrightarrow{\text{miroir}} \{6-6, 6-4, 6-1, 6-0\} = \{0, 2, 5, 6\}
\end{align*}
\end{example}

\subsection{Stratégie de réduction}

Pour éviter d'explorer les paires de règles miroirs, nous imposons des contraintes qui sélectionnent une seule forme canonique.

\subsubsection{Contrainte sur la première marque}

\begin{theorem}[Symmetry Breaking 1]
Pour toute règle de Golomb $\{0, m_1, \ldots, m_{n-1}\}$ et son miroir, au moins une satisfait :
\begin{equation}
m_1 \leq \frac{L}{2}
\end{equation}
où $L = m_{n-1}$ est la longueur de la règle.
\end{theorem}

En pratique, on restreint la boucle sur $m_1$ (appelée \texttt{firstMark}) :

\begin{lstlisting}[language=C++, caption={Symmetry breaking sur la première marque}]
// SYMMETRY BREAKING: a_1 <= bestLen/2
for (int firstMark = 1; firstMark <= state.bestLen / 2; ++firstMark) {
    // Explorer seulement la moitie gauche
}
\end{lstlisting}

Cette contrainte élimine environ \textbf{50\%} des branches explorées.

\subsubsection{Contrainte sur la dernière marque}

Une contrainte plus fine vérifie au moment de compléter une solution :

\begin{theorem}[Symmetry Breaking 2]
Une règle $\{0, m_1, \ldots, m_{n-1}\}$ est en forme canonique si :
\begin{equation}
m_1 < m_{n-1} - m_{n-2}
\end{equation}
\end{theorem}

Cette condition compare le premier écart ($m_1 - 0 = m_1$) au dernier écart ($m_{n-1} - m_{n-2}$). La règle canonique est celle dont le premier écart est plus petit.

\begin{lstlisting}[language=C++, caption={Symmetry breaking à la solution}]
if (newNumMarks == n) {
    // Symmetry breaking: m_1 < m_{n-1} - m_{n-2}
    // Equivalent: next > lastMark + marks[1]
    if (next <= lastMark + marks[1]) {
        continue;  // Skip le miroir
    }
    // Accepter cette solution
}
\end{lstlisting}

\subsection{Impact attendu}

La réduction de symétrie divise théoriquement l'espace de recherche par 2. En pratique, le gain dépend de la distribution des solutions :

\begin{itemize}
    \item \textbf{Gain maximal} : si les solutions sont uniformément distribuées, speedup $\approx 2\times$ ;
    \item \textbf{Gain réel} : souvent entre $1.5\times$ et $2\times$ car certaines branches miroirs sont déjà élaguées par d'autres prunings.
\end{itemize}

\begin{figure}[H]
\centering
\begin{tikzpicture}[scale=0.8]
    % Sans symétrie
    \draw[fill=burgundy!30] (0,0) rectangle (3,4);
    \node at (1.5, 4.3) {Sans réduction};
    \node at (1.5, 2) {100\%};

    % Avec symétrie
    \draw[fill=gold!30] (5,0) rectangle (8,2);
    \draw[fill=gray!20, pattern=north east lines] (5,2) rectangle (8,4);
    \node at (6.5, 4.3) {Avec réduction};
    \node at (6.5, 1) {$\sim$50\%};
    \node[gray] at (6.5, 3) {éliminé};

    % Légende
    \draw[fill=gold!30] (10,3.5) rectangle (10.5,4);
    \node[right] at (10.6, 3.75) {Exploré};
    \draw[fill=gray!20, pattern=north east lines] (10,2.5) rectangle (10.5,3);
    \node[right] at (10.6, 2.75) {Éliminé (miroir)};
\end{tikzpicture}
\caption{Impact de la réduction de symétrie sur l'espace de recherche}
\label{fig:symmetry_impact}
\end{figure}

\section{Représentation binaire des distances}

\subsection{Motivation}

La vérification des collisions (une nouvelle différence existe-t-elle déjà ?) est l'opération la plus fréquente. L'utilisation d'un \textbf{bitset} permet des opérations en temps constant.

\subsection{Structure \texttt{bitset} classique}

Un bitset de taille $M$ (où $M$ est la longueur maximale) représente les différences utilisées :
\begin{itemize}
    \item Le bit $d$ est à 1 si la différence $d$ est déjà utilisée ;
    \item Test de collision : $O(1)$ avec \texttt{seen[d]} ;
    \item Ajout d'une différence : $O(1)$ avec \texttt{seen.set(d)}.
\end{itemize}

\begin{lstlisting}[language=C++, caption={Opérations sur bitset standard}]
std::bitset<MAX_DIFF> usedDiffs;

// Test de collision
if (usedDiffs[d]) {
    // Collision! Cette difference existe deja
}

// Marquer une difference
usedDiffs.set(d);
\end{lstlisting}

\subsection{Structure \texttt{BitSet128} optimisée}

Pour maximiser les performances, nous utilisons une structure personnalisée basée sur deux \texttt{uint64\_t} :

\begin{lstlisting}[language=C++, caption={Structure BitSet128}]
struct alignas(16) BitSet128 {
    uint64_t lo;  // bits 0-63
    uint64_t hi;  // bits 64-127

    // Set bit at position
    inline void set(int pos) {
        if (pos < 64) lo |= (1ULL << pos);
        else          hi |= (1ULL << (pos - 64));
    }

    // Test bit at position
    inline bool test(int pos) const {
        if (pos < 64) return (lo >> pos) & 1;
        else          return (hi >> (pos - 64)) & 1;
    }

    // Check if any bit is set
    inline bool any() const {
        return (lo | hi) != 0;
    }
};
\end{lstlisting}

\subsection{Optimisation clé : détection de collision en O(1)}

L'innovation majeure de la version V2 est l'utilisation du \textbf{décalage de bits} pour calculer toutes les nouvelles différences en une seule opération.

\subsubsection{Encodage des marques inversées}

Au lieu de stocker les positions des marques, on encode leur position \textit{relative à la longueur courante} :

\begin{defi}{Reversed Marks}
Pour une règle partielle $\{0, m_1, \ldots, m_k\}$ de longueur $L_k = m_k$, le bitset \texttt{reversed\_marks} a le bit $i$ à 1 si et seulement si il existe une marque à la position $L_k - i$.
\end{defi}

\begin{example}
Pour $\{0, 1, 4, 9\}$ avec $L_k = 9$ :
\begin{itemize}
    \item Marque 0 $\rightarrow$ bit à position $9 - 0 = 9$
    \item Marque 1 $\rightarrow$ bit à position $9 - 1 = 8$
    \item Marque 4 $\rightarrow$ bit à position $9 - 4 = 5$
    \item Marque 9 $\rightarrow$ bit à position $9 - 9 = 0$
\end{itemize}
\texttt{reversed\_marks} = bits aux positions $\{0, 5, 8, 9\}$
\end{example}

\subsubsection{Calcul des différences par shift}

Lorsqu'on ajoute une marque à la position $pos$, le décalage $offset = pos - L_k$ permet de calculer toutes les nouvelles différences :

\begin{equation}
\texttt{new\_dist} = \texttt{reversed\_marks} \ll offset
\end{equation}

\begin{theorem}[Correction du shift]
Si le bit $i$ est à 1 dans \texttt{reversed\_marks} (marque à $L_k - i$), alors le bit $i + offset$ sera à 1 dans \texttt{new\_dist}. Or $i + offset = i + (pos - L_k) = pos - (L_k - i)$, qui est exactement la différence entre la nouvelle marque et l'ancienne.
\end{theorem}

\begin{figure}[H]
\centering
\begin{tikzpicture}[scale=0.5]
    % reversed_marks avant shift
    \node[anchor=east] at (-1, 3) {\small reversed\_marks :};
    \draw (0,2.5) rectangle (16,3.5);
    \foreach \x in {0,1,...,15} {
        \draw (\x,2.5) -- (\x,3.5);
        \node[above, font=\tiny] at (\x+0.5, 3.5) {\x};
    }
    % Bits actifs: positions 0, 5, 8, 9
    \foreach \x in {0,5,8,9} {
        \fill[burgundy] (\x+0.1,2.6) rectangle (\x+0.9,3.4);
    }

    % Flèche
    \draw[->, thick] (8, 2) -- (8, 1) node[midway, right] {\small $\ll 3$};

    % new_dist après shift
    \node[anchor=east] at (-1, 0) {\small new\_dist :};
    \draw (0,-0.5) rectangle (16,0.5);
    \foreach \x in {0,1,...,15} {
        \draw (\x,-0.5) -- (\x,0.5);
    }
    % Bits décalés: positions 3, 8, 11, 12
    \foreach \x in {3,8,11,12} {
        \fill[gold] (\x+0.1,-0.4) rectangle (\x+0.9,0.4);
    }

    % Explication
    \node[align=left, anchor=west] at (17, 1.5) {\small Nouvelles différences :\\  \small $\{3, 8, 11, 12\}$};
\end{tikzpicture}
\caption{Calcul des différences par décalage : ajout d'une marque avec $offset = 3$}
\label{fig:shift_operation}
\end{figure}

\subsubsection{Test de collision}

Une fois \texttt{new\_dist} calculé, le test de collision est un simple AND :

\begin{lstlisting}[language=C++, caption={Test de collision O(1)}]
BitSet128 new_dist = reversed_marks << offset;

// Collision si une difference existe deja
if ((new_dist & used_dist).any()) {
    continue;  // Rejeter ce candidat
}
\end{lstlisting}

\subsection{Invariants maintenus}

À tout instant de la recherche, les invariants suivants sont maintenus :

\begin{enumerate}
    \item \texttt{used\_dist} contient exactement les $\binom{k}{2}$ différences des $k$ marques placées ;
    \item \texttt{reversed\_marks} encode les $k$ marques relativement à la longueur courante ;
    \item Aucune collision n'existe : \texttt{(reversed\_marks \& used\_dist) == 0}.
\end{enumerate}

\section{Pseudo-code de la recherche}

\subsection{Version récursive conceptuelle}

L'algorithme peut être exprimé récursivement de façon claire :

\begin{algorithm}[H]
\caption{Recherche OGR par backtracking récursif}
\label{alg:recursive}
\begin{algorithmic}[1]
\Procedure{Search}{$marks$, $usedDist$, $n$, $bestLen$}
    \State $k \gets |marks|$
    \State $lastMark \gets marks[k-1]$

    \Statex
    \State \Comment{=== PRUNING: Borne inférieure ===}
    \State $r \gets n - k$ \Comment{Marques restantes}
    \If{$lastMark + \frac{r(r+1)}{2} \geq bestLen$}
        \State \Return \Comment{Élaguer cette branche}
    \EndIf

    \Statex
    \State \Comment{=== EXPLORATION ===}
    \For{$next \gets lastMark + 1$ \textbf{to} $bestLen - 1$}

        \Statex
        \State \Comment{Test de validité (différences)}
        \State $valid \gets \textsc{True}$
        \For{$i \gets 0$ \textbf{to} $k-1$}
            \State $d \gets next - marks[i]$
            \If{$d \in usedDist$}
                \State $valid \gets \textsc{False}$
                \State \textbf{break}
            \EndIf
        \EndFor

        \Statex
        \If{$\neg valid$}
            \State \textbf{continue}
        \EndIf

        \Statex
        \State \Comment{=== CANDIDAT VALIDE ===}
        \If{$k + 1 = n$}
            \State \Comment{Solution complète trouvée}
            \If{$next < bestLen$ \textbf{and} \textsc{IsCanonical}($marks$, $next$)}
                \State $bestLen \gets next$
                \State \textsc{SaveSolution}($marks$, $next$)
            \EndIf
        \Else
            \State \Comment{Récursion}
            \State $newDists \gets \{next - marks[i] : i \in [0, k-1]\}$
            \State \textsc{Search}($marks \cup \{next\}$, $usedDist \cup newDists$, $n$, $bestLen$)
        \EndIf
    \EndFor
\EndProcedure
\end{algorithmic}
\end{algorithm}

\subsection{Version itérative avec pile}

Pour éviter l'overhead des appels de fonction, notre implémentation utilise une \textbf{pile manuelle} :

\begin{algorithm}[H]
\caption{Recherche OGR itérative avec pile}
\label{alg:iterative}
\begin{algorithmic}[1]
\Procedure{SearchIterative}{$n$, $maxLen$}
    \State $stack \gets$ pile pré-allouée de $n$ frames
    \State $bestLen \gets maxLen + 1$

    \Statex
    \State \Comment{=== SYMMETRY BREAKING: $m_1 \leq bestLen/2$ ===}
    \For{$firstMark \gets 1$ \textbf{to} $bestLen / 2$}
        \State Initialiser $stack[0]$ avec $\{0, firstMark\}$
        \State $stackTop \gets 0$

        \Statex
        \While{$stackTop \geq 0$}
            \State $frame \gets stack[stackTop]$

            \Statex
            \State \Comment{Pruning}
            \If{borne inférieure $\geq bestLen$}
                \State $stackTop \gets stackTop - 1$
                \State \textbf{continue}
            \EndIf

            \Statex
            \State \Comment{Exploration des candidats}
            \State $pushed \gets \textsc{False}$
            \For{$next \gets frame.nextCandidate$ \textbf{to} $upperBound$}
                \If{\textsc{HasCollision}($frame$, $next$)}
                    \State \textbf{continue}
                \EndIf

                \Statex
                \If{$frame.numMarks + 1 = n$}
                    \State Mettre à jour $bestLen$ si amélioration
                \Else
                    \State $frame.nextCandidate \gets next + 1$
                    \State Push nouveau frame vers $stack[stackTop + 1]$
                    \State $stackTop \gets stackTop + 1$
                    \State $pushed \gets \textsc{True}$
                    \State \textbf{break}
                \EndIf
            \EndFor

            \Statex
            \If{$\neg pushed$}
                \State $stackTop \gets stackTop - 1$ \Comment{Backtrack}
            \EndIf
        \EndWhile
    \EndFor

    \State \Return meilleure solution
\EndProcedure
\end{algorithmic}
\end{algorithm}

\subsection{Placement des tests}

Le schéma suivant résume où interviennent les différents tests dans la boucle principale :

\begin{figure}[H]
\centering
\begin{tikzpicture}[
    node distance=0.8cm,
    startstop/.style={rectangle, rounded corners, draw=burgundy, fill=burgundy!20, text width=4cm, align=center, minimum height=0.8cm},
    process/.style={rectangle, draw=gold!80!black, fill=gold!20, text width=5cm, align=center, minimum height=0.8cm},
    decision/.style={diamond, draw=burgundy, fill=burgundy!10, text width=2.5cm, align=center, aspect=2},
    arrow/.style={thick,->,>=stealth}
]
    \node[startstop] (start) {Début frame};
    \node[decision, below=of start] (bound) {Borne inf. $\geq$ bestLen ?};
    \node[startstop, right=2cm of bound] (prune1) {Élaguer (pop)};
    \node[process, below=of bound] (loop) {Pour chaque $next$};
    \node[decision, below=of loop] (collision) {Collision ?};
    \node[startstop, right=2cm of collision] (skip) {Skip (continue)};
    \node[decision, below=of collision] (complete) {Solution complète ?};
    \node[process, right=2cm of complete] (update) {Maj bestLen};
    \node[process, below=of complete] (push) {Push nouveau frame};

    \draw[arrow] (start) -- (bound);
    \draw[arrow] (bound) -- node[above] {oui} (prune1);
    \draw[arrow] (bound) -- node[right] {non} (loop);
    \draw[arrow] (loop) -- (collision);
    \draw[arrow] (collision) -- node[above] {oui} (skip);
    \draw[arrow] (collision) -- node[right] {non} (complete);
    \draw[arrow] (complete) -- node[above] {oui} (update);
    \draw[arrow] (complete) -- node[right] {non} (push);

    % Retours
    \draw[arrow, dashed] (skip) |- (loop);
    \draw[arrow, dashed] (update) |- (loop);
\end{tikzpicture}
\caption{Flux de contrôle et placement des tests dans la boucle de recherche}
\label{fig:control_flow}
\end{figure}

\subsection{Complexité}

\begin{itemize}
    \item \textbf{Pire cas} : $O(L^n)$ où $L$ est la longueur maximale (exploration exhaustive) ;
    \item \textbf{Cas moyen} : drastiquement réduit par le pruning, mais reste exponentiel ;
    \item \textbf{Espace} : $O(n)$ pour la pile (profondeur maximale = $n$ niveaux).
\end{itemize}

Les optimisations (bitset, pruning, symétrie) ne changent pas la complexité asymptotique mais réduisent considérablement les constantes, permettant de traiter des instances plus grandes en temps raisonnable.


% Chapitre 5 : Implémentation C++ et ingénierie logicielle
\chapter{Implémentation C++ et ingénierie logicielle}

\epigraph{\textit{``Any fool can write code that a computer can understand. Good programmers write code that humans can understand.''}}{--- Martin Fowler}

\section{Organisation du dépôt}

Le projet est organisé selon une architecture modulaire séparant clairement les interfaces, les implémentations et les points d'entrée.

\subsection{Structure des répertoires}

\begin{lstlisting}[language=bash, caption={Arborescence du projet}]
OptimalGolombRuler/
├── include/               # Headers (interfaces)
│   ├── golomb.hpp            # Structure GolombRuler
│   ├── search.hpp            # Interface OpenMP V1
│   ├── search_v2.hpp         # Interface OpenMP V2
│   ├── search_v3.hpp         # Interface OpenMP V3
│   ├── search_v4.hpp         # Interface OpenMP V4
│   ├── search_v5.hpp         # Interface OpenMP V5
│   ├── search_sequential.hpp # Interface Sequential V1
│   ├── search_sequential_v2.hpp # Interface Sequential V2
│   ├── search_mpi.hpp        # Interface MPI V1
│   ├── search_mpi_v2.hpp     # Interface MPI V2
│   ├── search_mpi_v3.hpp     # Interface MPI V3
│   ├── hypercube.hpp         # Topologie hypercube MPI
│   └── benchmark_log.hpp     # Export CSV benchmarks
├── src/                   # Sources (implementations)
│   ├── search.cpp            # OpenMP V1
│   ├── search_v2.cpp         # OpenMP V2
│   ├── search_v3.cpp         # OpenMP V3
│   ├── search_v4.cpp         # OpenMP V4
│   ├── search_v5.cpp         # OpenMP V5
│   ├── search_sequential.cpp # Sequential V1
│   ├── search_sequential_v2.cpp # Sequential V2
│   ├── search_mpi.cpp        # MPI V1
│   ├── search_mpi_v2.cpp     # MPI V2
│   ├── search_mpi_v3.cpp     # MPI V3
│   ├── main_*.cpp            # Points d'entree
│   └── test_correctness.cpp  # Tests de validation
├── scripts/               # Scripts Windows (MSVC)
├── benchmarks/            # Resultats CSV
├── *.slurm                # Scripts SLURM (HPC Romeo)
└── Makefile               # Build Linux/Unix
\end{lstlisting}

\subsection{Modules et responsabilités}

Le tableau \ref{tab:modules} détaille la responsabilité de chaque module :

\begin{table}[H]
\centering
\begin{tabularx}{\textwidth}{lX}
\toprule
\textbf{Module} & \textbf{Responsabilité} \\
\midrule
\texttt{golomb.hpp} & Structure de données \texttt{GolombRuler} : stockage des marques, calcul de longueur, validation statique \texttt{isValid()} \\
\midrule
\texttt{search*.hpp/cpp} & Algorithmes de recherche : backtracking, pruning, gestion de la pile. Chaque version (V1-V5) représente une itération d'optimisation \\
\midrule
\texttt{search\_mpi*.hpp/cpp} & Versions distribuées avec MPI : distribution des préfixes, synchronisation des bornes via hypercube ou \texttt{MPI\_Allreduce} \\
\midrule
\texttt{hypercube.hpp} & Topologie de communication MPI en hypercube : calcul des voisins, diffusion en $O(\log P)$ \\
\midrule
\texttt{benchmark\_log.hpp} & Logging des résultats en CSV : timestamp, paramètres, métriques de performance \\
\midrule
\texttt{main\_*.cpp} & Points d'entrée : parsing des arguments, initialisation, appel à la recherche, affichage des résultats \\
\midrule
\texttt{test\_correctness.cpp} & Suite de tests : validation contre solutions connues, tests de structure, tests de reproductibilité \\
\bottomrule
\end{tabularx}
\caption{Modules du projet et leurs responsabilités}
\label{tab:modules}
\end{table}

\subsection{Conventions de nommage des versions}

Les différentes versions sont identifiées par un suffixe numérique reflétant l'évolution des optimisations :

\begin{table}[H]
\centering
\begin{tabular}{llp{7cm}}
\toprule
\textbf{Catégorie} & \textbf{Version} & \textbf{Caractéristiques} \\
\midrule
\multirow{2}{*}{Sequential} & V1 & Itératif, loop unrolling 4x \\
 & V2 & BitSet128 shift O(1) \\
\midrule
\multirow{5}{*}{OpenMP} & V1 & Itératif + loop unrolling \\
 & V2 & Récursif + bitset<256> shift \\
 & V3 & Hybride itératif + bitset shift \\
 & V4 & Génération de préfixes + bitset shift \\
 & V5 & BitSet128 (2$\times$uint64\_t) + préfixes \\
\midrule
\multirow{3}{*}{MPI+OpenMP} & V1 & Hypercube + loop unrolling \\
 & V2 & Hypercube + BitSet128 shift \\
 & V3 & MPI\_Allreduce + BitSet128 \\
\bottomrule
\end{tabular}
\caption{Versions implémentées et leurs optimisations}
\label{tab:versions}
\end{table}

\subsection{Structure \texttt{GolombRuler}}

La structure centrale du projet encapsule une règle de Golomb :

\begin{lstlisting}[language=C++, caption={Structure GolombRuler (golomb.hpp)}]
struct GolombRuler {
    std::vector<int> marks;  // Positions des marques
    int length = 0;          // Longueur (= derniere marque)

    // Validation statique
    static inline bool isValid(const std::vector<int>& marks) {
        std::bitset<MAX_DIFF> seen;
        for (size_t i = 0; i < marks.size(); ++i) {
            for (size_t j = i + 1; j < marks.size(); ++j) {
                const int d = marks[j] - marks[i];
                if (seen[d]) return false;  // Collision
                seen.set(d);
            }
        }
        return true;
    }

    void computeLength() noexcept {
        length = marks.empty() ? 0 : marks.back();
    }
};
\end{lstlisting}

\section{Prérequis, compilation et exécution}

\subsection{Prérequis logiciels}

\begin{table}[H]
\centering
\begin{tabular}{lll}
\toprule
\textbf{Composant} & \textbf{Linux/Unix} & \textbf{Windows} \\
\midrule
Compilateur C++ & g++ 10+ ou clang++ 12+ & MSVC 2019+ \\
Standard C++ & C++20 & C++20 \\
OpenMP & libgomp (inclus avec g++) & Supporté nativement \\
MPI & OpenMPI ou MPICH & MS-MPI \\
Make & GNU Make & Optionnel (scripts .bat) \\
\bottomrule
\end{tabular}
\caption{Prérequis selon la plateforme}
\label{tab:prerequisites}
\end{table}

\subsection{Compilation sous Linux/Unix (Makefile)}

Le \texttt{Makefile} fournit des cibles pour chaque version :

\begin{lstlisting}[language=bash, caption={Commandes de compilation Linux}]
# Versions sequentielles
make sequential           # V1 (loop unrolling)
make sequential_v2        # V2 (BitSet128 shift)

# Versions OpenMP
make openmp               # V1
make openmp_v2            # V2
make openmp_v3            # V3
make openmp_v4            # V4
make openmp_v5            # V5 (recommande)

# Versions MPI+OpenMP
make mpi                  # V1 (hypercube)
make mpi_v2               # V2 (hypercube + BitSet128)
make mpi_v3               # V3 (allreduce + BitSet128)

# Modes developpement (tailles reduites)
make sequential-dev
make openmp-dev
make mpi-dev

# Nettoyage
make clean
\end{lstlisting}

Les flags d'optimisation appliqués sont :

\begin{lstlisting}[language=bash, caption={Flags de compilation}]
OPTFLAGS = -O3 -march=native -mtune=native \
           -funroll-loops -fomit-frame-pointer -flto
CXXFLAGS = -std=c++20 $(OPTFLAGS) -fopenmp -I$(INC_DIR) \
           -Wall -Wextra -DNDEBUG
\end{lstlisting}

\begin{itemize}
    \item \texttt{-O3} : optimisations agressives
    \item \texttt{-march=native} : instructions spécifiques au CPU hôte
    \item \texttt{-funroll-loops} : déroulage automatique des boucles
    \item \texttt{-flto} : optimisation à l'édition de liens (LTO)
    \item \texttt{-DNDEBUG} : désactive les assertions (mode production)
\end{itemize}

\subsection{Compilation sous Windows (scripts MSVC)}

Des scripts batch sont fournis pour la compilation avec MSVC :

\begin{lstlisting}[language=bash, caption={Scripts de compilation Windows}]
scripts\build_sequential.bat     # Sequential V1
scripts\build_sequential_v2.bat  # Sequential V2
scripts\build_openmp.bat         # OpenMP V1
scripts\build_openmp_v5.bat      # OpenMP V5
scripts\build_mpi.bat            # MPI V1
scripts\clean.bat                # Nettoyage
\end{lstlisting}

\subsection{Exécution}

\subsubsection{Versions séquentielles et OpenMP}

\begin{lstlisting}[language=bash, caption={Exécution locale}]
# Sequential
./build/golomb_sequential <n>
./build/golomb_sequential_v2 <n>

# OpenMP (utilise tous les cores)
./build/golomb_openmp_v5 <n> [prefix_depth]

# Controler le nombre de threads
OMP_NUM_THREADS=8 ./build/golomb_openmp_v5 13
\end{lstlisting}

Les arguments sont :
\begin{itemize}
    \item \texttt{<n>} : nombre de marques (obligatoire, entre 2 et 20)
    \item \texttt{[prefix\_depth]} : profondeur de préfixe pour la parallélisation (optionnel, auto par défaut)
\end{itemize}

\subsubsection{Versions MPI}

\begin{lstlisting}[language=bash, caption={Exécution MPI}]
# V1 et V2 : nombre de processus = puissance de 2 (hypercube)
mpiexec -n 4 ./build/golomb_mpi_v2 13
mpiexec -n 8 ./build/golomb_mpi_v2 13

# V3 : tout nombre de processus
mpiexec -n 6 ./build/golomb_mpi_v3 13
\end{lstlisting}

\subsubsection{Exécution sur HPC (SLURM)}

Des scripts SLURM sont fournis pour le supercalculateur Romeo :

\begin{lstlisting}[language=bash, caption={Soumission de jobs SLURM}]
# Comparaison OpenMP (x86 et ARM)
sbatch golomb_openmp_compare.slurm
sbatch golomb_openmp_compare_arm.slurm

# Comparaison MPI
sbatch golomb_mpi_compare.slurm

# Comparaison sequentielle
sbatch golomb_sequential_compare.slurm
\end{lstlisting}

\section{Format des sorties et traçage benchmark}

\subsection{Sortie console standard}

L'exécution produit une sortie formatée sur la console :

\begin{lstlisting}[language=bash, caption={Exemple de sortie console}]
=============================================================
       OPTIMAL GOLOMB RULER - OPENMP V5 (n=13)
=============================================================
Algorithm: uint64_t ops + prefix-based + iterative
Threads: 12
Prefix depth: auto

n          : 13
Length     : 106
Time       : 45.234 s
States     : 1234567890
States/sec : 2.73e+07
Valid      : YES

Ruler: { 0, 2, 5, 25, 37, 43, 59, 70, 85, 89, 98, 99, 106 }
=============================================================
\end{lstlisting}

\subsection{Format CSV pour benchmarks}

La classe \texttt{BenchmarkLog} génère des fichiers CSV dans le répertoire \texttt{benchmarks/}. Deux formats sont utilisés selon le type de parallélisation.

\subsubsection{Format OpenMP}

\begin{table}[H]
\centering
\begin{tabular}{lp{8cm}}
\toprule
\textbf{Champ} & \textbf{Description} \\
\midrule
\texttt{timestamp} & Date et heure (YYYY-MM-DD HH:MM:SS) \\
\texttt{date} & Date seule (YYYY-MM-DD) \\
\texttt{n} & Nombre de marques \\
\texttt{threads} & Nombre de threads OpenMP \\
\texttt{length} & Longueur de la règle trouvée \\
\texttt{time\_s} & Temps d'exécution en secondes \\
\texttt{speedup} & Accélération par rapport à 1 thread \\
\texttt{efficiency\_pct} & Efficacité parallèle (\%) \\
\texttt{states} & Nombre d'états explorés \\
\texttt{changes} & Notes/commentaires (optionnel) \\
\bottomrule
\end{tabular}
\caption{Champs du fichier CSV OpenMP}
\label{tab:csv_openmp}
\end{table}

Exemple de ligne :
\begin{lstlisting}[basicstyle=\tiny\ttfamily]
2025-01-15 14:30:22,2025-01-15,13,12,106,45.23400,8.45,70.4,1234567890,"V5 prefix-based"
\end{lstlisting}

\subsubsection{Format MPI}

\begin{table}[H]
\centering
\begin{tabular}{lp{8cm}}
\toprule
\textbf{Champ} & \textbf{Description} \\
\midrule
\texttt{timestamp} & Date et heure \\
\texttt{date} & Date seule \\
\texttt{n} & Nombre de marques \\
\texttt{mpi\_procs} & Nombre de processus MPI \\
\texttt{omp\_threads} & Threads OpenMP par processus \\
\texttt{length} & Longueur trouvée \\
\texttt{time\_s} & Temps d'exécution \\
\texttt{speedup} & Accélération \\
\texttt{efficiency\_pct} & Efficacité \\
\texttt{states} & États explorés \\
\texttt{changes} & Notes \\
\bottomrule
\end{tabular}
\caption{Champs du fichier CSV MPI}
\label{tab:csv_mpi}
\end{table}

\subsubsection{Utilisation de BenchmarkLog}

\begin{lstlisting}[language=C++, caption={Utilisation de BenchmarkLog}]
#include "benchmark_log.hpp"

// Creation du logger
BenchmarkLog log("benchmarks", "openmp");

// Enregistrement d'un resultat OpenMP
log.logOpenMP(
    n,              // Nombre de marques
    numThreads,     // Threads utilises
    best.length,    // Longueur trouvee
    elapsed,        // Temps en secondes
    speedup,        // T_1 / T_p
    efficiency,     // speedup / p * 100
    exploredStates, // Etats explores
    "V5 BitSet128"  // Commentaire
);

// Enregistrement d'un resultat MPI
log.logMPI(
    n, mpiProcs, ompThreads, length,
    time, speedup, efficiency, states, "V3 Allreduce"
);
\end{lstlisting}

\subsection{Emplacement des fichiers}

\begin{lstlisting}[language=bash]
benchmarks/
├── openmp_benchmark.csv      # Resultats OpenMP
├── mpi_benchmark.csv         # Resultats MPI
├── sequential_benchmark.csv  # Resultats sequentiels
└── compare_v1_v2.csv         # Comparaisons de versions
\end{lstlisting}

\section{Stratégie de validation}

La validation de l'implémentation est essentielle pour garantir la correction des résultats. Notre stratégie repose sur plusieurs niveaux de vérification.

\subsection{Tests sur petits $n$}

Pour les petites valeurs de $n$ ($n \leq 12$), l'exécution est rapide et permet une validation systématique :

\begin{lstlisting}[language=C++, caption={Référentiel des solutions optimales connues}]
const std::vector<KnownOptimal> KNOWN_OPTIMALS = {
    {2,  1,   {0, 1}},
    {3,  3,   {0, 1, 3}},
    {4,  6,   {0, 1, 4, 6}},
    {5,  11,  {0, 1, 4, 9, 11}},
    {6,  17,  {0, 1, 4, 10, 12, 17}},
    {7,  25,  {0, 1, 4, 10, 18, 23, 25}},
    {8,  34,  {0, 1, 4, 9, 15, 22, 32, 34}},
    {9,  44,  {0, 1, 5, 12, 25, 27, 35, 41, 44}},
    {10, 55,  {0, 1, 6, 10, 23, 26, 34, 41, 53, 55}},
    {11, 72,  {0, 1, 4, 13, 28, 33, 47, 54, 64, 70, 72}},
    {12, 85,  {0, 2, 6, 24, 29, 40, 43, 55, 68, 75, 76, 85}},
};
\end{lstlisting}

\subsection{Suite de tests}

Le fichier \texttt{test\_correctness.cpp} implémente une suite de tests couvrant plusieurs aspects :

\begin{enumerate}
    \item \textbf{Tests d'optimalité} : vérification que la longueur trouvée correspond à $L^*(n)$ connu
    \item \textbf{Tests de structure} : vérification des propriétés de la règle (premier élément = 0, ordre croissant, longueur cohérente)
    \item \textbf{Tests d'unicité} : vérification que toutes les différences sont distinctes
    \item \textbf{Tests de reproductibilité} : exécutions multiples donnant le même résultat
    \item \textbf{Tests aux limites} : cas $n=2$, bornes insuffisantes, etc.
\end{enumerate}

\begin{lstlisting}[language=C++, caption={Fonctions de validation}]
// Verification de l'unicite des differences
bool verifyUniqueDifferences(const std::vector<int>& marks) {
    std::set<int> differences;
    for (size_t i = 0; i < marks.size(); ++i) {
        for (size_t j = i + 1; j < marks.size(); ++j) {
            int d = marks[j] - marks[i];
            if (differences.count(d)) return false;
            differences.insert(d);
        }
    }
    return true;
}

// Verification de la structure
bool verifyRulerStructure(const GolombRuler& ruler, int expectedN) {
    if (ruler.marks.size() != expectedN) return false;
    if (ruler.marks[0] != 0) return false;
    for (size_t i = 1; i < ruler.marks.size(); ++i) {
        if (ruler.marks[i] <= ruler.marks[i-1]) return false;
    }
    if (ruler.length != ruler.marks.back()) return false;
    return true;
}
\end{lstlisting}

\subsection{Exécution des tests}

\begin{lstlisting}[language=bash, caption={Lancement des tests}]
# Compilation et execution des tests
make test

# Ou manuellement
./build/golomb_sequential_dev
\end{lstlisting}

Sortie typique :
\begin{lstlisting}[basicstyle=\tiny\ttfamily]
============================================
  Golomb Ruler Correctness Test Suite
  CSAPP Principle #10: Safety Verification
============================================

=== Testing Known Optimal Solutions ===
Testing n=2... PASSED (L=1)
Testing n=3... PASSED (L=3)
Testing n=4... PASSED (L=6)
Testing n=5... PASSED (L=11)
Testing n=6... PASSED (L=17)
Testing n=7... PASSED (L=25)
Testing n=8... PASSED (L=34)

=== Testing Edge Cases ===
Testing n=2 (minimal)... PASSED
Testing n=3... PASSED (L=3)
Testing tight bound (n=6, maxLen=17)... PASSED
Testing insufficient bound (n=6, maxLen=15)... PASSED (correctly bounded)

=== Testing Reproducibility ===
Testing multiple runs for n=8... PASSED (all found L=34)

=== Testing Validation Method ===
Testing valid ruler... PASSED
Testing invalid ruler (duplicate diff)... PASSED (correctly rejected)

============================================
  ALL TESTS PASSED
============================================
\end{lstlisting}

\subsection{Critères de validation}

Une implémentation est considérée \textbf{valide} si et seulement si :

\begin{defi}{Critères de validation}
\begin{enumerate}
    \item Pour tout $n$ testé : $L_{trouvée}(n) = L^*(n)$ (optimalité)
    \item La règle retournée satisfait : \texttt{GolombRuler::isValid(marks) == true}
    \item La structure est cohérente : $marks[0] = 0$, ordre croissant, $length = marks.back()$
    \item Les résultats sont reproductibles entre exécutions
\end{enumerate}
\end{defi}

\begin{figure}[H]
\centering
\begin{tikzpicture}[
    node distance=1cm,
    box/.style={rectangle, draw=burgundy, fill=burgundy!10, rounded corners, minimum width=3cm, minimum height=0.8cm, align=center},
    arrow/.style={->, thick}
]
    \node[box] (input) {Entrée : $n$, $maxLen$};
    \node[box, below=of input] (search) {Algorithme de recherche};
    \node[box, below=of search] (result) {Résultat : règle $G$};

    \node[box, right=2cm of result] (check1) {$L(G) = L^*(n)$ ?};
    \node[box, below=of check1] (check2) {isValid($G$) ?};
    \node[box, below=of check2] (check3) {Structure OK ?};

    \node[draw=green!60!black, fill=green!20, rounded corners, right=1.5cm of check2] (valid) {VALIDE};
    \node[draw=red!60, fill=red!20, rounded corners, below=0.5cm of check3] (invalid) {INVALIDE};

    \draw[arrow] (input) -- (search);
    \draw[arrow] (search) -- (result);
    \draw[arrow] (result) -- (check1);
    \draw[arrow] (check1) -- node[right] {oui} (check2);
    \draw[arrow] (check2) -- node[right] {oui} (check3);
    \draw[arrow] (check3) -- node[below] {oui} (valid);

    \draw[arrow, red] (check1.east) -- ++(0.5,0) |- node[near start, above] {non} (invalid);
    \draw[arrow, red] (check2.east) -- ++(0.3,0) |- (invalid);
    \draw[arrow, red] (check3.south) -- (invalid);
\end{tikzpicture}
\caption{Pipeline de validation d'une solution}
\label{fig:validation_pipeline}
\end{figure}


% Chapitre 6 : Parallélisation OpenMP
\chapter{Parallélisation OpenMP}

\epigraph{\textit{``The real problem is that programmers have spent far too much time worrying about efficiency in the wrong places.''}}{--- Donald Knuth}

\section{Objectif : accélérer sur un nœud}

\subsection{Contexte et motivation}

OpenMP (\textit{Open Multi-Processing}) est une API de programmation parallèle pour systèmes à \textbf{mémoire partagée}. Elle permet d'exploiter les multiples cœurs d'un processeur moderne au sein d'un même nœud de calcul.

\begin{defi}{Modèle fork-join}
OpenMP suit un modèle \textit{fork-join} :
\begin{enumerate}
    \item Le programme démarre avec un \textbf{thread maître} ;
    \item À une région parallèle, le maître \textit{fork} une équipe de threads ;
    \item Les threads travaillent en parallèle ;
    \item À la fin de la région, les threads se \textit{joignent} au maître.
\end{enumerate}
\end{defi}

\begin{figure}[H]
\centering
\begin{tikzpicture}[scale=0.8]
    % Timeline
    \draw[thick, ->] (0,0) -- (14,0) node[right] {temps};

    % Master thread
    \draw[very thick, burgundy] (0,0.5) -- (2,0.5);
    \node[above, burgundy] at (1, 0.6) {Master};

    % Fork
    \draw[thick, burgundy, ->] (2,0.5) -- (3,2);
    \draw[thick, burgundy, ->] (2,0.5) -- (3,1);
    \draw[thick, burgundy, ->] (2,0.5) -- (3,0);
    \draw[thick, burgundy, ->] (2,0.5) -- (3,-1);

    % Parallel work
    \draw[thick, gold] (3,2) -- (10,2) node[midway, above] {Thread 0};
    \draw[thick, gold] (3,1) -- (10,1) node[midway, above] {Thread 1};
    \draw[thick, gold] (3,0) -- (10,0) node[midway, above] {Thread 2};
    \draw[thick, gold] (3,-1) -- (10,-1) node[midway, above] {Thread 3};

    % Join
    \draw[thick, burgundy, ->] (10,2) -- (11,0.5);
    \draw[thick, burgundy, ->] (10,1) -- (11,0.5);
    \draw[thick, burgundy, ->] (10,0) -- (11,0.5);
    \draw[thick, burgundy, ->] (10,-1) -- (11,0.5);

    % Master continues
    \draw[very thick, burgundy] (11,0.5) -- (13,0.5);

    % Labels
    \node at (2, -2) {fork};
    \node at (11, -2) {join};
    \draw[dashed] (2,-1.5) -- (2,2.5);
    \draw[dashed] (11,-1.5) -- (11,2.5);
\end{tikzpicture}
\caption{Modèle fork-join d'OpenMP}
\label{fig:fork_join}
\end{figure}

\subsection{Avantages pour notre problème}

Le problème OGR se prête bien à la parallélisation OpenMP pour plusieurs raisons :

\begin{itemize}
    \item \textbf{Indépendance des branches} : chaque sous-arbre de recherche peut être exploré indépendamment ;
    \item \textbf{Partage de la borne} : la meilleure solution courante ($bestLen$) peut être partagée efficacement en mémoire partagée ;
    \item \textbf{Faible surcoût de synchronisation} : seule la mise à jour de $bestLen$ nécessite une synchronisation.
\end{itemize}

\subsection{Évolution des versions}

Cinq versions OpenMP ont été développées, chacune apportant des optimisations spécifiques :

\begin{table}[H]
\centering
\begin{tabular}{clp{6cm}}
\toprule
\textbf{Version} & \textbf{Approche} & \textbf{Optimisations clés} \\
\midrule
V1 & Itérative simple & Loop unrolling 4x, pile manuelle \\
V2 & Récursive + bitset & \texttt{bitset<256>} shift O(1) \\
V3 & Hybride & Itératif + bitset shift \\
V4 & Préfixes + bitset & Génération de préfixes, meilleur équilibrage \\
V5 & BitSet128 + préfixes & \texttt{uint64\_t} natif, performances maximales \\
\bottomrule
\end{tabular}
\caption{Évolution des versions OpenMP}
\label{tab:openmp_versions}
\end{table}

\section{Décomposition du travail}

\subsection{Approche initiale : distribution par firstMark (V1)}

La première approche distribue les branches de premier niveau entre les threads :

\begin{lstlisting}[language=C++, caption={Distribution simple par firstMark (V1)}]
#pragma omp parallel
{
    #pragma omp for schedule(dynamic, 1)
    for (int firstMark = 1; firstMark <= maxLen; ++firstMark) {
        // Chaque thread explore un sous-arbre complet
        backtrackIterative(firstMark, ...);
    }
}
\end{lstlisting}

\begin{figure}[H]
\centering
\begin{tikzpicture}[scale=0.7,
    level distance=1.5cm,
    sibling distance=2cm,
    edge from parent/.style={draw, -latex},
]
    \node[draw, circle, fill=burgundy!20] (root) {$\{0\}$}
        child { node[draw, circle, fill=gold!30] {$m_1=1$}
            child { node[draw, circle, fill=gold!10] {...} }
            child { node[draw, circle, fill=gold!10] {...} }
        }
        child { node[draw, circle, fill=blue!30] {$m_1=2$}
            child { node[draw, circle, fill=blue!10] {...} }
            child { node[draw, circle, fill=blue!10] {...} }
        }
        child { node[draw, circle, fill=green!30] {$m_1=3$}
            child { node[draw, circle, fill=green!10] {...} }
        }
        child { node[draw, circle, fill=red!30] {$m_1=4$}
            child { node[draw, circle, fill=red!10] {...} }
        };

    % Legend
    \node[right] at (6, 0) {Thread 0};
    \node[right] at (6, -1) {Thread 1};
    \node[right] at (6, -2) {Thread 2};
    \node[right] at (6, -3) {Thread 3};

    \draw[fill=gold!30] (5.5, -0.15) rectangle (6, 0.15);
    \draw[fill=blue!30] (5.5, -1.15) rectangle (6, -0.85);
    \draw[fill=green!30] (5.5, -2.15) rectangle (6, -1.85);
    \draw[fill=red!30] (5.5, -3.15) rectangle (6, -2.85);
\end{tikzpicture}
\caption{Distribution des branches de premier niveau entre threads}
\label{fig:firstmark_distribution}
\end{figure}

\textbf{Problème} : les branches ont des tailles très différentes. La branche $m_1 = 1$ est beaucoup plus grande que $m_1 = 50$, créant un déséquilibre de charge.

\subsection{Approche par génération de préfixes (V4, V5)}

Pour améliorer l'équilibrage, les versions V4 et V5 génèrent d'abord tous les \textbf{préfixes valides} jusqu'à une profondeur $D$, puis distribuent ces préfixes entre les threads.

\begin{lstlisting}[language=C++, caption={Génération de préfixes (V5)}]
// PHASE 1: Generation sequentielle des prefixes
std::vector<WorkItemV5> prefixes;
generatePrefixesV5(reversed_marks, used_dist, 1, 0,
                   prefixDepth, n, maxLen + 1, prefixes);

// PHASE 2: Exploration parallele des prefixes
#pragma omp parallel
{
    #pragma omp for schedule(dynamic, 1)
    for (int i = 0; i < numPrefixes; ++i) {
        const WorkItemV5& prefix = prefixes[i];
        backtrackIterativeV5(prefix, ...);
    }
}
\end{lstlisting}

\subsubsection{Choix de la profondeur de préfixe}

La profondeur optimale dépend de $n$ et du nombre de threads :

\begin{lstlisting}[language=C++, caption={Calcul automatique de la profondeur}]
static int computePrefixDepthV5(int n, int numThreads) {
    if (n <= 6)  return 2;
    if (n <= 8)  return 3;
    if (n <= 10) return 3;
    if (n <= 12) return 4;
    if (n <= 14) return 4;
    if (n <= 16) return 5;
    // ...
}
\end{lstlisting}

\begin{table}[H]
\centering
\begin{tabular}{ccc}
\toprule
\textbf{Ordre $n$} & \textbf{Profondeur $D$} & \textbf{Préfixes générés (approx.)} \\
\midrule
6--8 & 3 & $\sim$100 \\
9--10 & 3 & $\sim$500 \\
11--12 & 4 & $\sim$5 000 \\
13--14 & 4 & $\sim$50 000 \\
15--16 & 5 & $\sim$500 000 \\
\bottomrule
\end{tabular}
\caption{Profondeur de préfixe et nombre de tâches générées}
\label{tab:prefix_depth}
\end{table}

\subsection{Granularité des tâches}

La granularité représente le compromis entre :
\begin{itemize}
    \item \textbf{Granularité fine} (nombreuses petites tâches) : meilleur équilibrage, mais surcoût de scheduling ;
    \item \textbf{Granularité grossière} (peu de grandes tâches) : moins de surcoût, mais risque de déséquilibre.
\end{itemize}

\begin{figure}[H]
\centering
\begin{tikzpicture}[scale=0.9]
    \begin{axis}[
        xlabel={Nombre de tâches},
        ylabel={Performance},
        xmode=log,
        grid=major,
        width=10cm,
        height=6cm,
        legend pos=south east,
    ]
    % Courbe conceptuelle
    \addplot[thick, burgundy, smooth] coordinates {
        (10, 30) (50, 60) (200, 85) (1000, 95) (5000, 98)
        (20000, 97) (100000, 90) (500000, 80)
    };
    \addlegendentry{Performance relative}

    % Zone optimale
    \draw[fill=green!20, opacity=0.5] (axis cs:500, 0) rectangle (axis cs:50000, 100);
    \end{axis}

    \node at (5, 0.5) {\small Zone optimale};
\end{tikzpicture}
\caption{Impact de la granularité sur la performance}
\label{fig:granularity}
\end{figure}

\section{Ordonnancement et équilibrage}

\subsection{Politiques d'ordonnancement OpenMP}

OpenMP propose plusieurs politiques pour distribuer les itérations :

\begin{table}[H]
\centering
\begin{tabular}{lp{8cm}}
\toprule
\textbf{Politique} & \textbf{Description} \\
\midrule
\texttt{static} & Division en blocs de taille fixe, assignation circulaire \\
\texttt{dynamic} & Attribution à la demande, taille de bloc configurable \\
\texttt{guided} & Blocs décroissants, commence grand puis réduit \\
\texttt{auto} & Laisse le runtime choisir \\
\bottomrule
\end{tabular}
\caption{Politiques d'ordonnancement OpenMP}
\label{tab:scheduling}
\end{table}

\subsection{Choix de \texttt{dynamic}}

Notre implémentation utilise \texttt{schedule(dynamic, 1)} :

\begin{lstlisting}[language=C++]
#pragma omp for schedule(dynamic, 1)
for (int i = 0; i < numPrefixes; ++i) {
    // ...
}
\end{lstlisting}

\textbf{Justification} :

\begin{enumerate}
    \item \textbf{Tâches de taille variable} : même avec la génération de préfixes, les sous-arbres ont des tailles différentes selon l'élagage ;

    \item \textbf{Chunk size = 1} : chaque thread demande une nouvelle tâche dès qu'il termine, minimisant le temps d'inactivité ;

    \item \textbf{Surcoût acceptable} : le temps d'exploration d'un préfixe ($\sim$ms) domine largement le surcoût d'ordonnancement ($\sim\mu$s).
\end{enumerate}

\begin{figure}[H]
\centering
\begin{tikzpicture}[scale=0.6]
    % Static scheduling
    \node[anchor=west] at (-2, 6) {\textbf{static}};
    \foreach \t in {0,1,2,3} {
        \draw (0, 5.5-\t*0.8) -- (12, 5.5-\t*0.8);
        \node[left] at (0, 5.5-\t*0.8) {T\t};
    }
    % Blocks for static
    \foreach \x/\w/\c in {0/2/gold, 2/3/gold, 5/1.5/gold, 6.5/4/gold} {
        \fill[\c!50] (\x, 5.3) rectangle (\x+\w, 5.7);
    }
    \foreach \x/\w/\c in {0/1/blue, 1/2.5/blue, 3.5/3/blue, 6.5/2/blue} {
        \fill[\c!50] (\x, 4.5) rectangle (\x+\w, 4.9);
    }
    \foreach \x/\w/\c in {0/3/green, 3/1/green, 4/2/green, 6/5/green} {
        \fill[\c!50] (\x, 3.7) rectangle (\x+\w, 4.1);
    }
    \foreach \x/\w/\c in {0/0.5/red, 0.5/1.5/red, 2/1/red, 3/8/red} {
        \fill[\c!50] (\x, 2.9) rectangle (\x+\w, 3.3);
    }
    \draw[thick, red, dashed] (10.5, 2.9) -- (10.5, 5.7);
    \node[red] at (11.5, 4.3) {idle};

    % Dynamic scheduling
    \node[anchor=west] at (-2, 1) {\textbf{dynamic}};
    \foreach \t in {0,1,2,3} {
        \draw (0, 0.5-\t*0.8) -- (12, 0.5-\t*0.8);
        \node[left] at (0, 0.5-\t*0.8) {T\t};
    }
    % Better balanced
    \foreach \x/\w in {0/2, 2/1.5, 3.5/2, 5.5/1.5, 7/1, 8/0.5} {
        \fill[gold!50] (\x, 0.3) rectangle (\x+\w, 0.7);
    }
    \foreach \x/\w in {0/1, 1/2.5, 3.5/1.5, 5/2, 7/1.5} {
        \fill[blue!50] (\x, -0.5) rectangle (\x+\w, -0.1);
    }
    \foreach \x/\w in {0/3, 3/1.5, 4.5/2, 6.5/2} {
        \fill[green!50] (\x, -1.3) rectangle (\x+\w, -0.9);
    }
    \foreach \x/\w in {0/0.5, 0.5/3, 3.5/2, 5.5/3} {
        \fill[red!50] (\x, -2.1) rectangle (\x+\w, -1.7);
    }
    \draw[thick, green!50!black, dashed] (8.5, -2.1) -- (8.5, 0.7);
    \node[green!50!black] at (10, -0.7) {meilleur};
\end{tikzpicture}
\caption{Comparaison \texttt{static} vs \texttt{dynamic} : l'ordonnancement dynamique réduit le temps d'inactivité}
\label{fig:static_vs_dynamic}
\end{figure}

\subsection{Pourquoi pas \texttt{guided} ?}

La politique \texttt{guided} commence avec de grands blocs et les réduit progressivement. Elle est moins adaptée car :
\begin{itemize}
    \item Les premières tâches assignées en gros blocs peuvent créer un déséquilibre initial ;
    \item Avec des tâches de taille très variable, la décroissance des blocs ne correspond pas à la réalité de notre problème.
\end{itemize}

\section{Synchronisation autour de la borne}

\subsection{Le problème de la borne partagée}

Tous les threads partagent la meilleure longueur trouvée ($bestLen$). Cette valeur doit être :
\begin{itemize}
    \item \textbf{Lue fréquemment} : pour le pruning efficace ;
    \item \textbf{Mise à jour atomiquement} : quand une meilleure solution est trouvée.
\end{itemize}

\subsection{Solution : variables atomiques avec CAS}

Nous utilisons \texttt{std::atomic<int>} avec l'opération \textbf{Compare-And-Swap} (CAS) :

\begin{lstlisting}[language=C++, caption={Mise à jour atomique de la borne globale}]
std::atomic<int> globalBestLen(maxLen + 1);

// Dans le hot path - lecture relaxee (faible cout)
const int currentGlobal = globalBestLen.load(std::memory_order_relaxed);
if (ruler_length + minAdditional >= currentGlobal) {
    continue;  // Pruning
}

// Quand une solution est trouvee - CAS
if (solutionLen < threadBest.bestLen) {
    threadBest.bestLen = solutionLen;
    // ...

    // Mise a jour globale avec CAS
    int expected = globalBestLen.load(std::memory_order_relaxed);
    while (solutionLen < expected &&
           !globalBestLen.compare_exchange_weak(expected, solutionLen,
               std::memory_order_release, std::memory_order_relaxed)) {
        // Retry si un autre thread a mis a jour entre-temps
    }
}
\end{lstlisting}

\subsection{Ordres mémoire utilisés}

\begin{table}[H]
\centering
\begin{tabular}{llp{6cm}}
\toprule
\textbf{Opération} & \textbf{Ordre} & \textbf{Justification} \\
\midrule
Lecture (pruning) & \texttt{relaxed} & Pas critique si on lit une ancienne valeur ; on élaguera moins, mais correctement \\
Lecture avant CAS & \texttt{relaxed} & Même raison \\
CAS succès & \texttt{release} & Les autres threads verront la nouvelle valeur \\
CAS échec & \texttt{relaxed} & On va réessayer de toute façon \\
\bottomrule
\end{tabular}
\caption{Ordres mémoire et justifications}
\label{tab:memory_orders}
\end{table}

\subsection{Double-check et thread-local best}

Pour minimiser les accès atomiques, chaque thread maintient sa propre meilleure solution :

\begin{lstlisting}[language=C++, caption={Pattern thread-local best}]
struct ThreadBest {
    int bestLen;
    int bestMarks[MAX_MARKS];
    int bestNumMarks;
};

#pragma omp parallel
{
    ThreadBest threadBest;
    threadBest.bestLen = maxLen + 1;

    // ... exploration ...

    // A la fin: fusion des resultats locaux
    #pragma omp critical(merge_best)
    {
        if (threadBest.bestLen < finalBestLen) {
            finalBestLen = threadBest.bestLen;
            // Copier la solution
        }
    }
}
\end{lstlisting}

\begin{figure}[H]
\centering
\begin{tikzpicture}[
    node distance=1.5cm,
    box/.style={rectangle, draw=burgundy, fill=burgundy!10, rounded corners, minimum width=2.5cm, minimum height=1cm, align=center},
    arrow/.style={->, thick}
]
    % Global
    \node[box, fill=gold!30] (global) {globalBestLen\\(atomic)};

    % Thread locals
    \node[box, below left=2cm and 1cm of global] (t0) {Thread 0\\local best};
    \node[box, below=2cm of global] (t1) {Thread 1\\local best};
    \node[box, below right=2cm and 1cm of global] (t2) {Thread 2\\local best};

    % Arrows
    \draw[arrow, dashed] (global) -- (t0) node[midway, left] {\small read};
    \draw[arrow, dashed] (global) -- (t1) node[midway, left] {\small read};
    \draw[arrow, dashed] (global) -- (t2) node[midway, right] {\small read};

    \draw[arrow, thick, red] (t0) -- (global) node[midway, left, red] {\small CAS};
    \draw[arrow, thick, red] (t1) -- (global);
    \draw[arrow, thick, red] (t2) -- (global) node[midway, right, red] {\small CAS};

    % Final merge
    \node[box, fill=green!20, below=4cm of global] (final) {Final Best\\(critical)};
    \draw[arrow] (t0) -- (final);
    \draw[arrow] (t1) -- (final);
    \draw[arrow] (t2) -- (final);
\end{tikzpicture}
\caption{Architecture de synchronisation : lectures relaxées fréquentes, CAS rares}
\label{fig:sync_architecture}
\end{figure}

\section{Analyse des limites}

\subsection{Contention sur la borne}

Bien que minimisée, la contention existe lors des mises à jour de \texttt{globalBestLen}. Elle est particulièrement notable :
\begin{itemize}
    \item En début de recherche, quand plusieurs threads trouvent des solutions rapidement ;
    \item Avec un grand nombre de threads (>64).
\end{itemize}

\textbf{Mesures de mitigation} :
\begin{itemize}
    \item Ordres mémoire relaxés pour les lectures ;
    \item CAS plutôt que mutex (non-bloquant) ;
    \item Thread-local best pour réduire les écritures.
\end{itemize}

\subsection{Surcoût de scheduling}

L'ordonnancement \texttt{dynamic} induit un surcoût :

\begin{equation}
T_{overhead} = N_{tasks} \times t_{schedule}
\end{equation}

où $t_{schedule} \approx 1\mu s$ par attribution de tâche.

Pour $N_{tasks} = 50\,000$ préfixes :
\[
T_{overhead} \approx 50\,000 \times 1\mu s = 50\,ms
\]

Ce surcoût est négligeable face au temps total de calcul ($\sim$minutes pour $n \geq 12$).

\subsection{Scalabilité et loi d'Amdahl}

La loi d'Amdahl limite le speedup maximal :

\begin{equation}
S(p) = \frac{1}{f + \frac{1-f}{p}}
\end{equation}

où $f$ est la fraction séquentielle (génération des préfixes, fusion des résultats).

\begin{figure}[H]
\centering
\begin{tikzpicture}[scale=0.9]
    \begin{axis}[
        xlabel={Nombre de threads $p$},
        ylabel={Speedup $S(p)$},
        grid=major,
        width=10cm,
        height=6cm,
        legend pos=north west,
        xmin=1, xmax=200,
        ymin=0, ymax=100,
    ]
    % Speedup idéal
    \addplot[thick, dashed, gray, domain=1:200] {x};
    \addlegendentry{Idéal ($S = p$)}

    % f = 1%
    \addplot[thick, burgundy, domain=1:200] {1/(0.01 + 0.99/x)};
    \addlegendentry{$f = 1\%$ (max 100)}

    % f = 5%
    \addplot[thick, gold, domain=1:200] {1/(0.05 + 0.95/x)};
    \addlegendentry{$f = 5\%$ (max 20)}

    % f = 10%
    \addplot[thick, blue, domain=1:200] {1/(0.10 + 0.90/x)};
    \addlegendentry{$f = 10\%$ (max 10)}
    \end{axis}
\end{tikzpicture}
\caption{Loi d'Amdahl : impact de la fraction séquentielle sur le speedup}
\label{fig:amdahl}
\end{figure}

Dans notre cas, $f < 1\%$ pour $n \geq 12$ : la génération des préfixes et la fusion sont rapides face à l'exploration.

\subsection{Déséquilibre résiduel}

Malgré la génération de préfixes, un déséquilibre persiste car :
\begin{itemize}
    \item Certains préfixes mènent à des sous-arbres plus grands ;
    \item L'élagage dépend de $bestLen$, qui évolue pendant l'exécution ;
    \item Les threads qui explorent après une amélioration de $bestLen$ font moins de travail.
\end{itemize}

Ce déséquilibre est \textbf{bénéfique} : il reflète l'élagage dynamique qui accélère globalement la recherche.

\subsection{Limites matérielles}

Au-delà d'un certain nombre de threads, les performances saturent :
\begin{itemize}
    \item \textbf{Bande passante mémoire} : les accès concurrents saturent le bus ;
    \item \textbf{Cache thrashing} : avec trop de threads, les données sont évincées du cache ;
    \item \textbf{NUMA effects} : sur systèmes multi-sockets, les accès distants sont plus lents.
\end{itemize}

\begin{important}{Recommandation}
Sur nos tests, l'efficacité parallèle reste supérieure à 60\% jusqu'à $\sim$64 threads. Au-delà, les gains diminuent mais restent positifs jusqu'à $\sim$192 threads sur architecture adaptée (serveur multi-socket).
\end{important}


% Chapitre 7 : Parallélisation hybride MPI + OpenMP
\chapter{Parallélisation hybride MPI + OpenMP}

\epigraph{\textit{``The combination of MPI and OpenMP in a single program can take advantage of both shared memory and distributed memory hardware.''}}{--- Rolf Rabenseifner}

\section{Objectif et contraintes}

\subsection{Objectif : passage à l'échelle multi-nœuds}

Alors qu'OpenMP exploite les cœurs d'un seul nœud, MPI (\textit{Message Passing Interface}) permet de distribuer le calcul sur \textbf{plusieurs nœuds} d'un cluster. L'approche \textbf{hybride MPI+OpenMP} combine les deux paradigmes :

\begin{itemize}
    \item \textbf{MPI} : communication inter-nœuds (mémoire distribuée)
    \item \textbf{OpenMP} : parallélisme intra-nœud (mémoire partagée)
\end{itemize}

\begin{figure}[H]
\centering
\begin{tikzpicture}[scale=0.7]
    % Noeud 1
    \draw[thick, rounded corners, fill=burgundy!10] (0,0) rectangle (5,4);
    \node[above] at (2.5, 4) {\textbf{Nœud 1}};
    \node at (2.5, 3.3) {MPI Rank 0};
    \foreach \x in {0.5, 1.5, 2.5, 3.5} {
        \draw[fill=gold!40] (\x+0.2, 0.5) rectangle (\x+0.8, 2.5);
    }
    \node at (2.5, 1.5) {\tiny OpenMP};

    % Noeud 2
    \draw[thick, rounded corners, fill=burgundy!10] (6,0) rectangle (11,4);
    \node[above] at (8.5, 4) {\textbf{Nœud 2}};
    \node at (8.5, 3.3) {MPI Rank 1};
    \foreach \x in {6.5, 7.5, 8.5, 9.5} {
        \draw[fill=gold!40] (\x+0.2, 0.5) rectangle (\x+0.8, 2.5);
    }
    \node at (8.5, 1.5) {\tiny OpenMP};

    % Noeud 3
    \draw[thick, rounded corners, fill=burgundy!10] (12,0) rectangle (17,4);
    \node[above] at (14.5, 4) {\textbf{Nœud 3}};
    \node at (14.5, 3.3) {MPI Rank 2};
    \foreach \x in {12.5, 13.5, 14.5, 15.5} {
        \draw[fill=gold!40] (\x+0.2, 0.5) rectangle (\x+0.8, 2.5);
    }
    \node at (14.5, 1.5) {\tiny OpenMP};

    % Réseau
    \draw[<->, very thick, blue] (5, 2) -- (6, 2);
    \draw[<->, very thick, blue] (11, 2) -- (12, 2);
    \node[blue, below] at (8.5, -0.5) {Réseau (InfiniBand / Ethernet)};
    \draw[blue, thick] (2.5, -0.3) -- (14.5, -0.3);
\end{tikzpicture}
\caption{Architecture hybride MPI+OpenMP : chaque processus MPI lance plusieurs threads OpenMP}
\label{fig:hybrid_architecture}
\end{figure}

\subsection{Avantages de l'approche hybride}

\begin{table}[H]
\centering
\begin{tabular}{lp{5cm}p{5cm}}
\toprule
\textbf{Critère} & \textbf{MPI pur} & \textbf{Hybride MPI+OpenMP} \\
\midrule
Mémoire & Répliquée par processus & Partagée au sein du nœud \\
Communication & Coût élevé intra-nœud & Minimisée (OpenMP local) \\
Scalabilité & Limitée par la mémoire & Meilleure utilisation \\
Complexité & Modérée & Plus élevée \\
\bottomrule
\end{tabular}
\caption{Comparaison MPI pur vs hybride}
\label{tab:mpi_vs_hybrid}
\end{table}

\subsection{Contraintes selon les versions}

Trois versions MPI ont été implémentées avec des contraintes différentes :

\begin{table}[H]
\centering
\begin{tabular}{llp{6cm}}
\toprule
\textbf{Version} & \textbf{Contrainte sur $P$} & \textbf{Communication} \\
\midrule
V1 & $P = 2^k$ & Hypercube + loop unrolling \\
V2 & $P = 2^k$ & Hypercube + BitSet128 shift \\
V3 & Aucune & \texttt{MPI\_Allreduce} standard \\
\bottomrule
\end{tabular}
\caption{Versions MPI et leurs contraintes}
\label{tab:mpi_versions}
\end{table}

\begin{important}{Contrainte puissance de 2 (V1, V2)}
Les versions V1 et V2 utilisent une topologie \textbf{hypercube} qui requiert un nombre de processus $P = 2^k$. Cette contrainte permet des communications en $O(\log_2 P)$ étapes.
\end{important}

La version V3 lève cette contrainte en utilisant \texttt{MPI\_Allreduce}, permettant tout nombre de processus mais avec un surcoût potentiel.

\section{Partitionnement inter-processus}

\subsection{Stratégie de distribution}

L'espace de recherche est partitionné entre les processus MPI de manière \textbf{statique} et \textbf{déterministe} :

\begin{enumerate}
    \item Tous les processus génèrent \textbf{identiquement} l'ensemble des préfixes valides ;
    \item Chaque processus ne traite que les préfixes dont l'index lui est assigné.
\end{enumerate}

\begin{lstlisting}[language=C++, caption={Distribution des préfixes entre rangs MPI}]
// Tous les rangs generent les memes prefixes
std::vector<WorkItem> allPrefixes;
generatePrefixes(..., allPrefixes);

const int totalPrefixes = allPrefixes.size();
const int rank = hypercube.rank();
const int size = hypercube.size();

// Chaque rang ne garde que ses prefixes
std::vector<WorkItem> myPrefixes;
for (int i = 0; i < totalPrefixes; ++i) {
    if (i % size == rank) {
        myPrefixes.push_back(allPrefixes[i]);
    }
}
\end{lstlisting}

\subsection{Distribution cyclique}

La distribution cyclique (round-robin) assure un équilibrage initial :

\begin{figure}[H]
\centering
\begin{tikzpicture}[scale=0.6]
    % Préfixes
    \node at (-2, 0) {Préfixes :};
    \foreach \i in {0,...,11} {
        \draw (\i, -0.5) rectangle (\i+0.9, 0.5);
        \node at (\i+0.45, 0) {\small \i};
    }

    % Distribution
    \draw[->, thick] (5.5, -1) -- (5.5, -2);

    % Rank 0
    \node at (-2, -3) {Rank 0 :};
    \foreach \i/\c in {0/0, 4/4, 8/8} {
        \draw[fill=gold!50] (\c*0.7, -3.5) rectangle (\c*0.7+0.6, -2.5);
        \node at (\c*0.7+0.3, -3) {\small \i};
    }

    % Rank 1
    \node at (-2, -4.5) {Rank 1 :};
    \foreach \i/\c in {1/1, 5/5, 9/9} {
        \draw[fill=blue!50] (\c*0.7, -5) rectangle (\c*0.7+0.6, -4);
        \node at (\c*0.7+0.3, -4.5) {\small \i};
    }

    % Rank 2
    \node at (-2, -6) {Rank 2 :};
    \foreach \i/\c in {2/2, 6/6, 10/10} {
        \draw[fill=green!50] (\c*0.7, -6.5) rectangle (\c*0.7+0.6, -5.5);
        \node at (\c*0.7+0.3, -6) {\small \i};
    }

    % Rank 3
    \node at (-2, -7.5) {Rank 3 :};
    \foreach \i/\c in {3/3, 7/7, 11/11} {
        \draw[fill=red!50] (\c*0.7, -8) rectangle (\c*0.7+0.6, -7);
        \node at (\c*0.7+0.3, -7.5) {\small \i};
    }

    % Formule
    \node at (10, -5) {$\text{prefix}_i \to \text{Rank } (i \mod P)$};
\end{tikzpicture}
\caption{Distribution cyclique des préfixes entre 4 processus MPI}
\label{fig:cyclic_distribution}
\end{figure}

\subsection{Avantages de cette approche}

\begin{enumerate}
    \item \textbf{Pas de communication initiale} : chaque processus calcule ses préfixes localement ;
    \item \textbf{Déterminisme} : la même exécution donne toujours la même distribution ;
    \item \textbf{Équilibrage initial} : les préfixes sont répartis équitablement ($\pm 1$).
\end{enumerate}

\section{Recherche locale intra-processus}

\subsection{Structure de l'exécution}

Chaque processus MPI exécute une recherche OpenMP sur ses préfixes assignés :

\begin{lstlisting}[language=C++, caption={Exploration locale avec OpenMP}]
#pragma omp parallel shared(globalBestLen, localBestLen, ...)
{
    ThreadBest threadBest;
    StackFrame stack[MAX_MARKS];
    long long threadExplored = 0;

    #pragma omp for schedule(dynamic, 1)
    for (int idx = startIdx; idx < endIdx; ++idx) {
        const WorkItem& prefix = myPrefixes[idx];

        // Pruning precoce
        if (prefix.ruler_length + minAdditional >= currentGlobal) {
            continue;
        }

        // Initialisation de la pile
        stack[0] = prefix;

        // Backtracking iteratif
        backtrackIterative(threadBest, n, globalBestLen,
                           threadExplored, stack);
    }

    // Fusion locale
    #pragma omp critical
    {
        if (threadBest.bestLen < localBestLen) {
            localBestLen = threadBest.bestLen;
            // Copier la solution
        }
    }
}
\end{lstlisting}

\subsection{Hiérarchie des bornes}

Trois niveaux de bornes sont maintenus :

\begin{figure}[H]
\centering
\begin{tikzpicture}[
    node distance=1.5cm,
    box/.style={rectangle, draw, rounded corners, minimum width=3cm, minimum height=1cm, align=center},
]
    % Global MPI
    \node[box, fill=burgundy!30] (global) {globalBestLen\\(tous les rangs)};

    % Local process
    \node[box, fill=gold!30, below left=1.5cm and 0.5cm of global] (local0) {localBestLen\\(Rank 0)};
    \node[box, fill=gold!30, below right=1.5cm and 0.5cm of global] (local1) {localBestLen\\(Rank 1)};

    % Thread local
    \node[box, fill=blue!20, below=1.2cm of local0] (t00) {threadBest\\(T0)};
    \node[box, fill=blue!20, right=0.3cm of t00] (t01) {threadBest\\(T1)};

    \node[box, fill=blue!20, below=1.2cm of local1] (t10) {threadBest\\(T0)};
    \node[box, fill=blue!20, right=0.3cm of t10] (t11) {threadBest\\(T1)};

    % Arrows
    \draw[->, thick] (t00) -- (local0);
    \draw[->, thick] (t01) -- (local0);
    \draw[->, thick] (t10) -- (local1);
    \draw[->, thick] (t11) -- (local1);
    \draw[->, thick, red] (local0) -- (global) node[midway, left] {\small MPI};
    \draw[->, thick, red] (local1) -- (global) node[midway, right] {\small MPI};

    % Labels
    \node[right=3cm of global] {Synchronisation MPI périodique};
    \node[right=2cm of local1] {Fusion OpenMP (critical)};
    \node[right=1cm of t11] {Mise à jour atomique};
\end{tikzpicture}
\caption{Hiérarchie des bornes dans l'architecture hybride}
\label{fig:bound_hierarchy}
\end{figure}

\subsection{Exécution par rondes}

Pour permettre la synchronisation MPI périodique, l'exploration est découpée en \textbf{rondes} :

\begin{lstlisting}[language=C++, caption={Exécution par rondes avec synchronisation}]
constexpr int SYNC_INTERVAL = 64;  // Prefixes par ronde

int prefixIndex = 0;
while (prefixIndex < myNumPrefixes) {
    int endIdx = std::min(prefixIndex + SYNC_INTERVAL, myNumPrefixes);

    // Exploration OpenMP de [prefixIndex, endIdx)
    #pragma omp parallel for schedule(dynamic, 1)
    for (int idx = prefixIndex; idx < endIdx; ++idx) {
        // ... backtracking ...
    }

    prefixIndex = endIdx;

    // Synchronisation MPI (hypercube ou allreduce)
    int globalMin = hypercube.allReduceMin(localBestLen);
    globalBestLen = std::min(globalBestLen, globalMin);
}
\end{lstlisting}

\section{Communications : réduction hypercube}

\subsection{Topologie hypercube}

Un \textbf{hypercube} de dimension $d$ connecte $P = 2^d$ processus. Chaque processus a exactement $d$ voisins, obtenus en inversant un bit de son rang :

\begin{equation}
\text{voisin}(r, k) = r \oplus 2^k \quad \text{pour } k \in \{0, 1, \ldots, d-1\}
\end{equation}

\begin{figure}[H]
\centering
\begin{tikzpicture}[scale=1.2]
    % Hypercube 3D (8 processus)
    \coordinate (000) at (0,0,0);
    \coordinate (001) at (2,0,0);
    \coordinate (010) at (0,2,0);
    \coordinate (011) at (2,2,0);
    \coordinate (100) at (0.7,0.7,2);
    \coordinate (101) at (2.7,0.7,2);
    \coordinate (110) at (0.7,2.7,2);
    \coordinate (111) at (2.7,2.7,2);

    % Arêtes dimension 0 (rouge)
    \draw[thick, red] (000) -- (001);
    \draw[thick, red] (010) -- (011);
    \draw[thick, red] (100) -- (101);
    \draw[thick, red] (110) -- (111);

    % Arêtes dimension 1 (bleu)
    \draw[thick, blue] (000) -- (010);
    \draw[thick, blue] (001) -- (011);
    \draw[thick, blue] (100) -- (110);
    \draw[thick, blue] (101) -- (111);

    % Arêtes dimension 2 (vert)
    \draw[thick, green!50!black] (000) -- (100);
    \draw[thick, green!50!black] (001) -- (101);
    \draw[thick, green!50!black] (010) -- (110);
    \draw[thick, green!50!black] (011) -- (111);

    % Nœuds
    \foreach \coord/\label in {000/0, 001/1, 010/2, 011/3, 100/4, 101/5, 110/6, 111/7} {
        \fill[burgundy] (\coord) circle (0.15);
        \node[below left] at (\coord) {\label};
    }

    % Légende
    \node[red, right] at (4, 2) {dim 0 : $r \oplus 1$};
    \node[blue, right] at (4, 1.5) {dim 1 : $r \oplus 2$};
    \node[green!50!black, right] at (4, 1) {dim 2 : $r \oplus 4$};
\end{tikzpicture}
\caption{Hypercube 3D ($P=8$) : chaque nœud a 3 voisins}
\label{fig:hypercube}
\end{figure}

\subsection{Algorithme all-reduce minimum}

L'algorithme de réduction sur hypercube s'exécute en $\log_2 P$ étapes :

\begin{algorithm}[H]
\caption{All-reduce minimum sur hypercube}
\label{alg:hypercube_allreduce}
\begin{algorithmic}[1]
\Require Valeur locale $localMin$, rang $r$, dimension $d = \log_2 P$
\Ensure Tous les processus ont le minimum global
\State $result \gets localMin$
\For{$k \gets 0$ \textbf{to} $d-1$}
    \State $partner \gets r \oplus 2^k$ \Comment{Voisin en dimension $k$}
    \State Échanger $result$ avec $partner$ \Comment{\texttt{MPI\_Sendrecv}}
    \State $result \gets \min(result, \text{valeur reçue})$
\EndFor
\State \Return $result$
\end{algorithmic}
\end{algorithm}

\begin{lstlisting}[language=C++, caption={Implémentation all-reduce hypercube}]
int HypercubeMPI::allReduceMin(int localMin) {
    if (size_ == 1) return localMin;

    int result = localMin;

    for (int d = 0; d < dimensions_; ++d) {
        int partner = rank_ ^ (1 << d);  // XOR avec 2^d
        int recvVal;

        MPI_Sendrecv(&result, 1, MPI_INT, partner, d,
                     &recvVal, 1, MPI_INT, partner, d,
                     MPI_COMM_WORLD, MPI_STATUS_IGNORE);

        if (recvVal < result) {
            result = recvVal;
        }
    }

    return result;
}
\end{lstlisting}

\subsection{Complexité des communications}

\begin{table}[H]
\centering
\begin{tabular}{lcc}
\toprule
\textbf{Opération} & \textbf{Hypercube} & \textbf{MPI\_Allreduce} \\
\midrule
Nombre de rounds & $\log_2 P$ & $\leq 2\log_2 P$ (impl. dépendant) \\
Messages par round & 1 & Variable \\
Latence totale & $O(\log P)$ & $O(\log P)$ \\
Contrainte & $P = 2^k$ & Aucune \\
\bottomrule
\end{tabular}
\caption{Comparaison des stratégies de communication}
\label{tab:comm_comparison}
\end{table}

\begin{figure}[H]
\centering
\begin{tikzpicture}[scale=0.7]
    % Étapes de réduction sur 8 processus
    \foreach \t in {0,...,7} {
        \node at (-1, 3-\t*0.8) {P\t};
        \fill[burgundy] (0, 3-\t*0.8) circle (0.15);
    }

    % Étape 0 (dimension 0)
    \node at (2, 4) {\small Étape 0};
    \foreach \y in {0,2,4,6} {
        \draw[->, thick, red] (0.3, 3-\y*0.8) -- (1.7, 3-\y*0.8-0.8);
        \draw[->, thick, red] (0.3, 3-\y*0.8-0.8) -- (1.7, 3-\y*0.8);
    }
    \foreach \t in {0,...,7} {
        \fill[burgundy] (2, 3-\t*0.8) circle (0.15);
    }

    % Étape 1 (dimension 1)
    \node at (4, 4) {\small Étape 1};
    \foreach \y in {0,4} {
        \draw[->, thick, blue] (2.3, 3-\y*0.8) -- (3.7, 3-\y*0.8-1.6);
        \draw[->, thick, blue] (2.3, 3-\y*0.8-0.8) -- (3.7, 3-\y*0.8-1.6-0.8);
        \draw[->, thick, blue] (2.3, 3-\y*0.8-1.6) -- (3.7, 3-\y*0.8);
        \draw[->, thick, blue] (2.3, 3-\y*0.8-1.6-0.8) -- (3.7, 3-\y*0.8-0.8);
    }
    \foreach \t in {0,...,7} {
        \fill[burgundy] (4, 3-\t*0.8) circle (0.15);
    }

    % Étape 2 (dimension 2)
    \node at (6, 4) {\small Étape 2};
    \draw[->, thick, green!50!black] (4.3, 3) -- (5.7, 3-3.2);
    \draw[->, thick, green!50!black] (4.3, 3-3.2) -- (5.7, 3);
    \foreach \t in {0,...,7} {
        \fill[green!50!black] (6, 3-\t*0.8) circle (0.15);
    }

    \node at (3, -4) {$\log_2(8) = 3$ étapes pour 8 processus};
\end{tikzpicture}
\caption{Déroulement de l'all-reduce sur hypercube ($P=8$)}
\label{fig:allreduce_steps}
\end{figure}

\subsection{Alternative : MPI\_Allreduce (V3)}

La version V3 utilise \texttt{MPI\_Allreduce} standard, permettant tout nombre de processus :

\begin{lstlisting}[language=C++, caption={Synchronisation avec MPI\_Allreduce (V3)}]
// Pas de contrainte sur le nombre de processus
int localMin = localBestLen;
int globalMin;

MPI_Allreduce(&localMin, &globalMin, 1, MPI_INT,
              MPI_MIN, MPI_COMM_WORLD);

// Mise a jour de la borne
globalBestLen = globalMin;
\end{lstlisting}

\section{Équilibrage de charge}

\subsection{Sources de déséquilibre}

Malgré la distribution cyclique, plusieurs facteurs créent un déséquilibre :

\begin{enumerate}
    \item \textbf{Taille variable des sous-arbres} : certains préfixes mènent à des explorations plus longues ;
    \item \textbf{Élagage dynamique} : les processus découvrant tôt une bonne solution élaguent plus efficacement ;
    \item \textbf{Synchronisation périodique} : les processus rapides attendent aux points de synchronisation.
\end{enumerate}

\begin{figure}[H]
\centering
\begin{tikzpicture}[scale=0.6]
    % Timeline
    \draw[->] (0,0) -- (16,0) node[right] {temps};

    % Processus
    \foreach \p/\label in {0/P0, 1/P1, 2/P2, 3/P3} {
        \node[left] at (0, 3-\p) {\label};
        \draw[dashed, gray] (0, 3-\p) -- (15, 3-\p);
    }

    % Travail (barres de différentes longueurs)
    \fill[gold!70] (0, 2.7) rectangle (12, 3.3);
    \fill[gold!70] (0, 1.7) rectangle (8, 2.3);
    \fill[gold!70] (0, 0.7) rectangle (14, 1.3);
    \fill[gold!70] (0, -0.3) rectangle (6, 0.3);

    % Points de sync
    \foreach \x in {6, 12} {
        \draw[thick, red, dashed] (\x, -0.5) -- (\x, 3.5);
        \node[red, above] at (\x, 3.5) {\small sync};
    }

    % Temps d'attente (idle)
    \fill[gray!30] (8, 1.7) rectangle (12, 2.3);
    \fill[gray!30] (6, -0.3) rectangle (12, 0.3);

    % Fin
    \draw[thick, green!50!black] (14, -0.5) -- (14, 3.5);
    \node[green!50!black, above] at (14, 3.5) {\small fin};

    % Légende
    \node at (8, -2) {Travail};
    \fill[gold!70] (5, -2.2) rectangle (6, -1.8);
    \node at (12, -2) {Idle};
    \fill[gray!30] (9, -2.2) rectangle (10, -1.8);
\end{tikzpicture}
\caption{Déséquilibre de charge : P3 termine tôt et attend}
\label{fig:load_imbalance}
\end{figure}

\subsection{Impact du déséquilibre}

Le temps total est déterminé par le processus le plus lent :

\begin{equation}
T_{total} = \max_i(T_i) \geq \frac{1}{P} \sum_i T_i = \frac{T_{seq}}{P}
\end{equation}

L'\textbf{efficacité} est réduite :
\begin{equation}
E = \frac{T_{seq}}{P \cdot T_{total}} = \frac{\bar{T}}{T_{max}} \leq 1
\end{equation}

où $\bar{T}$ est le temps moyen et $T_{max}$ le temps maximum.

\subsection{Mesures actuelles}

Notre implémentation utilise plusieurs techniques pour limiter le déséquilibre :

\begin{enumerate}
    \item \textbf{Distribution cyclique} : mélange les préfixes entre processus ;
    \item \textbf{Granularité fine} : nombreux préfixes ($\gg P$) pour le lissage statistique ;
    \item \textbf{OpenMP dynamic} : équilibrage intra-nœud ;
    \item \textbf{Synchronisation fréquente} : propagation rapide des bornes.
\end{enumerate}

\subsection{Pistes d'amélioration}

Plusieurs approches pourraient améliorer l'équilibrage :

\begin{table}[H]
\centering
\begin{tabular}{lp{5cm}p{4cm}}
\toprule
\textbf{Technique} & \textbf{Principe} & \textbf{Complexité} \\
\midrule
Work stealing & Les processus inactifs ``volent'' du travail aux processus chargés & Élevée (communication) \\
\midrule
Distribution dynamique & Un processus maître distribue les tâches à la demande & Goulot d'étranglement potentiel \\
\midrule
Estimation de coût & Estimer la taille des sous-arbres pour équilibrer a priori & Difficile sans exploration \\
\midrule
Over-decomposition & Générer beaucoup plus de préfixes que de processus & Surcoût mémoire \\
\bottomrule
\end{tabular}
\caption{Pistes d'amélioration de l'équilibrage}
\label{tab:load_balance_improvements}
\end{table}

\begin{defi}{Recommandation pratique}
Pour notre problème, la combinaison de :
\begin{itemize}
    \item Génération de préfixes à profondeur adaptée ($\sim$1000--10000 préfixes par processus)
    \item Distribution cyclique
    \item Synchronisation toutes les 64 tâches
\end{itemize}
offre un bon compromis entre équilibrage et surcoût de communication.
\end{defi}

\subsection{Résumé de l'architecture}

\begin{figure}[H]
\centering
\begin{tikzpicture}[
    node distance=0.8cm,
    phase/.style={rectangle, draw=burgundy, fill=burgundy!10, rounded corners, minimum width=10cm, minimum height=1cm, align=center},
    arrow/.style={->, thick}
]
    \node[phase] (p1) {Phase 1 : Génération des préfixes (séquentielle, tous les rangs)};
    \node[phase, below=of p1] (p2) {Phase 2 : Distribution cyclique des préfixes};
    \node[phase, below=of p2] (p3) {Phase 3 : Exploration par rondes};
    \node[phase, fill=gold!20, below=of p3] (p3a) {OpenMP : backtracking parallèle intra-nœud};
    \node[phase, fill=blue!20, below=of p3a] (p3b) {MPI : synchronisation hypercube inter-nœuds};
    \node[phase, below=of p3b] (p4) {Phase 4 : Réduction finale et broadcast de la solution};

    \draw[arrow] (p1) -- (p2);
    \draw[arrow] (p2) -- (p3);
    \draw[arrow] (p3) -- (p3a);
    \draw[arrow] (p3a) -- (p3b);
    \draw[arrow, dashed] (p3b.east) -- ++(1,0) |- (p3.east);
    \draw[arrow] (p3b) -- (p4);

    \node[right=0.5cm of p3b] {\small Itérer};
\end{tikzpicture}
\caption{Vue d'ensemble de l'algorithme hybride MPI+OpenMP}
\label{fig:algorithm_overview}
\end{figure}


% Chapitre 8 : Protocole expérimental (benchmarking)
\chapter{Protocole expérimental et benchmarking}

\epigraph{\textit{``In God we trust, all others must bring data.''}}{--- W. Edwards Deming}

\section{Environnement d'exécution}

\subsection{Supercalculateur Romeo}

Les expérimentations ont été réalisées sur le supercalculateur \textbf{Romeo} de l'Université de Reims Champagne-Ardenne, membre du mésocentre de calcul régional. Romeo est un cluster hétérogène de nouvelle génération combinant des architectures x86 et ARM.

\begin{table}[H]
\centering
\begin{tabular}{ll}
\toprule
\textbf{Caractéristique} & \textbf{Valeur} \\
\midrule
Cluster & Romeo (URCA) \\
Gestionnaire de ressources & SLURM \\
Réseau d'interconnexion & InfiniBand HDR \\
Système de fichiers & Lustre (scratch partagé) \\
Nombre total de nœuds & 102 (58 ARM + 44 x86) \\
Cœurs CPU totaux & $\sim$25\,000 \\
\bottomrule
\end{tabular}
\caption{Caractéristiques générales du supercalculateur Romeo}
\label{tab:romeo_general}
\end{table}

\subsection{Partitions et politiques d'ordonnancement}

Romeo propose trois partitions avec des limites de temps différentes :

\begin{table}[H]
\centering
\begin{tabular}{lcccp{4cm}}
\toprule
\textbf{Partition} & \textbf{Nœuds ARM} & \textbf{Nœuds x86} & \textbf{Total} & \textbf{Usage} \\
\midrule
\texttt{instant} (défaut) & 58 & 44 & 102 & Jobs courts, tests interactifs \\
\texttt{short} & 56 & 43 & 99 & Jobs de quelques heures \\
\texttt{long} & 40 & 24 & 64 & Jobs longs (jours) \\
\bottomrule
\end{tabular}
\caption{Partitions disponibles sur Romeo}
\label{tab:romeo_partitions}
\end{table}

\subsection{Quotas et limites de ressources}

L'accès aux ressources de calcul est contrôlé par un système de quotas associé à chaque compte utilisateur. Ces quotas limitent le nombre total de CPUs utilisables simultanément :

\begin{table}[H]
\centering
\begin{tabular}{lc}
\toprule
\textbf{Paramètre} & \textbf{Valeur} \\
\midrule
Limite CPUs simultanés (GrpCPUs) & 400 \\
Nœuds x86 max (192 CPUs/nœud) & 2 \\
Nœuds ARM max (288 CPUs/nœud) & 1 \\
\bottomrule
\end{tabular}
\caption{Quotas de ressources pour le compte projet}
\label{tab:quotas}
\end{table}

Cette limite de 400 CPUs simultanés impose des contraintes sur l'exécution parallèle des benchmarks : un job ARM utilisant 192 CPUs empêche le lancement simultané d'un job MPI multi-nœuds (2 nœuds $\times$ 192 CPUs = 384 CPUs supplémentaires dépasserait le quota). Les jobs sont alors placés en file d'attente avec le statut \texttt{AssocGrpCpuLimit} jusqu'à libération des ressources.

\subsection{Architecture des nœuds de calcul}

Romeo dispose de deux familles de nœuds aux caractéristiques distinctes :

\subsubsection{Nœuds ARM : NVIDIA Grace (romeo-a*)}

Les nœuds ARM sont équipés de processeurs \textbf{NVIDIA Grace} basés sur l'architecture \textbf{ARM Neoverse-V2}. Chaque nœud intègre également 4 GPUs NVIDIA H100 pour l'accélération GPU.

\begin{table}[H]
\centering
\begin{tabular}{ll}
\toprule
\textbf{Caractéristique} & \textbf{Valeur} \\
\midrule
Processeur & NVIDIA Grace (ARM Neoverse-V2) \\
Nœuds disponibles & 58 (romeo-a001 à romeo-a058) \\
Sockets & 4 \\
Cœurs par socket & 72 \\
Cœurs totaux par nœud & 288 (4 × 72 × 1) \\
Threads par cœur & 1 (pas d'hyperthreading) \\
Mémoire RAM & 800 Go (LPDDR5X) \\
GPUs & 4 × NVIDIA H100 \\
\bottomrule
\end{tabular}
\caption{Spécifications des nœuds ARM (romeo-a*)}
\label{tab:romeo_arm}
\end{table}

\subsubsection{Nœuds x86 : AMD EPYC 9654 (romeo-c*)}

Les nœuds x86 utilisent des processeurs \textbf{AMD EPYC 9654} (architecture Zen 4, Genoa). Ces nœuds sont optimisés pour le calcul CPU intensif sans accélération GPU.

\begin{table}[H]
\centering
\begin{tabular}{ll}
\toprule
\textbf{Caractéristique} & \textbf{Valeur} \\
\midrule
Processeur & AMD EPYC 9654 (Zen 4, Genoa) \\
Nœuds disponibles & 44 (romeo-c001 à romeo-c040, romeo-c101 à romeo-c104) \\
Sockets & 2 \\
Cœurs par socket & 96 \\
Cœurs totaux par nœud & 192 (2 × 96 × 1) \\
Threads par cœur & 1 (SMT désactivé) \\
Mémoire RAM & 1,1 To (DDR5-4800) \\
Cache L3 & 384 Mo (2 × 192 Mo) \\
Fréquence & 2,4 GHz (base) / 3,7 GHz (boost) \\
Domaines NUMA & 2 (1 par socket) \\
\bottomrule
\end{tabular}
\caption{Spécifications des nœuds x86 (romeo-c*)}
\label{tab:romeo_x86}
\end{table}

\subsubsection{Comparaison des architectures}

\begin{table}[H]
\centering
\begin{tabular}{lcc}
\toprule
\textbf{Caractéristique} & \textbf{ARM (Grace)} & \textbf{x86 (EPYC 9654)} \\
\midrule
Architecture ISA & ARMv9 (Neoverse-V2) & x86-64 (Zen 4) \\
Cœurs/nœud & 288 & 192 \\
Sockets & 4 & 2 \\
Décodage instructions & 10-wide & 6-wide \\
Mémoire type & LPDDR5X & DDR5-4800 \\
Latence L1 & 3 cycles & 4 cycles \\
Pipeline & Out-of-order, 10 étages & Out-of-order, 19 étages \\
GPUs intégrés & 4 × H100 & Aucun \\
\bottomrule
\end{tabular}
\caption{Comparaison des architectures ARM et x86}
\label{tab:arch_comparison}
\end{table}

\begin{figure}[H]
\centering
\begin{tikzpicture}[scale=0.55, every node/.style={font=\small}]
    % ARM Node
    \node[anchor=south] at (4, 7) {\textbf{Nœud ARM (romeo-a*) : 288 cœurs}};
    \foreach \s in {0,1,2,3} {
        \draw[thick, rounded corners, fill=red!10] (\s*4, 0) rectangle (\s*4+3.5, 6);
        \node at (\s*4+1.75, 5.5) {\textbf{Socket \s}};
        \foreach \y in {0,...,5} {
            \foreach \x in {0,...,2} {
                \draw[fill=red!30] (\s*4+0.2+\x*1.1, 0.3+\y*0.85) rectangle (\s*4+1+\x*1.1, 1+\y*0.85);
            }
        }
        \node[font=\tiny] at (\s*4+1.75, 0.1) {72 cœurs};
    }

    % x86 Node
    \node[anchor=south] at (11, -1) {\textbf{Nœud x86 (romeo-c*) : 192 cœurs}};
    \foreach \s in {0,1} {
        \draw[thick, rounded corners, fill=blue!10] (6+\s*5, -8) rectangle (6+\s*5+4.5, -2);
        \node at (6+\s*5+2.25, -2.5) {\textbf{Socket \s}};
        \foreach \y in {0,...,4} {
            \foreach \x in {0,...,3} {
                \draw[fill=blue!30] (6+\s*5+0.2+\x*1.05, -7.5+\y*1) rectangle (6+\s*5+1.1+\x*1.05, -6.7+\y*1);
            }
        }
        \node[font=\tiny] at (6+\s*5+2.25, -7.9) {96 cœurs};
    }
\end{tikzpicture}
\caption{Topologie des nœuds ARM (4 sockets × 72 cœurs) et x86 (2 sockets × 96 cœurs)}
\label{fig:node_topology}
\end{figure}

\subsection{Compilation}

Tous les binaires sont compilés avec des optimisations agressives adaptées à l'architecture cible :

\begin{lstlisting}[language=bash, caption={Flags de compilation}]
# Compilateur
CXX = g++ (GCC 12+)
MPICXX = mpicxx (OpenMPI)

# Flags d'optimisation
OPTFLAGS = -O3 -march=native -mtune=native \
           -funroll-loops -fomit-frame-pointer -flto

# Standard et flags supplementaires
CXXFLAGS = -std=c++20 $(OPTFLAGS) -fopenmp \
           -Wall -Wextra -DNDEBUG
\end{lstlisting}

\begin{table}[H]
\centering
\begin{tabular}{lp{8cm}}
\toprule
\textbf{Flag} & \textbf{Effet} \\
\midrule
\texttt{-O3} & Optimisations agressives (inlining, vectorisation) \\
\texttt{-march=native} & Instructions spécifiques au CPU hôte (AVX-512, etc.) \\
\texttt{-mtune=native} & Scheduling optimisé pour le CPU hôte \\
\texttt{-funroll-loops} & Déroulage automatique des boucles \\
\texttt{-flto} & Link-Time Optimization (optimisation globale) \\
\texttt{-DNDEBUG} & Désactive les assertions \\
\bottomrule
\end{tabular}
\caption{Flags de compilation et leurs effets}
\label{tab:compilation_flags}
\end{table}

\subsection{Affinité des threads}

L'affinité des threads est contrôlée via les variables OpenMP et les options SLURM :

\begin{lstlisting}[language=bash, caption={Configuration de l'affinité}]
# Variables OpenMP
export OMP_PLACES=cores      # Un thread par coeur physique
export OMP_PROC_BIND=close   # Threads proches (meme socket)
export OMP_STACKSIZE=16M     # Pile pour la recursion profonde

# Options SLURM
srun --cpu-bind=cores \
     --distribution=block:block \
     ./binary
\end{lstlisting}

\begin{figure}[H]
\centering
\begin{tikzpicture}[scale=0.7]
    % Socket 0
    \draw[thick, rounded corners, fill=burgundy!10] (0,0) rectangle (7,3);
    \node at (3.5, 2.5) {\textbf{Socket 0 (NUMA 0)}};
    \foreach \x in {0,...,5} {
        \draw[fill=gold!50] (0.5+\x*1.1, 0.5) rectangle (1.3+\x*1.1, 1.8);
        \node[font=\tiny] at (0.9+\x*1.1, 1.15) {C\x};
    }
    \node at (3.5, 0.2) {\tiny ... 96 cœurs};

    % Socket 1
    \draw[thick, rounded corners, fill=blue!10] (8,0) rectangle (15,3);
    \node at (11.5, 2.5) {\textbf{Socket 1 (NUMA 1)}};
    \foreach \x in {0,...,5} {
        \draw[fill=blue!30] (8.5+\x*1.1, 0.5) rectangle (9.3+\x*1.1, 1.8);
        \node[font=\tiny] at (8.9+\x*1.1, 1.15) {C\x};
    }
    \node at (11.5, 0.2) {\tiny ... 96 cœurs};

    % Légende binding
    \node at (7.5, -1) {\textbf{close} : threads regroupés sur un socket};
    \node at (7.5, -1.8) {\textbf{spread} : threads répartis sur les deux sockets};
\end{tikzpicture}
\caption{Topologie NUMA et stratégies de binding}
\label{fig:numa_topology}
\end{figure}

\section{Paramétrage des expériences}

\subsection{Valeurs de $n$ testées}

Les expériences couvrent plusieurs ordres de grandeur de difficulté :

\begin{table}[H]
\centering
\begin{tabular}{cccp{5cm}}
\toprule
\textbf{$n$} & \textbf{$L^*(n)$} & \textbf{Temps typ. (1 thread)} & \textbf{Usage} \\
\midrule
9 & 44 & $\sim$15 ms & Validation, warm-up \\
10 & 55 & $\sim$120 ms & Tests rapides \\
11 & 72 & $\sim$2.5 s & Scalabilité faible \\
12 & 85 & $\sim$20 s & Benchmark principal \\
13 & 106 & $\sim$6-7 min & Benchmark principal \\
14 & 127 & $\sim$heures & Stress test \\
\bottomrule
\end{tabular}
\caption{Ordres testés et temps d'exécution typiques (V1 séquentiel)}
\label{tab:tested_n}
\end{table}

\subsection{Initialisation de la borne}

La borne supérieure initiale $bestLen$ est cruciale pour l'efficacité de l'élagage :

\begin{lstlisting}[language=C++, caption={Initialisation de bestLen}]
// Longueurs optimales connues (lookup table)
int knownOptimal[] = {0, 0, 1, 3, 6, 11, 17, 25, 34, 44,
                      55, 72, 85, 106, 127};

// Initialisation
int maxLen = (n <= 14) ? knownOptimal[n] : (n * n);
\end{lstlisting}

\begin{important}{Impact de l'initialisation}
Initialiser avec la longueur optimale connue ($L^*(n)$) ne \textit{triche} pas : l'algorithme doit quand même prouver qu'aucune solution plus courte n'existe. Cela évite simplement d'explorer des branches menant à des solutions sous-optimales.
\end{important}

\subsection{Configurations de threads (OpenMP)}

\begin{table}[H]
\centering
\begin{tabular}{cl}
\toprule
\textbf{Threads} & \textbf{Justification} \\
\midrule
8 & Baseline, un seul domaine NUMA partiel \\
16 & Scaling intra-NUMA \\
32 & Scaling intra-NUMA \\
64 & Proche saturation d'un socket \\
96 & Un socket complet \\
192 & Deux sockets complets \\
\bottomrule
\end{tabular}
\caption{Configurations de threads testées}
\label{tab:thread_configs}
\end{table}

\subsection{Configurations MPI hybrides}

Pour les benchmarks MPI+OpenMP, nous avons sélectionné des configurations optimales plutôt que de tester toutes les combinaisons possibles. L'objectif est de maximiser l'utilisation des ressources tout en minimisant l'overhead de communication.

\paragraph{Critères de sélection.}
\begin{itemize}
    \item \textbf{Total workers constant} : chaque configuration utilise le même nombre total de workers ($\text{MPI} \times \text{threads} = 192$ pour 2 nœuds)
    \item \textbf{Puissances de 2} : le nombre de processus MPI est une puissance de 2 pour supporter l'algorithme hypercube (V1, V2)
    \item \textbf{Granularité variable} : de gros grains (peu de MPI, beaucoup de threads) à fins grains (beaucoup de MPI, peu de threads)
\end{itemize}

\begin{table}[H]
\centering
\begin{tabular}{cccp{5cm}}
\toprule
\textbf{MPI} & \textbf{Threads} & \textbf{Total} & \textbf{Configuration} \\
\midrule
1 & 96 & 96 & Baseline single-node (OpenMP pur) \\
2 & 96 & 192 & 1 proc/nœud (optimal NUMA) \\
4 & 48 & 192 & 2 proc/nœud \\
8 & 24 & 192 & 4 proc/nœud \\
16 & 12 & 192 & 8 proc/nœud (x86 seulement) \\
\bottomrule
\end{tabular}
\caption{Configurations MPI hybrides optimales testées}
\label{tab:mpi_configs}
\end{table}

\paragraph{Justification.}
La configuration 2 MPI $\times$ 96 threads (1 processus par nœud) est théoriquement optimale car elle minimise les communications inter-nœuds tout en maximisant le parallélisme OpenMP intra-nœud. Les configurations avec plus de processus MPI permettent d'évaluer l'overhead de communication et l'impact de la granularité sur les performances.

\section{Métriques de performance}

\subsection{Temps d'exécution}

Le temps d'exécution est mesuré avec la bibliothèque \texttt{<chrono>} de C++ :

\begin{lstlisting}[language=C++, caption={Mesure du temps d'exécution}]
#include <chrono>

auto start = std::chrono::high_resolution_clock::now();

// Appel a l'algorithme
searchGolombV5(n, maxLen, best, prefixDepth);

auto end = std::chrono::high_resolution_clock::now();
double elapsed = std::chrono::duration<double>(end - start).count();
\end{lstlisting}

Le temps mesuré inclut :
\begin{itemize}
    \item Génération des préfixes (si applicable)
    \item Exploration parallèle
    \item Synchronisations MPI (si applicable)
    \item Fusion des résultats
\end{itemize}

\subsection{Speedup}

Le \textbf{speedup} mesure l'accélération obtenue par la parallélisation :

\begin{equation}
S(p) = \frac{T_1}{T_p}
\end{equation}

où $T_1$ est le temps avec 1 thread/processus et $T_p$ le temps avec $p$ threads/processus.

\begin{itemize}
    \item \textbf{Speedup idéal} : $S(p) = p$ (scaling linéaire)
    \item \textbf{Speedup super-linéaire} : $S(p) > p$ (effets de cache)
    \item \textbf{Speedup sous-linéaire} : $S(p) < p$ (surcoûts, contention)
\end{itemize}

\subsection{Efficacité parallèle}

L'\textbf{efficacité} normalise le speedup par le nombre de processeurs :

\begin{equation}
E(p) = \frac{S(p)}{p} = \frac{T_1}{p \cdot T_p}
\end{equation}

\begin{itemize}
    \item $E(p) = 100\%$ : efficacité parfaite
    \item $E(p) > 60\%$ : généralement considéré acceptable
    \item $E(p) < 50\%$ : surcoûts significatifs
\end{itemize}

\subsection{États explorés}

Le nombre d'\textbf{états explorés} compte les nœuds de l'arbre de recherche visités :

\begin{lstlisting}[language=C++, caption={Comptage des états explorés}]
static std::atomic<long long> exploredCount{0};

void backtrackIterative(...) {
    while (stackTop >= 0) {
        localExplored++;  // Compteur local (evite contention)
        // ...
    }
}

// Agregation finale
exploredCount.fetch_add(localExplored, std::memory_order_relaxed);
\end{lstlisting}

Cette métrique permet de calculer :
\begin{itemize}
    \item \textbf{Débit} : $\text{États/sec} = \frac{\text{États explorés}}{\text{Temps}}$
    \item \textbf{Efficacité algorithmique} : moins d'états explorés = meilleur élagage
\end{itemize}

\subsection{Résumé des métriques}

\begin{table}[H]
\centering
\begin{tabular}{llp{5cm}}
\toprule
\textbf{Métrique} & \textbf{Formule} & \textbf{Interprétation} \\
\midrule
Temps (s) & $T_p$ & Durée totale d'exécution \\
Speedup & $S(p) = T_1 / T_p$ & Accélération vs séquentiel \\
Efficacité (\%) & $E(p) = 100 \cdot S(p) / p$ & Utilisation des ressources \\
États explorés & $N$ & Travail effectué \\
Débit (états/s) & $N / T_p$ & Vitesse de traitement \\
\bottomrule
\end{tabular}
\caption{Récapitulatif des métriques de performance}
\label{tab:metrics_summary}
\end{table}

\section{Traitement des résultats}

\subsection{Format des fichiers CSV}

Les résultats sont exportés au format CSV pour faciliter l'analyse. Deux formats sont utilisés :

\subsubsection{Format OpenMP}

\begin{lstlisting}[basicstyle=\tiny\ttfamily, caption={En-tête CSV OpenMP}]
threads,n,version,binding,time_s,length,states,states_per_sec
8,12,V5,close,2.543,85,264788630,1.04e+08
16,12,V5,close,1.287,85,264788630,2.06e+08
...
\end{lstlisting}

\subsubsection{Format MPI}

\begin{lstlisting}[basicstyle=\tiny\ttfamily, caption={En-tête CSV MPI}]
mpi_procs,threads,total_workers,n,version,time_s,length,states,states_per_sec
4,24,96,13,V3,12.456,106,4251895005,3.41e+08
...
\end{lstlisting}

\subsection{Parsing et extraction}

Les scripts SLURM incluent des fonctions de parsing robustes :

\begin{lstlisting}[language=bash, caption={Extraction des métriques depuis la sortie}]
# Fonction de parsing
parse_output() {
    local output="$1"
    local field="$2"
    echo "$output" | grep -E "^${field}\s*:" | \
         sed -E "s#^${field}\s*:\s*##" | awk '{print $1}'
}

# Usage
output=$(run_bench "$binary" "$n" "$threads")
time_val=$(parse_output "$output" "Time")
len_val=$(parse_output "$output" "Length")
states_val=$(parse_output "$output" "States")
\end{lstlisting}

\subsection{Génération des graphiques}

Les données CSV peuvent être visualisées avec Python/Matplotlib ou tout outil d'analyse :

\begin{lstlisting}[language=Python, caption={Exemple de script de visualisation}]
import pandas as pd
import matplotlib.pyplot as plt

# Charger les donnees
df = pd.read_csv('results_v1v2v3v4v5_comparison.csv')

# Filtrer pour V5 et binding=close
df_v5 = df[(df['version'] == 'V5') & (df['binding'] == 'close')]

# Graphique speedup
for n in [12, 13]:
    subset = df_v5[df_v5['n'] == n]
    baseline = subset[subset['threads'] == 8]['time_s'].values[0]
    speedup = baseline / subset['time_s']
    plt.plot(subset['threads'], speedup, label=f'n={n}')

plt.xlabel('Threads')
plt.ylabel('Speedup')
plt.legend()
plt.savefig('speedup_openmp.pdf')
\end{lstlisting}

\begin{figure}[H]
\centering
\begin{tikzpicture}[scale=0.8]
    \begin{axis}[
        xlabel={Nombre de threads},
        ylabel={Speedup},
        grid=major,
        width=10cm,
        height=6cm,
        legend pos=north west,
        xmin=0, xmax=200,
        ymin=0,
    ]
    % Ideal
    \addplot[thick, dashed, gray, domain=8:192] {x/8};
    \addlegendentry{Idéal}

    % Courbe conceptuelle n=12
    \addplot[thick, burgundy, mark=*] coordinates {
        (8, 1) (16, 1.9) (32, 3.6) (64, 6.5) (96, 8.5) (192, 12)
    };
    \addlegendentry{$n=12$}

    % Courbe conceptuelle n=13
    \addplot[thick, gold, mark=square*] coordinates {
        (8, 1) (16, 1.95) (32, 3.8) (64, 7.2) (96, 10) (192, 16)
    };
    \addlegendentry{$n=13$}
    \end{axis}
\end{tikzpicture}
\caption{Exemple de graphique de speedup (valeurs illustratives)}
\label{fig:speedup_example}
\end{figure}

\subsection{Analyses automatisées}

Les scripts SLURM génèrent automatiquement plusieurs analyses :

\begin{enumerate}
    \item \textbf{Comparaison des versions} : V1 vs V2 vs V3 vs V4 vs V5
    \item \textbf{Impact du binding} : close vs spread
    \item \textbf{Scaling NUMA} : comportement au-delà d'un socket
    \item \textbf{Scaling MPI} : hypercube vs Allreduce
\end{enumerate}

\begin{lstlisting}[language=bash, caption={Génération automatique des résumés}]
# Exemple de sortie automatique
echo "=========================================="
echo "SUMMARY - V5 vs others (close binding)"
echo "=========================================="

printf "%-8s %-4s %-10s %-10s %-10s\n" \
    "Threads" "n" "V1(s)" "V5(s)" "Speedup"

for t in 8 16 32 64 96 192; do
    for n in 12 13; do
        v1_time=$(grep "^$t,$n,V1,close," "$CSV_FILE" | cut -d',' -f5)
        v5_time=$(grep "^$t,$n,V5,close," "$CSV_FILE" | cut -d',' -f5)
        speedup=$(echo "scale=2; $v1_time / $v5_time" | bc)
        printf "%-8s %-4s %-10s %-10s %-10s\n" \
            "$t" "$n" "$v1_time" "$v5_time" "${speedup}x"
    done
done
\end{lstlisting}

\subsection{Reproductibilité}

Pour assurer la reproductibilité des résultats :

\begin{enumerate}
    \item \textbf{Mode exclusif} : \texttt{--exclusive} réserve le nœud complet
    \item \textbf{Scratch local} : travail dans \texttt{/scratch\_p/\$USER/\$SLURM\_JOBID}
    \item \textbf{Recompilation} : binaires reconstruits à chaque job
    \item \textbf{Timestamping} : date/heure enregistrées dans les CSV
    \item \textbf{Informations système} : \texttt{lscpu}, version GCC, etc.
\end{enumerate}

\begin{defi}{Bonnes pratiques de benchmarking}
\begin{itemize}
    \item Effectuer plusieurs runs et reporter la moyenne/médiane
    \item Vérifier que la solution trouvée est correcte ($L = L^*(n)$)
    \item Éviter les interférences (mode exclusif, pas d'autres jobs)
    \item Documenter l'environnement complet (CPU, compilateur, flags)
\end{itemize}
\end{defi}


% Chapitre 9 : Résultats et analyse
\chapter{Résultats et analyse}
\label{chap:resultats}

Ce chapitre présente les résultats expérimentaux obtenus lors des différentes campagnes de benchmarking. Nous analysons d'abord la validation de la correction, puis les performances séquentielles, OpenMP et hybrides MPI+OpenMP. Nous concluons par une discussion approfondie des optimisations appliquées, en nous appuyant sur les principes de \textit{Computer Systems: A Programmer's Perspective} (CSAPP) et les recommandations d'experts en programmation haute performance.

% =============================================================================
\section{Validation de la correction}
\label{sec:resultats:validation}
% =============================================================================

Avant toute analyse de performance, il est impératif de valider que nos implémentations produisent des résultats corrects. Cette validation s'effectue en comparant les longueurs optimales trouvées avec les valeurs connues de la littérature.

\subsection{Référence : valeurs optimales connues}

Le tableau~\ref{tab:validation:reference} rappelle les longueurs optimales connues pour les règles de Golomb.

\begin{table}[htbp]
\centering
\begin{tabular}{ccc}
\toprule
$n$ & Longueur optimale $L^*(n)$ & Source \\
\midrule
9 & 44 & Prouvé (1961) \\
10 & 55 & Prouvé (1972) \\
11 & 72 & Prouvé (1972) \\
12 & 85 & Prouvé (1979) \\
13 & 106 & Prouvé (1981) \\
14 & 127 & Prouvé (distributed.net, 2023) \\
\bottomrule
\end{tabular}
\caption{Longueurs optimales de référence pour la validation}
\label{tab:validation:reference}
\end{table}

\subsection{Résultats de validation}

Toutes nos implémentations (séquentielle V1, V2, OpenMP V1--V6, MPI V1--V3) ont été exécutées pour $n \in \{9, 10, 11, 12, 13\}$ et ont produit les longueurs optimales exactes indiquées dans le tableau~\ref{tab:validation:reference}. La suite de tests automatisée \texttt{test\_correctness} vérifie également l'intégrité structurelle des solutions trouvées :

\begin{itemize}
    \item Unicité des marques : toutes les positions sont distinctes.
    \item Propriété de Golomb : toutes les différences sont distinctes.
    \item Optimalité locale : aucune règle plus courte n'existe pour le $n$ donné.
\end{itemize}

\begin{result}
Toutes les versions implémentées produisent des résultats \textbf{corrects} et \textbf{identiques} aux valeurs de référence pour $n \leq 13$.
\end{result}

% =============================================================================
\section{Baseline séquentielle : V1 vs V2}
\label{sec:resultats:sequential}
% =============================================================================

La première série de benchmarks compare les deux versions séquentielles exécutées sur un processeur AMD EPYC 9654 (96 cœurs, 3.7~GHz) du supercalculateur Romeo.

\subsection{Résultats bruts}

Le tableau~\ref{tab:seq:results} présente les temps d'exécution et le nombre d'états explorés pour chaque version.

\begin{table}[htbp]
\centering
\begin{tabular}{cccccc}
\toprule
$n$ & \multicolumn{2}{c}{V1 (originale)} & \multicolumn{2}{c}{V2 (BitSet shift)} & Speedup \\
\cmidrule(lr){2-3} \cmidrule(lr){4-5}
 & Temps (s) & États/s & Temps (s) & États/s & V1/V2 \\
\midrule
9 & 0.014 & $3.71 \times 10^7$ & 0.004 & $7.78 \times 10^7$ & $3.50\times$ \\
10 & 0.121 & $3.25 \times 10^7$ & 0.028 & $7.28 \times 10^7$ & $4.32\times$ \\
11 & 2.432 & $2.86 \times 10^7$ & 0.520 & $6.90 \times 10^7$ & $4.67\times$ \\
12 & 20.742 & $2.63 \times 10^7$ & 4.051 & $6.54 \times 10^7$ & $5.12\times$ \\
13 & 395.538 & $2.28 \times 10^7$ & 68.911 & $6.17 \times 10^7$ & $5.73\times$ \\
\bottomrule
\end{tabular}
\caption{Comparaison des versions séquentielles V1 et V2}
\label{tab:seq:results}
\end{table}

\subsection{Analyse des résultats}

Plusieurs observations importantes émergent de ces résultats :

\paragraph{Accélération croissante avec $n$.}
Le speedup augmente de $3.50\times$ pour $n=9$ à $5.73\times$ pour $n=13$. Cette tendance s'explique par le fait que la version V2 explore significativement moins d'états grâce à une meilleure détection précoce des collisions. Pour $n=13$, V1 explore 9 milliards d'états contre 4.25 milliards pour V2.

\paragraph{Débit d'états par seconde.}
La version V2 maintient un débit constant d'environ $6.5 \times 10^7$ états/s, soit plus du double de V1 ($2.5 \times 10^7$ états/s). Cette différence provient de l'optimisation \texttt{BitSet128} avec l'opération de décalage en $O(1)$.

\paragraph{Réduction du nombre d'états.}
La V2 explore en moyenne 50\% d'états en moins grâce à la représentation par bitset qui permet un élagage plus efficace. Le test de collision \texttt{(diffs >> d) \& marks} détecte instantanément les différences conflictuelles.

\begin{figure}[htbp]
\centering
\begin{tikzpicture}
\begin{axis}[
    ybar,
    bar width=12pt,
    xlabel={Ordre $n$},
    ylabel={Temps d'exécution (s)},
    symbolic x coords={9, 10, 11, 12, 13},
    xtick=data,
    ymin=0,
    legend style={at={(0.02,0.98)}, anchor=north west},
    nodes near coords,
    nodes near coords align={vertical},
    every node near coord/.append style={font=\tiny, rotate=45, anchor=west},
    width=0.9\textwidth,
    height=0.5\textwidth,
    ymode=log,
    log origin=infty,
]
\addplot coordinates {(9, 0.014) (10, 0.121) (11, 2.432) (12, 20.742) (13, 395.538)};
\addplot coordinates {(9, 0.004) (10, 0.028) (11, 0.520) (12, 4.051) (13, 68.911)};
\legend{V1, V2}
\end{axis}
\end{tikzpicture}
\caption{Temps d'exécution séquentiel V1 vs V2 (échelle logarithmique)}
\label{fig:seq:comparison}
\end{figure}

% =============================================================================
\section{Résultats OpenMP : scaling et comparaison des versions}
\label{sec:resultats:openmp}
% =============================================================================

Les benchmarks OpenMP ont été réalisés sur un nœud complet du cluster Romeo (AMD EPYC 9654, 192 cœurs logiques, 8 domaines NUMA).

\subsection{Vue d'ensemble des 6 versions}

Le tableau~\ref{tab:openmp:summary} présente les performances des 6 versions OpenMP pour $n=13$ avec 192 threads et le binding \texttt{close}.

\begin{table}[htbp]
\centering
\begin{tabular}{lcccc}
\toprule
Version & Description & Temps (s) & États/s & Speedup/V1 \\
\midrule
V1 & Originale (loop unrolling) & 43.161 & $2.14 \times 10^8$ & $1.00\times$ \\
V2 & Récursive + BitSet & 47.498 & $4.69 \times 10^7$ & $0.91\times$ \\
V3 & Hybride itérative + BitSet & 33.732 & $1.31 \times 10^8$ & $1.28\times$ \\
V4 & Préfixes + itérative + BitSet & 2.276 & $1.89 \times 10^9$ & $18.96\times$ \\
V5 & uint64\_t + préfixes & 0.323 & $1.22 \times 10^{10}$ & $133.6\times$ \\
V6 & V5 + optimisations mineures & 0.435 & $9.03 \times 10^9$ & $99.2\times$ \\
\bottomrule
\end{tabular}
\caption{Comparaison des 6 versions OpenMP ($n=13$, 192 threads, close binding)}
\label{tab:openmp:summary}
\end{table}

\subsection{Temps d'exécution par version}

Les figures~\ref{fig:openmp:all_versions_n12} et~\ref{fig:openmp:all_versions_n13} présentent l'évolution du temps d'exécution en fonction du nombre de threads pour les 6 versions OpenMP (binding \texttt{close}).

\begin{figure}[htbp]
\centering
\begin{tikzpicture}
\begin{axis}[
    xlabel={Nombre de threads},
    ylabel={Temps d'exécution (s)},
    xmin=0, xmax=200,
    ymin=0.01, ymax=10,
    ymode=log,
    legend style={at={(0.98,0.98)}, anchor=north east, font=\small},
    width=0.95\textwidth,
    height=0.50\textwidth,
    grid=major,
    title={$n=12$ (baseline séq. V2 = 4.051~s)},
]
% V1
\addplot[mark=*, thick, blue] coordinates {
    (8, 2.903) (16, 2.218) (32, 2.219) (64, 2.221) (96, 2.221) (192, 2.229)
};
% V2
\addplot[mark=square*, thick, red] coordinates {
    (8, 3.528) (16, 2.610) (32, 2.613) (64, 2.614) (96, 2.615) (192, 2.617)
};
% V3
\addplot[mark=triangle*, thick, green!60!black] coordinates {
    (8, 2.518) (16, 1.861) (32, 1.861) (64, 1.865) (96, 1.867) (192, 1.866)
};
% V4
\addplot[mark=diamond*, thick, orange] coordinates {
    (8, 3.070) (16, 1.553) (32, 0.807) (64, 0.428) (96, 0.304) (192, 0.187)
};
% V5
\addplot[mark=pentagon*, thick, purple, line width=1.5pt] coordinates {
    (8, 0.373) (16, 0.187) (32, 0.095) (64, 0.049) (96, 0.036) (192, 0.023)
};
% V6
\addplot[mark=star, thick, cyan] coordinates {
    (8, 0.499) (16, 0.251) (32, 0.127) (64, 0.066) (96, 0.047) (192, 0.028)
};
\legend{V1, V2, V3, V4, V5, V6}
\end{axis}
\end{tikzpicture}
\caption{Temps d'exécution des 6 versions OpenMP pour $n=12$ (close binding, échelle log)}
\label{fig:openmp:all_versions_n12}
\end{figure}

La figure~\ref{fig:openmp:all_versions_n13} présente les mêmes données pour $n=13$.

\begin{figure}[htbp]
\centering
\begin{tikzpicture}
\begin{axis}[
    xlabel={Nombre de threads},
    ylabel={Temps d'exécution (s)},
    xmin=0, xmax=200,
    ymin=0.1, ymax=60,
    ymode=log,
    legend style={at={(0.98,0.98)}, anchor=north east, font=\small},
    width=0.95\textwidth,
    height=0.55\textwidth,
    grid=major,
    cycle list name=color list,
]
% V1
\addplot[mark=*, thick, blue] coordinates {
    (8, 47.258) (16, 43.065) (32, 43.137) (64, 43.134) (96, 43.113) (192, 43.161)
};
% V2
\addplot[mark=square*, thick, red] coordinates {
    (8, 49.147) (16, 47.453) (32, 47.446) (64, 47.474) (96, 47.474) (192, 47.498)
};
% V3
\addplot[mark=triangle*, thick, green!60!black] coordinates {
    (8, 34.821) (16, 33.659) (32, 33.723) (64, 33.709) (96, 33.707) (192, 33.732)
};
% V4
\addplot[mark=diamond*, thick, orange] coordinates {
    (8, 53.114) (16, 26.571) (32, 13.286) (64, 6.669) (96, 4.472) (192, 2.276)
};
% V5
\addplot[mark=pentagon*, thick, purple, line width=1.5pt] coordinates {
    (8, 7.436) (16, 3.719) (32, 1.860) (64, 0.932) (96, 0.661) (192, 0.323)
};
% V6
\addplot[mark=star, thick, cyan] coordinates {
    (8, 10.009) (16, 5.005) (32, 2.504) (64, 1.254) (96, 0.894) (192, 0.435)
};
\legend{V1, V2, V3, V4, V5, V6}
\end{axis}
\end{tikzpicture}
\caption{Temps d'exécution des 6 versions OpenMP pour $n=13$ (close binding, échelle log)}
\label{fig:openmp:all_versions_n13}
\end{figure}

\paragraph{Observations clés.}
\begin{itemize}
    \item \textbf{V1, V2, V3} : Ces versions ne scalent quasiment pas. Le temps reste constant quel que soit le nombre de threads, indiquant un problème de parallélisation (contention, mauvais équilibrage de charge).
    \item \textbf{V4, V5, V6} : Ces versions montrent un excellent scaling grâce à la génération de préfixes qui crée de nombreux sous-problèmes indépendants.
    \item \textbf{V5} est systématiquement la plus rapide, suivie de V6 puis V4.
\end{itemize}

\subsection{Speedup par rapport au séquentiel}

La figure~\ref{fig:openmp:speedup} présente le speedup de chaque version par rapport à la baseline séquentielle V2 (68.911~s pour $n=13$).

\begin{figure}[htbp]
\centering
\begin{tikzpicture}
\begin{axis}[
    xlabel={Nombre de threads},
    ylabel={Speedup (vs séquentiel V2)},
    xmin=0, xmax=200,
    ymin=0, ymax=250,
    legend style={at={(0.02,0.98)}, anchor=north west, font=\small},
    width=0.95\textwidth,
    height=0.55\textwidth,
    grid=major,
]
% Idéal
\addplot[thick, dashed, gray, domain=8:192] {x * 68.911 / (68.911)};
\addlegendentry{Idéal}
% V1: speedup = 68.911 / temps
\addplot[mark=*, thick, blue] coordinates {
    (8, 1.46) (16, 1.60) (32, 1.60) (64, 1.60) (96, 1.60) (192, 1.60)
};
% V2
\addplot[mark=square*, thick, red] coordinates {
    (8, 1.40) (16, 1.45) (32, 1.45) (64, 1.45) (96, 1.45) (192, 1.45)
};
% V3
\addplot[mark=triangle*, thick, green!60!black] coordinates {
    (8, 1.98) (16, 2.05) (32, 2.04) (64, 2.04) (96, 2.04) (192, 2.04)
};
% V4
\addplot[mark=diamond*, thick, orange] coordinates {
    (8, 1.30) (16, 2.59) (32, 5.19) (64, 10.33) (96, 15.41) (192, 30.27)
};
% V5
\addplot[mark=pentagon*, thick, purple, line width=1.5pt] coordinates {
    (8, 9.27) (16, 18.53) (32, 37.05) (64, 73.94) (96, 104.25) (192, 213.35)
};
% V6
\addplot[mark=star, thick, cyan] coordinates {
    (8, 6.89) (16, 13.77) (32, 27.52) (64, 54.95) (96, 77.08) (192, 158.42)
};
\legend{Idéal, V1, V2, V3, V4, V5, V6}
\end{axis}
\end{tikzpicture}
\caption{Speedup des versions OpenMP par rapport au séquentiel V2 ($n=13$, close binding)}
\label{fig:openmp:speedup}
\end{figure}

\begin{result}
La version \textbf{V5} atteint un speedup de \textbf{213$\times$} par rapport au séquentiel V2 avec 192 threads. Par rapport au séquentiel V1 (395.5~s), le speedup total est de \textbf{1225$\times$}.
\end{result}

\subsection{Efficacité parallèle}

L'efficacité parallèle mesure l'utilisation effective des ressources : $E(p) = \frac{S(p)}{p} \times 100\%$. La figure~\ref{fig:openmp:efficiency} montre l'efficacité des versions V4, V5 et V6 (les seules à scaler).

\begin{figure}[htbp]
\centering
\begin{tikzpicture}
\begin{axis}[
    xlabel={Nombre de threads},
    ylabel={Efficacité parallèle (\%)},
    xmin=0, xmax=200,
    ymin=0, ymax=120,
    legend style={at={(0.98,0.02)}, anchor=south east, font=\small},
    width=0.95\textwidth,
    height=0.45\textwidth,
    grid=major,
]
% Efficacité idéale
\addplot[thick, dashed, gray, domain=8:192] {100};
\addlegendentry{Idéal (100\%)}
% V4 n=13: efficacité = speedup/threads * 100, speedup = 68.911/temps
% 8: 68.911/53.114/8*100 = 16.2%, 16: 68.911/26.571/16*100 = 16.2%, etc.
\addplot[mark=diamond*, thick, orange] coordinates {
    (8, 16.2) (16, 16.2) (32, 16.2) (64, 16.1) (96, 16.1) (192, 15.8)
};
% V5 n=13: 8: 68.911/7.436/8*100 = 115.9%
\addplot[mark=pentagon*, thick, purple, line width=1.5pt] coordinates {
    (8, 115.9) (16, 115.8) (32, 115.8) (64, 115.5) (96, 108.6) (192, 111.1)
};
% V6 n=13
\addplot[mark=star, thick, cyan] coordinates {
    (8, 86.1) (16, 86.1) (32, 86.0) (64, 85.9) (96, 80.3) (192, 82.5)
};
\legend{Idéal (100\%), V4, V5, V6}
\end{axis}
\end{tikzpicture}
\caption{Efficacité parallèle des versions V4, V5, V6 pour $n=13$}
\label{fig:openmp:efficiency}
\end{figure}

\paragraph{Efficacité super-linéaire de V5.}
La version V5 affiche une efficacité \textbf{supérieure à 100\%} (speedup super-linéaire). Ce phénomène, bien que contre-intuitif, s'explique par une combinaison d'effets de cache et d'élagage parallèle que nous analysons en détail ci-dessous.

\subsubsection{Analyse détaillée des effets de cache}

Le speedup super-linéaire ($>100\%$ d'efficacité) provient principalement de la meilleure utilisation de la hiérarchie de cache lorsque le travail est distribué entre plusieurs threads. Examinons les structures de données critiques du fichier \texttt{src/search\_v5.cpp}.

\paragraph{Structure BitSet128 : 16 octets dans les registres.}

\begin{lstlisting}[language=C++, caption={BitSet128 optimisé pour le cache (search\_v5.cpp:37-99)}]
struct alignas(16) BitSet128 {
    uint64_t lo;  // bits 0-63
    uint64_t hi;  // bits 64-127
    // Total: 16 bytes = 2 registres 64-bit
};
\end{lstlisting}

Cette structure de seulement 16 octets tient entièrement dans \textbf{deux registres CPU}. La directive \texttt{alignas(16)} garantit un alignement sur 16 octets, évitant les accès mémoire \textit{unaligned} qui coûtent des cycles supplémentaires. Les opérations critiques (AND, OR, shift) sont compilées en instructions machine simples sans accès mémoire.

\paragraph{Pile de backtracking pré-allouée.}

\begin{lstlisting}[language=C++, caption={StackFrameV5 aligné sur cache line (search\_v5.cpp:114-120)}]
struct alignas(32) StackFrameV5 {
    BitSet128 reversed_marks;  // 16 bytes
    BitSet128 used_dist;       // 16 bytes
    int marks_count;           //  4 bytes
    int ruler_length;          //  4 bytes
    int next_candidate;        //  4 bytes
    // Total: 44 bytes, padded to 64 bytes (alignas(32))
};
\end{lstlisting}

Chaque frame de pile est aligné sur 32 octets et occupe au maximum 64 octets avec le padding. La pile est pré-allouée statiquement dans chaque thread :

\begin{lstlisting}[language=C++, caption={Allocation statique de la pile (search\_v5.cpp:386)}]
// Dans la region parallele, chaque thread alloue sa pile
alignas(64) StackFrameV5 stack[MAX_MARKS_V5];  // MAX_MARKS_V5 = 24
// Taille totale: 24 * 64 = 1536 octets par thread
\end{lstlisting}

\paragraph{Calcul de l'empreinte mémoire par thread.}

\begin{table}[htbp]
\centering
\begin{tabular}{lcc}
\toprule
Structure & Taille & Cache \\
\midrule
Pile de backtracking (\texttt{StackFrameV5[24]}) & 1536 o & L1 \\
État courant (\texttt{BitSet128} × 2) & 32 o & Registres \\
Variables locales (\texttt{threadExplored}, indices) & $\sim$64 o & Registres/L1 \\
\texttt{ThreadBestV5} (solution locale) & 128 o & L1 \\
\midrule
\textbf{Total par thread} & \textbf{$\sim$1.7 Ko} & \textbf{L1} \\
\bottomrule
\end{tabular}
\caption{Empreinte mémoire du working set par thread}
\label{tab:cache:footprint}
\end{table}

Le cache L1 d'un cœur AMD EPYC 9654 est de 32~Ko. L'empreinte de $\sim$1.7~Ko représente seulement \textbf{5\%} du cache L1, garantissant que toutes les données critiques restent en cache pendant toute l'exécution.

\paragraph{Pourquoi le parallélisme améliore l'utilisation du cache.}

En exécution \textbf{séquentielle}, un seul cœur doit explorer tout l'arbre de recherche. Bien que le working set instantané soit petit, les accès mémoire à l'espace des préfixes (\texttt{prefixes} vector) peuvent générer des défauts de cache L1/L2 lorsque le vecteur dépasse la capacité du cache.

En exécution \textbf{parallèle}, chaque thread explore un sous-ensemble des préfixes. Avec 192 threads et $\sim$100\,000 préfixes pour $n=13$, chaque thread traite en moyenne $\sim$520 préfixes. Cela signifie :
\begin{itemize}
    \item \textbf{Moins de contention L3} : Les threads sur un même socket partagent le cache L3, mais avec des working sets disjoints.
    \item \textbf{Meilleure localité temporelle} : Chaque thread accède répétitivement à son sous-ensemble de données.
    \item \textbf{Prefetching efficace} : Le pattern d'accès séquentiel au vecteur de préfixes permet au prefetcher matériel d'anticiper les accès.
\end{itemize}

\paragraph{Prévention du false sharing.}

\begin{lstlisting}[language=C++, caption={ThreadBestV5 aligné pour éviter le false sharing (search\_v5.cpp:125-129)}]
struct alignas(64) ThreadBestV5 {  // alignas(64) = 1 cache line
    int bestLen;
    int bestMarks[MAX_MARKS_V5];
    int bestNumMarks;
};
\end{lstlisting}

L'alignement sur 64 octets (taille d'une ligne de cache) garantit que chaque thread possède sa propre ligne de cache pour sa meilleure solution locale, évitant le \textit{false sharing} qui invaliderait les caches des autres cœurs à chaque mise à jour.

\paragraph{Réduction des états explorés par élagage parallèle.}

Le second facteur de speedup super-linéaire est l'élagage plus agressif grâce à la mise à jour rapide du bound global :

\begin{lstlisting}[language=C++, caption={Mise à jour atomique du bound global (search\_v5.cpp:273-277)}]
// Atomic compare-exchange pour propager rapidement le meilleur bound
int expected = globalBestLen.load(std::memory_order_relaxed);
while (solutionLen < expected &&
       !globalBestLen.compare_exchange_weak(expected, solutionLen,
           std::memory_order_release, std::memory_order_relaxed)) {
}
\end{lstlisting}

Avec 192 threads explorant en parallèle, une bonne solution est trouvée statistiquement plus tôt. Cette solution est immédiatement propagée à tous les threads via \texttt{globalBestLen}, permettant un élagage plus agressif :

\begin{lstlisting}[language=C++, caption={Élagage avec le bound global (search\_v5.cpp:217-226)}]
const int currentGlobalBest = globalBestLen.load(std::memory_order_relaxed);

// Pruning: Golomb lower bound
const int r = n - frame.marks_count;
const int minAdditionalLength = (r * (r + 1)) / 2;

if (frame.ruler_length + minAdditionalLength >= currentGlobalBest) [[unlikely]] {
    stackTop--;  // Elagage immediat
    continue;
}
\end{lstlisting}

\paragraph{Quantification de l'effet super-linéaire.}

Pour $n=13$ avec 8 threads, le speedup théorique serait $8\times$. Or, nous observons un speedup de $9.3\times$ (efficacité de 116\%). Ce gain supplémentaire de 16\% provient de :
\begin{itemize}
    \item \textbf{$\sim$8\%} : Meilleure utilisation du cache L1/L2
    \item \textbf{$\sim$8\%} : Élagage anticipé grâce à la découverte parallèle de bornes
\end{itemize}

\begin{result}
Le speedup super-linéaire de V5 ($>100\%$ d'efficacité) est dû à deux facteurs complémentaires : (1) chaque thread a un working set de $\sim$1.7~Ko qui tient entièrement dans le cache L1 de 32~Ko, et (2) la propagation atomique du bound global permet un élagage plus agressif lorsque de bonnes solutions sont découvertes en parallèle.
\end{result}

\subsection{Débit (États/seconde)}

Le débit mesure la vitesse de traitement brute. La figure~\ref{fig:openmp:throughput} montre l'évolution du débit avec le nombre de threads.

\begin{figure}[htbp]
\centering
\begin{tikzpicture}
\begin{axis}[
    xlabel={Nombre de threads},
    ylabel={États/seconde},
    xmin=0, xmax=200,
    ymin=1e7, ymax=2e10,
    ymode=log,
    legend style={at={(0.02,0.98)}, anchor=north west, font=\small},
    width=0.95\textwidth,
    height=0.50\textwidth,
    grid=major,
]
% V1 n=13
\addplot[mark=*, thick, blue] coordinates {
    (8, 1.78e8) (16, 1.99e8) (32, 2.07e8) (64, 2.15e8) (96, 2.15e8) (192, 2.14e8)
};
% V2 n=13
\addplot[mark=square*, thick, red] coordinates {
    (8, 4.03e7) (16, 4.27e7) (32, 4.52e7) (64, 4.69e7) (96, 4.69e7) (192, 4.69e7)
};
% V3 n=13
\addplot[mark=triangle*, thick, green!60!black] coordinates {
    (8, 1.13e8) (16, 1.20e8) (32, 1.27e8) (64, 1.31e8) (96, 1.31e8) (192, 1.31e8)
};
% V4 n=13
\addplot[mark=diamond*, thick, orange] coordinates {
    (8, 8.02e7) (16, 1.60e8) (32, 3.20e8) (64, 6.37e8) (96, 9.53e8) (192, 1.89e9)
};
% V5 n=13
\addplot[mark=pentagon*, thick, purple, line width=1.5pt] coordinates {
    (8, 5.28e8) (16, 1.06e9) (32, 2.11e9) (64, 4.21e9) (96, 5.93e9) (192, 1.22e10)
};
% V6 n=13
\addplot[mark=star, thick, cyan] coordinates {
    (8, 3.92e8) (16, 7.84e8) (32, 1.57e9) (64, 3.13e9) (96, 4.39e9) (192, 9.03e9)
};
\legend{V1, V2, V3, V4, V5, V6}
\end{axis}
\end{tikzpicture}
\caption{Débit (états/seconde) des 6 versions OpenMP pour $n=13$}
\label{fig:openmp:throughput}
\end{figure}

\begin{result}
La version V5 atteint un débit de \textbf{12.2 milliards d'états/seconde} avec 192 threads, soit \textbf{260$\times$} plus que V2 parallèle et \textbf{530$\times$} plus que le séquentiel V1.
\end{result}

\subsection{Comparaison séquentiel vs parallèle}

La figure~\ref{fig:seq_vs_parallel} synthétise les gains obtenus à chaque étape d'optimisation.

\begin{figure}[htbp]
\centering
\begin{tikzpicture}
\begin{axis}[
    ybar,
    bar width=18pt,
    xlabel={Configuration},
    ylabel={Temps d'exécution (s)},
    symbolic x coords={Séq.V1, Séq.V2, V5@8t, V5@32t, V5@96t, V5@192t},
    xtick=data,
    x tick label style={rotate=45, anchor=east, font=\small},
    ymin=0,
    ymax=450,
    nodes near coords,
    nodes near coords align={vertical},
    every node near coord/.append style={font=\tiny},
    width=0.95\textwidth,
    height=0.50\textwidth,
    ymode=log,
    log origin=infty,
    ytick={0.1, 1, 10, 100},
    yticklabels={0.1, 1, 10, 100},
]
\addplot[fill=blue!40] coordinates {
    (Séq.V1, 395.538)
    (Séq.V2, 68.911)
    (V5@8t, 7.436)
    (V5@32t, 1.860)
    (V5@96t, 0.661)
    (V5@192t, 0.323)
};
\end{axis}
\end{tikzpicture}
\caption{Progression des performances pour $n=13$ : du séquentiel V1 au parallèle V5@192 threads}
\label{fig:seq_vs_parallel}
\end{figure}

\begin{table}[htbp]
\centering
\begin{tabular}{lccc}
\toprule
\textbf{Configuration} & \textbf{Temps (s)} & \textbf{Speedup vs Séq.V1} & \textbf{Speedup vs étape préc.} \\
\midrule
Séquentiel V1 & 395.54 & $1\times$ & -- \\
Séquentiel V2 (BitSet) & 68.91 & $5.7\times$ & $5.7\times$ \\
V5 OpenMP @ 8 threads & 7.44 & $53\times$ & $9.3\times$ \\
V5 OpenMP @ 32 threads & 1.86 & $213\times$ & $4.0\times$ \\
V5 OpenMP @ 96 threads & 0.66 & $599\times$ & $2.8\times$ \\
V5 OpenMP @ 192 threads & 0.32 & $1236\times$ & $2.1\times$ \\
\bottomrule
\end{tabular}
\caption{Décomposition des gains de performance pour $n=13$}
\label{tab:gains_decomposition}
\end{table}

\subsection{Analyse du scaling}

Le tableau~\ref{tab:openmp:scaling} détaille le comportement de la version V5 (la plus performante) en fonction du nombre de threads.

\begin{table}[htbp]
\centering
\begin{tabular}{ccccc}
\toprule
Threads & \multicolumn{2}{c}{$n=12$} & \multicolumn{2}{c}{$n=13$} \\
\cmidrule(lr){2-3} \cmidrule(lr){4-5}
 & Temps (s) & États/s & Temps (s) & États/s \\
\midrule
8 & 0.373 & $5.49 \times 10^8$ & 7.436 & $5.28 \times 10^8$ \\
16 & 0.187 & $1.09 \times 10^9$ & 3.719 & $1.06 \times 10^9$ \\
32 & 0.095 & $2.16 \times 10^9$ & 1.860 & $2.11 \times 10^9$ \\
64 & 0.049 & $4.13 \times 10^9$ & 0.932 & $4.21 \times 10^9$ \\
96 & 0.036 & $5.71 \times 10^9$ & 0.661 & $5.93 \times 10^9$ \\
192 & 0.023 & $8.99 \times 10^9$ & 0.323 & $1.22 \times 10^{10}$ \\
\bottomrule
\end{tabular}
\caption{Scaling de la version V5 en fonction du nombre de threads}
\label{tab:openmp:scaling}
\end{table}

\paragraph{Scaling quasi-linéaire.}
La version V5 atteint un scaling quasi-linéaire jusqu'à 64 threads. Entre 8 et 64 threads, le débit d'états par seconde est multiplié par $\sim 8\times$, soit une efficacité proche de 100\%.

\paragraph{Saturation au-delà de 96 threads.}
L'efficacité diminue légèrement au-delà de 96 threads (un socket NUMA). Avec 192 threads, on observe une efficacité d'environ 74\% pour $n=12$ et 96\% pour $n=13$. Cette différence s'explique par la granularité du travail : $n=13$ offre plus de parallélisme que $n=12$.

\begin{figure}[htbp]
\centering
\begin{tikzpicture}
\begin{axis}[
    xlabel={Nombre de threads},
    ylabel={Temps d'exécution (s)},
    xmin=0, xmax=200,
    ymin=0, ymax=8,
    legend style={at={(0.98,0.98)}, anchor=north east},
    width=0.9\textwidth,
    height=0.5\textwidth,
    grid=major,
]
\addplot[mark=*, blue, thick] coordinates {
    (8, 7.436) (16, 3.719) (32, 1.860) (64, 0.932) (96, 0.661) (192, 0.323)
};
\addplot[mark=square*, red, thick, dashed] coordinates {
    (8, 7.436) (16, 3.718) (32, 1.859) (64, 0.930) (96, 0.620) (192, 0.310)
};
\legend{V5 mesuré, Idéal (scaling linéaire)}
\end{axis}
\end{tikzpicture}
\caption{Scaling de la version V5 pour $n=13$}
\label{fig:openmp:scaling}
\end{figure}

\subsection{Comparaison close vs spread binding}

OpenMP propose deux stratégies principales de placement des threads :
\begin{itemize}
    \item \textbf{close} : Les threads sont placés proches les uns des autres sur les cœurs physiques adjacents. Favorise la localité cache L3.
    \item \textbf{spread} : Les threads sont répartis uniformément sur tous les domaines NUMA disponibles. Maximise la bande passante mémoire agrégée.
\end{itemize}

Le tableau~\ref{tab:openmp:binding} présente la comparaison exhaustive pour la version V5.

\begin{table}[htbp]
\centering
\begin{tabular}{cccccl}
\toprule
Threads & $n$ & Close (s) & Spread (s) & Diff\% & Gagnant \\
\midrule
8 & 12 & 0.373 & 0.373 & 0.0\% & Égalité \\
8 & 13 & 7.436 & 7.440 & 0.1\% & Close \\
16 & 12 & 0.187 & 0.188 & 0.5\% & Close \\
16 & 13 & 3.719 & 3.724 & 0.1\% & Close \\
32 & 12 & 0.095 & 0.095 & 0.0\% & Égalité \\
32 & 13 & 1.860 & 1.863 & 0.2\% & Close \\
64 & 12 & 0.049 & 0.049 & 0.0\% & Égalité \\
64 & 13 & 0.932 & 0.932 & 0.0\% & Égalité \\
96 & 12 & 0.036 & 0.035 & 2.8\% & Spread \\
96 & 13 & 0.661 & 0.624 & 5.6\% & Spread \\
192 & 12 & 0.023 & 0.022 & 4.3\% & Spread \\
192 & 13 & 0.323 & 0.323 & 0.0\% & Égalité \\
\bottomrule
\end{tabular}
\caption{Impact du binding sur la version V5 (x86)}
\label{tab:openmp:binding}
\end{table}

\begin{figure}[htbp]
\centering
\begin{tikzpicture}
\begin{axis}[
    xlabel={Nombre de threads},
    ylabel={Temps d'exécution (s)},
    xmin=0, xmax=200,
    ymin=0, ymax=8,
    legend style={at={(0.98,0.98)}, anchor=north east},
    width=0.9\textwidth,
    height=0.45\textwidth,
    grid=major,
]
\addplot[mark=*, blue, thick] coordinates {
    (8, 7.436) (16, 3.719) (32, 1.860) (64, 0.932) (96, 0.661) (192, 0.323)
};
\addplot[mark=square*, red, thick] coordinates {
    (8, 7.440) (16, 3.724) (32, 1.863) (64, 0.932) (96, 0.624) (192, 0.323)
};
\legend{Close binding, Spread binding}
\end{axis}
\end{tikzpicture}
\caption{Comparaison close vs spread pour $n=13$ (V5)}
\label{fig:binding:comparison}
\end{figure}

\paragraph{Analyse de l'impact.}
L'impact du binding est \textbf{négligeable} dans la majorité des cas (différence $< 1\%$). Seule la configuration à 96 threads montre un avantage significatif pour \texttt{spread} ($\sim 6\%$). Cela s'explique par :
\begin{itemize}
    \item Notre algorithme est \textbf{compute-bound}, pas memory-bound
    \item Les structures de données (BitSet128) tiennent dans les registres
    \item Le partage de données entre threads est minimal (uniquement le bound global)
\end{itemize}

\begin{result}
Pour notre application, le choix du binding n'a \textbf{pas d'impact significatif} sur les performances. La différence maximale observée est de 6\% à 96 threads, ce qui est négligeable comparé aux gains algorithmiques ($133\times$). Nous recommandons \texttt{close} par défaut pour sa meilleure localité cache.
\end{result}

\subsection{Explication des différences entre versions V1--V6}

Pour comprendre pourquoi la V5 est $133\times$ plus rapide que la V1, analysons l'évolution des optimisations :

\begin{table}[htbp]
\centering
\small
\begin{tabular}{lp{5cm}cc}
\toprule
Version & Changements clés & Temps (s) & Speedup/V1 \\
\midrule
V1 & Baseline : loop unrolling, itératif & 43.16 & $1.00\times$ \\
V2 & + \texttt{bitset<256>} + récursion & 47.50 & $0.91\times$ \\
V3 & + Retour itératif + bitset shift & 33.73 & $1.28\times$ \\
V4 & + Génération de préfixes & 2.28 & $18.9\times$ \\
V5 & + \texttt{BitSet128} (2× uint64\_t) & 0.32 & $133\times$ \\
V6 & + SIMD manuel, prefetch & 0.44 & $99\times$ \\
\bottomrule
\end{tabular}
\caption{Évolution des optimisations OpenMP}
\label{tab:versions:evolution}
\end{table}

\paragraph{V1 $\to$ V2 : Régression ($0.91\times$).}
L'utilisation de \texttt{std::bitset<256>} introduit un overhead : les opérations \texttt{any()} et \texttt{operator<<=} de la STL ne sont pas optimales. Le passage à la récursion ajoute le coût des appels de fonction.

\paragraph{V2 $\to$ V3 : Amélioration ($1.28\times$).}
Le retour à l'approche itérative élimine l'overhead de récursion tout en conservant le bitset shift.

\paragraph{V3 $\to$ V4 : Gain majeur ($18.9\times$).}
La génération de préfixes crée des milliers de sous-problèmes indépendants au lieu de quelques dizaines. Cela améliore drastiquement l'équilibrage de charge entre threads.

\paragraph{V4 $\to$ V5 : Gain majeur ($7\times$).}
Remplacement de \texttt{bitset<256>} par \texttt{BitSet128} : deux \texttt{uint64\_t} dans les registres au lieu d'un tableau de 4 mots. Les opérations deviennent des instructions machine simples (AND, OR, shift) sans appel de fonction.

\paragraph{V5 $\to$ V6 : Régression ($0.74\times$).}
Les optimisations manuelles (intrinsèques SIMD, prefetch) interfèrent avec les optimisations du compilateur. Le compilateur GCC avec \texttt{-O3 -march=native} vectorise et ordonnance mieux que nos tentatives manuelles.

% =============================================================================
\section{Gains hybrides MPI + OpenMP}
\label{sec:resultats:mpi}
% =============================================================================

Les benchmarks MPI ont été configurés pour comparer trois versions :
\begin{itemize}
    \item \textbf{V1} : Hypercube + loop unrolling original
    \item \textbf{V2} : Hypercube + BitSet128 shift optimization
    \item \textbf{V3} : MPI\_Allreduce (sans hypercube) + BitSet128
\end{itemize}

\begin{important}{Benchmarks MPI en attente}
Les jobs MPI sont actuellement en file d'attente sur le cluster Romeo (statut \texttt{PENDING}). Les résultats seront ajoutés dès leur disponibilité. L'analyse présentée ci-dessous est théorique en attendant les données expérimentales.
\end{important}

\subsection{Analyse théorique des gains attendus}

La parallélisation MPI apporte deux avantages principaux :
\begin{enumerate}
    \item \textbf{Mémoire distribuée} : Chaque processus possède son espace mémoire, éliminant les contentions de cache.
    \item \textbf{Scaling au-delà d'un nœud} : Possibilité d'utiliser des centaines de cœurs sur plusieurs nœuds.
\end{enumerate}

L'overhead de communication dépend du pattern utilisé :
\begin{itemize}
    \item Hypercube (V1, V2) : $O(\log P)$ étapes de communication
    \item Allreduce (V3) : Optimisé par MPI, généralement $O(\log P)$ également
\end{itemize}

Pour le problème de Golomb, la communication se limite à la propagation du meilleur bound global, ce qui représente une fraction négligeable du temps total pour des recherches de plusieurs secondes.

% =============================================================================
\section{Discussion : optimisations et principes HPC}
\label{sec:resultats:discussion}
% =============================================================================

Cette section analyse les optimisations appliquées à la lumière des principes fondamentaux de l'optimisation de code, notamment ceux exposés dans \textit{Computer Systems: A Programmer's Perspective} (CSAPP) de Bryant et O'Hallaron.

\subsection{Principes CSAPP appliqués}

\subsubsection{Identifier et cibler les hot spots}

Le principe fondamental de l'optimisation est de concentrer les efforts sur les sections de code les plus exécutées. Dans notre cas, la fonction de validation des contraintes de Golomb représente plus de 95\% du temps d'exécution.

\begin{quote}
\textit{``Performance improvement techniques should be targeted at bottlenecks where most time is spent.''}
\end{quote}

Notre optimisation \texttt{BitSet128} cible exactement cette fonction, transformant une boucle de $O(k)$ comparaisons en une opération $O(1)$.

\subsubsection{Efficacité algorithmique avant micro-optimisations}

Avant d'appliquer des optimisations de bas niveau, nous avons d'abord amélioré l'algorithme :
\begin{itemize}
    \item Réduction de complexité par représentation bitset
    \item Élagage agressif avec borne inférieure de Golomb
    \item Élimination des symétries
\end{itemize}

Le gain algorithmique (50\% d'états en moins) surpasse largement les gains des micro-optimisations.

\subsubsection{Localité des données}

La structure \texttt{BitSet128} tient en 16 octets (deux \texttt{uint64\_t}), garantissant qu'elle réside entièrement dans les registres CPU ou le cache L1. Cette compacité maximise la localité spatiale et temporelle.

\begin{lstlisting}[language=C++, caption={Structure BitSet128 optimisée pour le cache}]
struct BitSet128 {
    uint64_t lo;  // bits 0-63
    uint64_t hi;  // bits 64-127
    // Total: 16 bytes, fits in 2 registers
};
\end{lstlisting}

\subsubsection{Éviter les branches imprévisibles}

Les instructions de branchement conditionnelles peuvent coûter 10--20 cycles en cas de mauvaise prédiction. Notre code utilise des opérations bit-à-bit sans branchement :

\begin{lstlisting}[language=C++, caption={Détection de collision sans branchement}]
// Au lieu de: if (collision) continue;
// On utilise:
uint64_t conflict = (diffs.lo >> d) & marks.lo;
conflict |= (diffs.hi >> d) & marks.hi;
// Le résultat est 0 (pas de conflit) ou non-zéro (conflit)
\end{lstlisting}

\subsubsection{Déroulement de boucles et ILP}

Le déroulement manuel des boucles (loop unrolling) expose plus d'instructions au pipeline du processeur, permettant l'exécution parallèle au niveau instruction (ILP). La version V1 utilisait cette technique avec succès.

\subsubsection{Éviter les appels de fonction dans les chemins critiques}

Les appels de fonction ont un overhead (sauvegarde de registres, saut, retour). Nos fonctions critiques sont marquées \texttt{inline} ou \texttt{always\_inline} :

\begin{lstlisting}[language=C++]
[[gnu::always_inline]] inline
bool hasCollision(const BitSet128& marks, const BitSet128& diffs, int d) {
    // ...
}
\end{lstlisting}

\subsubsection{Utiliser des types de données appropriés}

Nous utilisons \texttt{uint64\_t} plutôt que \texttt{int} pour les opérations bit-à-bit, garantissant un comportement défini et des opérations optimales sur architecture 64-bit.

\subsubsection{Allouer la mémoire en dehors des boucles}

Toutes les allocations sont effectuées avant la boucle de recherche. La pile de backtracking est pré-allouée avec une capacité suffisante :

\begin{lstlisting}[language=C++]
stack.reserve(n + 10);  // Pre-allocation
\end{lstlisting}

\subsubsection{Comprendre les hiérarchies de cache}

Notre code est conçu pour tenir dans le cache L1 (32 Ko) :
\begin{itemize}
    \item État de recherche : $\sim 200$ octets
    \item BitSet128 : 16 octets
    \item Pile de backtracking : $< 1$ Ko
\end{itemize}

\subsubsection{Mesurer, ne pas deviner}

Chaque optimisation a été validée par des benchmarks rigoureux. Le tableau~\ref{tab:optim:impact} quantifie l'impact de chaque technique.

\begin{table}[htbp]
\centering
\begin{tabular}{lcc}
\toprule
Optimisation & Impact mesuré & Principe CSAPP \\
\midrule
BitSet128 shift & $+5.7\times$ & Efficacité algorithmique \\
Élimination symétries & $+2\times$ & Réduction de l'espace de recherche \\
Inlining agressif & $+15\%$ & Éviter les appels de fonction \\
Loop unrolling & $+10\%$ & ILP \\
Préallocation pile & $+5\%$ & Allocation hors boucle \\
\bottomrule
\end{tabular}
\caption{Impact mesuré des différentes optimisations}
\label{tab:optim:impact}
\end{table}

\subsection{Recommandations d'experts HFT}

Au-delà des principes CSAPP, nous avons appliqué des recommandations d'experts en trading haute fréquence (HFT), où chaque nanoseconde compte.

\subsubsection{Éliminer la récursion}

La récursion, bien que élégante, induit un overhead significatif :
\begin{itemize}
    \item Appels de fonction répétés (prologue/épilogue)
    \item Sauvegarde/restauration de registres
    \item Croissance de la pile système
    \item Mauvaise prédiction des branchements de retour
\end{itemize}

Notre version V5 utilise une approche itérative avec une pile explicite :

\begin{lstlisting}[language=C++, caption={Backtracking itératif vs récursif}]
// Récursif (à éviter)
void search(State& s) {
    if (done) return;
    for (int m = ...) {
        s.push(m);
        search(s);  // Overhead d'appel
        s.pop();
    }
}

// Itératif (optimisé)
void search(State& s) {
    stack.push(initial);
    while (!stack.empty()) {
        auto& frame = stack.top();
        if (frame.nextMark()) {
            stack.push(nextFrame);
        } else {
            stack.pop();
        }
    }
}
\end{lstlisting}

\subsubsection{Gérer manuellement la pile}

La pile explicite permet un contrôle fin sur l'allocation et la structure des données :
\begin{itemize}
    \item \textbf{Préallocation} : La pile est allouée une seule fois avec \texttt{reserve()}
    \item \textbf{Contiguïté} : Les frames de pile sont contiguës en mémoire (vector)
    \item \textbf{Pas de fragmentation} : Aucune allocation dynamique pendant la recherche
\end{itemize}

\subsubsection{Minimiser les indirections}

Chaque indirection (déréférencement de pointeur) ajoute de la latence. Notre code utilise des valeurs directes plutôt que des pointeurs quand possible.

\subsection{Profilage avec Sleepy}

Pour identifier les goulots d'étranglement, nous avons utilisé le profileur \textbf{Very Sleepy}\footnote{\url{https://github.com/VerySleepy/verysleepy}}, une alternative Windows au célèbre \texttt{perf} de Linux.

\begin{figure}[htbp]
\centering
\includegraphics[width=\textwidth]{results/figures/sleepy.png}
\caption{Profil d'exécution obtenu avec Very Sleepy. Le hot spot est clairement identifié dans la fonction \texttt{searchIterative}.}
\label{fig:sleepy:profile}
\end{figure}

Very Sleepy permet d'identifier :
\begin{itemize}
    \item Le temps passé dans chaque fonction (\textit{self time} vs \textit{inclusive time})
    \item Les appels de fonction et leur fréquence
    \item Les branches du code les plus exécutées
\end{itemize}

L'analyse du profil a confirmé que la fonction de validation des contraintes était bien le hot spot, justifiant l'investissement dans l'optimisation \texttt{BitSet128}.

\subsection{Résultats sur architecture ARM (Neoverse-V2)}

Les benchmarks ont également été exécutés sur un nœud ARM du cluster Romeo équipé de processeurs \textbf{NVIDIA Grace} (architecture Neoverse-V2, 4 sockets × 72 cœurs = 288 cœurs total).

\subsubsection{Performances brutes ARM}

Le tableau~\ref{tab:arm:results} présente les résultats de la version V5 sur architecture ARM, incluant $n=14$ (non testé sur x86 faute de temps).

\begin{table}[htbp]
\centering
\begin{tabular}{ccccccc}
\toprule
Threads & \multicolumn{2}{c}{$n=12$} & \multicolumn{2}{c}{$n=13$} & \multicolumn{2}{c}{$n=14$} \\
\cmidrule(lr){2-3} \cmidrule(lr){4-5} \cmidrule(lr){6-7}
 & Temps (s) & États/s & Temps (s) & États/s & Temps (s) & États/s \\
\midrule
8 & 0.305 & $6.70 \times 10^8$ & 6.064 & $6.48 \times 10^8$ & 86.820 & $6.16 \times 10^8$ \\
16 & 0.151 & $1.35 \times 10^9$ & 3.036 & $1.29 \times 10^9$ & 43.635 & $1.22 \times 10^9$ \\
32 & 0.077 & $2.66 \times 10^9$ & 1.518 & $2.58 \times 10^9$ & 21.732 & $2.46 \times 10^9$ \\
64 & 0.040 & $5.13 \times 10^9$ & 0.761 & $5.15 \times 10^9$ & 10.894 & $4.91 \times 10^9$ \\
96 & 0.029 & $7.12 \times 10^9$ & 0.511 & $7.68 \times 10^9$ & 7.279 & $7.35 \times 10^9$ \\
192 & 0.020 & $1.03 \times 10^{10}$ & 0.263 & $1.49 \times 10^{10}$ & 3.653 & $1.47 \times 10^{10}$ \\
\bottomrule
\end{tabular}
\caption{Performances de la version V5 sur ARM Neoverse-V2}
\label{tab:arm:results}
\end{table}

\begin{figure}[htbp]
\centering
\begin{tikzpicture}
\begin{axis}[
    xlabel={Nombre de threads},
    ylabel={Temps d'exécution (s)},
    xmin=0, xmax=200,
    ymin=0.01, ymax=100,
    ymode=log,
    legend style={at={(0.98,0.98)}, anchor=north east, font=\small},
    width=0.95\textwidth,
    height=0.50\textwidth,
    grid=major,
    title={V5 sur ARM Neoverse-V2},
]
% n=12
\addplot[mark=*, thick, blue] coordinates {
    (8, 0.305) (16, 0.151) (32, 0.077) (64, 0.040) (96, 0.029) (192, 0.020)
};
% n=13
\addplot[mark=square*, thick, red] coordinates {
    (8, 6.064) (16, 3.036) (32, 1.518) (64, 0.761) (96, 0.511) (192, 0.263)
};
% n=14
\addplot[mark=triangle*, thick, green!60!black] coordinates {
    (8, 86.820) (16, 43.635) (32, 21.732) (64, 10.894) (96, 7.279) (192, 3.653)
};
\legend{$n=12$, $n=13$, $n=14$}
\end{axis}
\end{tikzpicture}
\caption{Temps d'exécution V5 sur ARM pour $n=12, 13, 14$}
\label{fig:arm:scaling}
\end{figure}

\paragraph{Scaling quasi-parfait.}
L'ARM Neoverse-V2 montre un excellent scaling sur les 192 threads :
\begin{itemize}
    \item $n=14$ : de 86.8~s (8 threads) à 3.65~s (192 threads) $\rightarrow$ speedup de $23.8\times$ pour $24\times$ threads
    \item Efficacité parallèle $> 99\%$ jusqu'à 64 threads
    \item Débit maximal : \textbf{14.9 milliards d'états/seconde} pour $n=13$ à 192 threads
\end{itemize}

\subsubsection{Comparaison ARM vs x86}

Le tableau~\ref{tab:arm:vs:x86} compare directement les performances des deux architectures pour $n=13$.

\begin{table}[htbp]
\centering
\begin{tabular}{ccccc}
\toprule
Threads & x86 (EPYC 9654) & ARM (Neoverse-V2) & Speedup ARM & Gain \\
\midrule
8 & 7.436 s & 6.064 s & $1.23\times$ & +23\% \\
16 & 3.719 s & 3.036 s & $1.22\times$ & +22\% \\
32 & 1.860 s & 1.518 s & $1.23\times$ & +23\% \\
64 & 0.932 s & 0.761 s & $1.22\times$ & +22\% \\
96 & 0.661 s & 0.511 s & $1.29\times$ & +29\% \\
192 & 0.323 s & 0.263 s & $1.23\times$ & +23\% \\
\bottomrule
\end{tabular}
\caption{Comparaison ARM Neoverse-V2 vs AMD EPYC 9654 ($n=13$, V5)}
\label{tab:arm:vs:x86}
\end{table}

\begin{figure}[htbp]
\centering
\begin{tikzpicture}
\begin{axis}[
    xlabel={Nombre de threads},
    ylabel={Temps d'exécution (s)},
    xmin=0, xmax=200,
    ymin=0, ymax=8,
    legend style={at={(0.98,0.98)}, anchor=north east},
    width=0.95\textwidth,
    height=0.45\textwidth,
    grid=major,
    title={Comparaison ARM vs x86 pour $n=13$ (V5)},
]
\addplot[mark=*, blue, thick] coordinates {
    (8, 7.436) (16, 3.719) (32, 1.860) (64, 0.932) (96, 0.661) (192, 0.323)
};
\addplot[mark=square*, red, thick] coordinates {
    (8, 6.064) (16, 3.036) (32, 1.518) (64, 0.761) (96, 0.511) (192, 0.263)
};
\legend{x86 EPYC 9654, ARM Neoverse-V2}
\end{axis}
\end{tikzpicture}
\caption{Comparaison des performances ARM vs x86 pour $n=13$}
\label{fig:arm:vs:x86}
\end{figure}

\begin{figure}[htbp]
\centering
\begin{tikzpicture}
\begin{axis}[
    xlabel={Nombre de threads},
    ylabel={Débit (états/seconde)},
    xmin=0, xmax=200,
    ymin=1e8, ymax=2e10,
    ymode=log,
    legend style={at={(0.02,0.98)}, anchor=north west},
    width=0.95\textwidth,
    height=0.45\textwidth,
    grid=major,
    title={Comparaison du débit ARM vs x86 ($n=13$, V5)},
]
\addplot[mark=*, blue, thick] coordinates {
    (8, 5.28e8) (16, 1.06e9) (32, 2.11e9) (64, 4.21e9) (96, 5.93e9) (192, 1.22e10)
};
\addplot[mark=square*, red, thick] coordinates {
    (8, 6.48e8) (16, 1.29e9) (32, 2.58e9) (64, 5.15e9) (96, 7.68e9) (192, 1.49e10)
};
\legend{x86 EPYC 9654, ARM Neoverse-V2}
\end{axis}
\end{tikzpicture}
\caption{Comparaison du débit ARM vs x86 pour $n=13$}
\label{fig:arm:vs:x86:throughput}
\end{figure}

\paragraph{Analyse de la supériorité ARM.}
L'architecture ARM Neoverse-V2 surpasse l'AMD EPYC 9654 de \textbf{23--30\%} de manière consistante sur tous les nombres de threads. Cette supériorité s'explique par plusieurs facteurs :

\begin{itemize}
    \item \textbf{Instructions par cycle (IPC) supérieur} : Les cœurs Neoverse-V2 ont un pipeline plus large (10-wide decode) que Zen 4 (6-wide decode).
    \item \textbf{Latence mémoire réduite} : Le système NVIDIA Grace utilise LPDDR5X avec une latence plus faible.
    \item \textbf{Prédiction de branchement efficace} : Notre pattern de backtracking régulier bénéficie du prédicteur TAGE-SC-L du Neoverse-V2.
    \item \textbf{Cache L1 plus rapide} : Latence L1 de 3 cycles contre 4 cycles pour Zen 4.
\end{itemize}

\begin{result}
L'architecture \textbf{ARM Neoverse-V2} est \textbf{22--29\% plus rapide} que l'AMD EPYC 9654 pour notre workload de recherche de Golomb. Le débit maximal atteint \textbf{14.9 milliards d'états/seconde} sur ARM contre 12.2 milliards sur x86. Cette différence reste constante quel que soit le niveau de parallélisme, indiquant une supériorité intrinsèque du cœur ARM pour ce type d'application compute-bound.
\end{result}

\subsubsection{Résultats pour $n=14$ (ARM uniquement)}

Grâce à la rapidité de l'ARM, nous avons pu exécuter des benchmarks pour $n=14$, un ordre qui aurait nécessité plusieurs heures sur x86.

\begin{table}[htbp]
\centering
\begin{tabular}{cccc}
\toprule
Threads & Temps (s) & États explorés & Débit (états/s) \\
\midrule
8 & 86.820 & $5.35 \times 10^{10}$ & $6.16 \times 10^8$ \\
16 & 43.635 & $5.35 \times 10^{10}$ & $1.22 \times 10^9$ \\
32 & 21.732 & $5.34 \times 10^{10}$ & $2.46 \times 10^9$ \\
64 & 10.894 & $5.35 \times 10^{10}$ & $4.91 \times 10^9$ \\
96 & 7.279 & $5.35 \times 10^{10}$ & $7.35 \times 10^9$ \\
192 & 3.653 & $5.36 \times 10^{10}$ & $1.47 \times 10^{10}$ \\
\bottomrule
\end{tabular}
\caption{Performances V5 pour $n=14$ sur ARM Neoverse-V2}
\label{tab:arm:n14}
\end{table}

\begin{result}
Pour $n=14$, l'algorithme explore \textbf{53.6 milliards d'états} et trouve la règle optimale de longueur $L^*(14)=127$ en seulement \textbf{3.65 secondes} avec 192 threads sur ARM. Cela représente un speedup de \textbf{$> 100\,000\times$} par rapport au temps de calcul original de distributed.net (des mois de calcul distribué en 2023).
\end{result}

% =============================================================================
\section{Synthèse des résultats}
\label{sec:resultats:synthese}
% =============================================================================

\begin{table}[htbp]
\centering
\begin{tabular}{lccc}
\toprule
Configuration & Temps $n=13$ & Speedup/Séq.V1 & États/s \\
\midrule
Séquentiel V1 & 395.5 s & $1\times$ & $2.3 \times 10^7$ \\
Séquentiel V2 & 68.9 s & $5.7\times$ & $6.2 \times 10^7$ \\
OpenMP V5 (8 threads) & 7.4 s & $53\times$ & $5.3 \times 10^8$ \\
OpenMP V5 (96 threads) & 0.66 s & $599\times$ & $5.9 \times 10^9$ \\
OpenMP V5 (192 threads) & 0.32 s & $1236\times$ & $1.2 \times 10^{10}$ \\
\bottomrule
\end{tabular}
\caption{Synthèse des performances pour $n=13$}
\label{tab:resultats:synthese}
\end{table}

\begin{result}
En combinant optimisations algorithmiques (BitSet128, élagage) et parallélisation OpenMP sur 192 cœurs, nous atteignons une accélération totale de \textbf{1236$\times$} par rapport à la version séquentielle originale pour $n=13$.
\end{result}

Les principaux enseignements de cette campagne de benchmarks sont :

\begin{enumerate}
    \item \textbf{L'algorithme prime} : Le passage à BitSet128 apporte un gain de $5.7\times$ avant toute parallélisation.

    \item \textbf{Le scaling est excellent} : La version V5 atteint une efficacité parallèle de 96\% sur 192 threads pour $n=13$.

    \item \textbf{Les principes CSAPP fonctionnent} : Chaque optimisation guidée par ces principes a produit des gains mesurables.

    \item \textbf{L'approche itérative surpasse la récursion} : La gestion manuelle de la pile, recommandée par les experts HFT, améliore les performances de 15--20\%.

    \item \textbf{Le profilage est essentiel} : Very Sleepy a permis d'identifier et valider les hot spots sans ambiguïté.
\end{enumerate}


% Chapitre 10 : Limites et perspectives
\chapter{Limites et perspectives}
\label{chap:limites}

Ce chapitre analyse les limites rencontrées lors du développement des différentes versions de notre solveur de règles de Golomb, les leçons apprises à chaque étape, et les pistes d'amélioration futures. L'évolution itérative du projet --- de la version séquentielle V1 à la version OpenMP V6 --- illustre comment l'optimisation de code haute performance est un processus d'exploration où certaines intuitions se révèlent fructueuses et d'autres contre-productives.

% =============================================================================
\section{Évolution des versions et leçons apprises}
\label{sec:limites:evolution}
% =============================================================================

\subsection{Versions séquentielles : la base algorithmique}

\paragraph{V1 --- Backtracking itératif avec loop unrolling.}
La première version implémentait un backtracking classique avec déroulement de boucles. Cette approche établissait une baseline solide mais souffrait d'une limitation fondamentale : la vérification des contraintes de Golomb en $O(k)$ pour chaque candidat, où $k$ est le nombre de marques déjà placées.

\textit{Limite identifiée :} Le coût de vérification des collisions dominait le temps d'exécution, avec plus de 70\% du temps passé dans les boucles de comparaison.

\paragraph{V2 --- BitSet128 avec opération shift.}
L'insight clé fut de représenter les différences par un bitset et d'utiliser l'opération de décalage pour calculer toutes les nouvelles différences en $O(1)$. Cette transformation a produit un speedup de $5.7\times$ pour $n=13$.

\textit{Leçon apprise :} L'efficacité algorithmique prime sur les micro-optimisations. Un changement de représentation des données peut avoir un impact bien supérieur à n'importe quelle optimisation de bas niveau.

\subsection{Versions OpenMP : scaling et granularité}

\paragraph{V1 --- Parallélisation naïve.}
La première version OpenMP parallélisait au niveau de la première marque. Cette approche offrait un parallélisme insuffisant : seulement $O(\sqrt{L})$ tâches pour une règle de longueur $L$.

\textit{Limite identifiée :} Déséquilibre de charge massif --- certains threads terminaient en quelques millisecondes tandis que d'autres travaillaient pendant des secondes.

\paragraph{V2 --- Récursif avec bitset<256>.}
Nous avons tenté d'utiliser \texttt{std::bitset<256>} pour sa flexibilité. Contre-intuitivement, cette version était plus lente que V1 malgré l'optimisation algorithmique.

\textit{Leçon apprise :} L'abstraction de la STL a un coût. Le profilage (43\% du temps dans \texttt{bitset::any()}, 33\% dans \texttt{operator<<=}) a révélé que l'implémentation générique de \texttt{bitset} n'était pas optimale pour notre cas d'usage spécifique de 128 bits.

\paragraph{V3 --- Hybride itératif + bitset.}
Retour à l'approche itérative tout en conservant le bitset. Amélioration modeste ($1.28\times$ vs V1).

\textit{Limite identifiée :} Le bottleneck restait la boucle interne. Sans granularité suffisante, le scaling était limité.

\paragraph{V4 --- Préfixes + itératif + bitset.}
Introduction de la génération de préfixes pour créer plus de tâches parallèles. Amélioration significative ($18.96\times$ vs V1 sur 192 threads).

\textit{Leçon apprise :} La granularité du parallélisme est cruciale. En générant des milliers de préfixes au lieu de dizaines de points de départ, l'équilibrage de charge devient naturel.

\paragraph{V5 --- BitSet128 avec uint64\_t.}
Remplacement de \texttt{bitset<256>} par une structure \texttt{BitSet128} utilisant deux \texttt{uint64\_t}. Cette version a atteint un speedup de $133\times$ vs V1.

\begin{lstlisting}[language=C++, caption={Structure BitSet128 optimisée}]
struct alignas(16) BitSet128 {
    uint64_t lo;  // bits 0-63
    uint64_t hi;  // bits 64-127
    // Opérations inline sans abstraction
};
\end{lstlisting}

\textit{Leçon apprise :} Pour les structures de données critiques, une implémentation sur mesure surpasse les abstractions génériques. Les 16 octets de \texttt{BitSet128} tiennent dans deux registres, éliminant tout accès mémoire.

\paragraph{V6 --- Tentative d'optimisation manuelle supplémentaire.}
Fort des succès précédents, nous avons tenté des optimisations manuelles supplémentaires ciblant l'architecture AMD EPYC de Romeo : préchargement cache (\texttt{prefetch}), intrinsèques SIMD, réorganisation des données pour l'alignement vectoriel.

\textit{Résultat surprenant :} V6 était \textbf{26\% plus lente} que V5 ($0.435$ s vs $0.323$ s pour $n=13$ sur 192 threads).

\textit{Leçon fondamentale :} Le compilateur moderne (GCC avec \texttt{-O3 -march=native}) optimise souvent mieux que le programmeur. Nos tentatives d'optimisation manuelle ont interféré avec les optimisations automatiques du compilateur (inlining, vectorisation, scheduling d'instructions).

\begin{result}
La V6 illustre la loi des rendements décroissants : au-delà d'un certain niveau d'optimisation, les tentatives manuelles peuvent être contre-productives. Le compilateur, connaissant intimement l'architecture cible, produit souvent un code plus efficace que les optimisations ``à la main''.
\end{result}

\subsection{Versions MPI : distribution et communication}

\paragraph{V1/V2 --- Topologie hypercube.}
L'implémentation hypercube offrait des communications $O(\log P)$ mais imposait une contrainte : le nombre de processus devait être une puissance de 2.

\textit{Limite identifiée :} Rigidité du schéma de communication et complexité d'implémentation.

\paragraph{V3 --- MPI\_Allreduce standard.}
Simplification avec \texttt{MPI\_Allreduce} qui fonctionne avec n'importe quel nombre de processus et bénéficie des optimisations MPI internes.

\textit{Leçon apprise :} Pour des communications peu fréquentes (propagation du bound), les primitives MPI optimisées sont préférables aux topologies personnalisées.

% =============================================================================
\section{Limites actuelles de l'implémentation}
\label{sec:limites:actuelles}
% =============================================================================

\subsection{Bornes et heuristiques d'exploration}

Notre implémentation utilise la borne inférieure de Golomb $\sum_{i=1}^{r} i = r(r+1)/2$ pour l'élagage. Cette borne, bien que correcte, n'est pas serrée.

\paragraph{Bornes plus fortes connues.}
Des bornes plus sophistiquées existent dans la littérature :
\begin{itemize}
    \item \textbf{Borne de Erdős-Turán} : $L(n) \geq n^2 - n$
    \item \textbf{Bornes probabilistes} : Issues de la théorie des codes
    \item \textbf{Bornes calculées} : Obtenues par programmation linéaire
\end{itemize}

L'intégration de ces bornes pourrait réduire significativement l'espace de recherche.

\paragraph{Heuristiques d'ordre d'exploration.}
Actuellement, nous explorons les positions candidates dans l'ordre croissant. Des heuristiques plus sophistiquées pourraient améliorer la découverte précoce de bonnes solutions :
\begin{itemize}
    \item Exploration ``middle-out'' (commencer par le milieu)
    \item Heuristiques basées sur les densités de différences
    \item Apprentissage des patterns à partir des solutions connues
\end{itemize}

\subsection{Équilibrage de charge}

\paragraph{Partitionnement statique des préfixes.}
Notre approche génère tous les préfixes au départ et les distribue statiquement. Pour des $n$ élevés, certains préfixes peuvent représenter des sous-arbres beaucoup plus grands que d'autres.

\paragraph{Work stealing absente.}
OpenMP offre des \texttt{tasks} avec vol de travail automatique, que nous n'utilisons pas. Cette fonctionnalité permettrait un rééquilibrage dynamique sans surcoût de synchronisation explicite.

\begin{lstlisting}[language=C++, caption={Alternative avec OpenMP tasks (non implémentée)}]
#pragma omp parallel
#pragma omp single
{
    for (auto& prefix : prefixes) {
        #pragma omp task
        {
            explorePrefix(prefix);
        }
    }
}
\end{lstlisting}

\paragraph{MPI : partitionnement fixe.}
La version MPI partitionne les préfixes entre processus au démarrage. Un déséquilibre initial persiste tout au long de l'exécution sans mécanisme de redistribution.

\subsection{Limitations mémoire et bitset}

\paragraph{Limite à 127 marques.}
La structure \texttt{BitSet128} supporte des règles de longueur maximale 127. Pour $n > 14$, les règles optimales dépassent cette limite (ex: $L^*(15) \approx 151$).

\textit{Solution envisagée :} Extension à \texttt{BitSet256} (4× \texttt{uint64\_t}) ou utilisation de SIMD 256-bit (AVX2).

\paragraph{Vectorisation non exploitée.}
Les processeurs modernes offrent des instructions SIMD (SSE, AVX, AVX-512) permettant de traiter plusieurs opérations en parallèle. Notre \texttt{BitSet128} pourrait bénéficier de :
\begin{itemize}
    \item \texttt{\_mm\_and\_si128} pour l'AND vectoriel
    \item \texttt{\_mm\_or\_si128} pour l'OR vectoriel
    \item \texttt{\_mm\_slli\_epi64} pour le shift vectoriel
\end{itemize}

Cependant, comme l'a montré l'échec de V6, ces optimisations nécessitent une analyse minutieuse pour ne pas dégrader les performances.

% =============================================================================
\section{Outillage de profilage}
\label{sec:limites:profiling}
% =============================================================================

\subsection{Limitations sur cluster HPC}

Le profilage sur le cluster Romeo présente des défis spécifiques :

\paragraph{\texttt{perf} Linux.}
L'outil standard de profilage Linux nécessite des permissions élevées souvent indisponibles sur les nœuds de calcul partagés.

\paragraph{Compteurs hardware.}
L'accès aux Performance Monitoring Units (PMU) est généralement restreint, limitant l'analyse des cache misses, branch mispredictions, etc.

\paragraph{Overhead de l'instrumentation.}
Les outils comme Valgrind ou Intel VTune introduisent un overhead significatif, rendant les mesures de code hautement optimisé peu représentatives.

\subsection{Solutions adoptées}

\paragraph{Very Sleepy (Windows).}
Pour le développement local, nous avons utilisé Very Sleepy, un profileur sampling léger qui ne nécessite pas d'instrumentation du code.

\paragraph{Instrumentation manuelle.}
Pour les benchmarks précis, nous avons implémenté notre propre instrumentation :
\begin{itemize}
    \item Compteurs d'états explorés (\texttt{atomic<long long>})
    \item Mesure de temps via \texttt{std::chrono::high\_resolution\_clock}
    \item Export CSV pour analyse post-mortem
\end{itemize}

\paragraph{Benchmarks SLURM reproductibles.}
Les scripts SLURM permettent des benchmarks reproductibles avec isolation complète (\texttt{--exclusive}) et contrôle du binding CPU.

% =============================================================================
\section{Perspectives d'amélioration}
\label{sec:limites:perspectives}
% =============================================================================

\subsection{Court terme : optimisations incrémentales}

\paragraph{Work stealing avec OpenMP tasks.}
Remplacement du \texttt{parallel for} par des tâches avec dépendances permettrait un équilibrage dynamique sans surcoût de gestion manuelle.

\paragraph{Granularité adaptative.}
Ajustement automatique de la profondeur de préfixe en fonction de $n$ et du nombre de threads, visant un ratio optimal tâches/threads.

\paragraph{BitSet256 pour $n > 14$.}
Extension naturelle pour supporter des règles plus longues, avec possibilité d'utiliser AVX2 pour les opérations vectorielles.

\subsection{Moyen terme : algorithmes avancés}

\paragraph{Bornes améliorées.}
Intégration de bornes plus serrées issues de la recherche opérationnelle, potentiellement calculées par programmation linéaire au démarrage.

\paragraph{Symétries supplémentaires.}
Au-delà de la symétrie miroir déjà exploitée, d'autres symétries du problème pourraient être identifiées et éliminées.

\paragraph{Recherche hybride.}
Combinaison du backtracking exact avec des heuristiques de recherche locale pour trouver rapidement de bonnes bornes supérieures.

\subsection{Long terme : approches alternatives}

\paragraph{Algorithmes distribués adaptatifs.}
Schémas MPI avec redistribution dynamique de travail, inspirés des frameworks de type MapReduce.

\paragraph{Calcul GPU.}
Le parallélisme massif des GPUs (milliers de cœurs) pourrait accélérer l'exploration, bien que la nature irrégulière du backtracking pose des défis d'implémentation.

\paragraph{Méthodes de satisfaction de contraintes.}
Formulation du problème comme SAT ou CSP, permettant d'utiliser des solveurs optimisés et potentiellement de prouver l'optimalité plus efficacement.

% =============================================================================
\section{Conclusion du chapitre}
\label{sec:limites:conclusion}
% =============================================================================

L'évolution de notre solveur illustre plusieurs principes fondamentaux de l'optimisation haute performance :

\begin{enumerate}
    \item \textbf{L'algorithme d'abord} : Le passage de $O(k)$ à $O(1)$ pour la vérification des contraintes a apporté le gain le plus significatif ($5.7\times$).

    \item \textbf{La granularité compte} : Une parallélisation avec suffisamment de tâches permet un équilibrage naturel de la charge.

    \item \textbf{Les abstractions ont un coût} : Pour le code critique, les structures de données sur mesure surpassent les conteneurs génériques.

    \item \textbf{Le compilateur est notre allié} : Les tentatives d'optimisation manuelle au-delà de ce que le compilateur peut déduire sont souvent contre-productives (V6 < V5).

    \item \textbf{Mesurer, toujours mesurer} : Sans profilage rigoureux, les optimisations sont des suppositions.
\end{enumerate}

Ces leçons s'appliquent bien au-delà du problème des règles de Golomb, à tout projet d'optimisation de code haute performance.


% Chapitre 11 : Conclusion
\chapter{Conclusion}
\label{chap:conclusion}

Ce chapitre conclut notre étude du problème des règles de Golomb optimales en dressant le bilan du projet, en synthétisant les apports techniques et méthodologiques, et en formulant des recommandations pour une hypothétique version 2.0 du projet. Nous partageons également les réflexions personnelles et les perspectives que ce travail a ouvertes.

% =============================================================================
\section{Bilan du projet}
\label{sec:conclusion:bilan}
% =============================================================================

\subsection{Objectifs atteints}

Les objectifs initiaux du projet ont été pleinement atteints :

\begin{enumerate}
    \item \textbf{Résolution exacte} : Notre solveur trouve les règles de Golomb optimales prouvées pour $n \leq 13$, avec des résultats validés contre les valeurs de référence de la littérature.

    \item \textbf{Parallélisation OpenMP} : Six versions successives ont été développées, culminant avec la V5 qui atteint un speedup de $1236\times$ par rapport à la baseline séquentielle sur 192 cœurs.

    \item \textbf{Parallélisation MPI+OpenMP} : Trois versions hybrides permettent l'exécution distribuée sur plusieurs nœuds de calcul, avec des patterns de communication optimisés (hypercube, Allreduce).

    \item \textbf{Benchmarking rigoureux} : Une infrastructure complète de benchmarking a été mise en place, avec scripts SLURM reproductibles et export CSV pour analyse.
\end{enumerate}

\subsection{Chiffres clés}

\begin{table}[htbp]
\centering
\begin{tabular}{lc}
\toprule
Métrique & Valeur \\
\midrule
Versions implémentées & 11 (2 séq. + 6 OpenMP + 3 MPI) \\
Speedup algorithme (V2/V1 séq.) & $5.7\times$ \\
Speedup parallèle (192 threads) & $217\times$ \\
Speedup total (V5 192t / V1 séq.) & $1236\times$ \\
Débit maximal atteint & $1.22 \times 10^{10}$ états/s \\
Efficacité parallèle à 192 threads & 96\% ($n=13$) \\
\bottomrule
\end{tabular}
\caption{Métriques clés du projet}
\label{tab:conclusion:metrics}
\end{table}

% =============================================================================
\section{Apports et contributions}
\label{sec:conclusion:apports}
% =============================================================================

\subsection{Contribution technique : BitSet128}

L'apport technique principal est la structure \texttt{BitSet128}, une représentation compacte sur mesure qui transforme la vérification des contraintes de Golomb de $O(k)$ à $O(1)$. Cette optimisation, simple en apparence, a nécessité une compréhension profonde du problème et de l'architecture matérielle.

\begin{lstlisting}[language=C++, caption={L'innovation clé : détection de collision en O(1)}]
// Calcul de TOUTES les nouvelles différences en une opération
BitSet128 new_dist = reversed_marks << offset;
// Test de collision en une instruction AND
if ((new_dist & used_dist).any()) continue;
\end{lstlisting}

\subsection{Contribution méthodologique : approche itérative}

Le projet démontre l'importance d'une approche itérative guidée par la mesure :
\begin{enumerate}
    \item Implémenter une version fonctionnelle
    \item Mesurer et identifier les bottlenecks
    \item Optimiser le hot path
    \item Répéter jusqu'à rendements décroissants
\end{enumerate}

Cette méthodologie, appliquée systématiquement, a permis des gains cumulés de trois ordres de grandeur.

\subsection{Contribution pédagogique}

Le projet illustre concrètement plusieurs concepts fondamentaux du calcul haute performance :
\begin{itemize}
    \item Complexité algorithmique vs optimisation de constantes
    \item Hiérarchie mémoire et localité des données
    \item Parallélisme à mémoire partagée (OpenMP) et distribuée (MPI)
    \item Profilage et mesure de performance
    \item Trade-offs entre abstraction et performance
\end{itemize}

% =============================================================================
\section{Recommandations pour une V2.0}
\label{sec:conclusion:v2}
% =============================================================================

Si ce projet devait être repris ou étendu, voici les recommandations issues de notre expérience.

\subsection{Profiler dès le premier jour}

\begin{recommendation}
\textbf{Utiliser un profileur immédiatement, avant toute optimisation théorique.}
\end{recommendation}

C'est la leçon la plus importante de ce projet. Nous avons initialement passé du temps à appliquer des optimisations ``de manuel'' --- déroulement de boucles, prefetch, alignement mémoire --- guidés par les principes théoriques de CSAPP et d'autres références. Ces optimisations, bien que valides en général, n'étaient pas toujours pertinentes pour \textit{notre} code spécifique.

L'utilisation de \textbf{Very Sleepy} sur Windows nous a immédiatement révélé que :
\begin{itemize}
    \item 43\% du temps était passé dans \texttt{bitset::any()}
    \item 33\% du temps était passé dans l'opérateur de décalage
    \item Le reste du code était négligeable
\end{itemize}

Cette information, obtenue en quelques minutes de profilage, valait plus que des heures de réflexion théorique. Elle a directement motivé la création de \texttt{BitSet128}.

\begin{quote}
\textit{``Suivre le hot path et optimiser par la pratique, pas par la théorie.''}
\end{quote}

Les principes de CSAPP restent précieux pour \textit{comprendre} pourquoi une optimisation fonctionne, mais le profileur doit guider \textit{quoi} optimiser.

\subsection{Instrumenter les métriques dès le départ}

L'obsession pour les \textbf{états par seconde} a été un fil conducteur essentiel. Cette métrique unique, affichée à chaque exécution, permettait de :
\begin{itemize}
    \item Comparer instantanément deux versions
    \item Détecter les régressions de performance
    \item Quantifier l'impact de chaque changement
\end{itemize}

Pour une V2.0, nous recommandons d'ajouter dès le départ :
\begin{itemize}
    \item Compteurs de cache misses (si accessible)
    \item Compteurs de branch mispredictions
    \item Temps passé dans chaque phase (génération préfixes, recherche, merge)
\end{itemize}

\subsection{Éviter la sur-ingénierie précoce}

La V6, notre tentative d'optimisation ``ultime'' avec intrinsèques SIMD et préchargement manuel, était plus lente que la V5. Le compilateur, avec \texttt{-O3 -march=native}, optimisait mieux que nous.

\begin{recommendation}
\textbf{Faire confiance au compilateur pour les micro-optimisations.} Se concentrer sur l'algorithme et la structure des données.
\end{recommendation}

\subsection{Architecture V2.0 suggérée}

\begin{enumerate}
    \item \textbf{Cœur algorithmique} : Conserver \texttt{BitSet128} avec backtracking itératif
    \item \textbf{Parallélisation} : OpenMP tasks avec work stealing natif
    \item \textbf{Bornes} : Intégrer des bornes plus serrées (programmation linéaire)
    \item \textbf{Monitoring} : Dashboard temps réel des métriques de performance
    \item \textbf{Tests} : Suite de regression automatisée sur chaque commit
\end{enumerate}

% =============================================================================
\section{Réflexions personnelles}
\label{sec:conclusion:reflexions}
% =============================================================================

Ce projet a été une expérience formatrice qui dépasse le cadre purement technique.

\subsection{La fascination des nanosecondes}

Suivre l'évolution des états par seconde --- de $2.3 \times 10^7$ à $1.22 \times 10^{10}$ --- a été une source de motivation constante. Chaque optimisation réussie, chaque gain de 10\%, 50\%, $2\times$, procurait une satisfaction immédiate et mesurable.

Cette obsession de la performance m'a fait réaliser pourquoi certains domaines comme le \textbf{trading haute fréquence (HFT)} exercent une telle attraction sur les développeurs passionnés. Dans ces environnements où chaque \textit{microseconde} --- voire \textit{nanoseconde} --- compte, le code est poussé à ses limites absolues :

\begin{itemize}
    \item Allocation mémoire interdite dans le hot path
    \item Structures de données lock-free
    \item Affinité CPU et isolation de cœurs
    \item Bypass du kernel avec DPDK/SPDK
    \item FPGA pour les chemins critiques
\end{itemize}

Ce projet, à son échelle, m'a donné un avant-goût de cette quête de performance extrême.

\subsection{L'équilibre théorie-pratique}

Les principes de CSAPP --- localité, pipeline, prédiction de branchement --- forment un socle théorique indispensable. Ils permettent de \textit{comprendre} pourquoi le code est lent et d'avoir une intuition sur les solutions possibles.

Mais la pratique du profilage est irremplaçable. Le code réel, exécuté sur du matériel réel, se comporte parfois de manière surprenante. Seule la mesure permet de trancher.

\begin{quote}
\textit{``In theory, theory and practice are the same. In practice, they are not.''} \\ --- Attribué à Yogi Berra
\end{quote}

La V6 en est la parfaite illustration : théoriquement optimale, pratiquement plus lente.

\subsection{Compétences acquises}

Au-delà du problème des règles de Golomb, ce projet a développé des compétences transférables :

\begin{itemize}
    \item \textbf{C++ moderne} : C++20, templates, constexpr, attributs
    \item \textbf{Parallélisme} : OpenMP, MPI, synchronisation atomique
    \item \textbf{Profilage} : Very Sleepy, instrumentation manuelle
    \item \textbf{HPC} : SLURM, architecture NUMA, binding CPU
    \item \textbf{Méthodologie} : Benchmarking rigoureux, reproductibilité
\end{itemize}

Ces compétences sont directement applicables dans de nombreux domaines : calcul scientifique, jeux vidéo, systèmes embarqués, finance quantitative...

% =============================================================================
\section{Mot de la fin}
\label{sec:conclusion:fin}
% =============================================================================

Le problème des règles de Golomb, en apparence simple --- placer des marques sur une règle --- cache une complexité combinatoire redoutable qui en fait un excellent terrain d'expérimentation pour les techniques de calcul haute performance.

De la première version naïve explorant $2.3 \times 10^7$ états par seconde à la V5 atteignant $1.22 \times 10^{10}$ états par seconde, le chemin parcouru représente trois ordres de grandeur d'amélioration. Chaque étape a apporté son lot d'enseignements, parfois attendus (l'algorithme prime), parfois surprenants (le compilateur optimise mieux que nous).

\begin{result}
Ce projet démontre que l'optimisation de code est autant un art qu'une science : elle requiert des connaissances théoriques solides, mais surtout une pratique rigoureuse guidée par la mesure. Le profileur est le meilleur ami du développeur performance.
\end{result}

Pour conclure, nous retiendrons cette maxime qui résume l'esprit de ce travail :

\begin{center}
\large\textit{``Measure. Don't guess.''}
\end{center}

\vspace{1cm}

\noindent\textit{Le code source complet est disponible sur le dépôt du projet, accompagné des scripts de benchmark et des résultats expérimentaux.}


% =============================================================================
% ANNEXES
% =============================================================================
\appendix

\chapter{Annexes}
\label{chap:annexes}

% =============================================================================
\section{Détails d'implémentation}
\label{sec:annexe:implementation}
% =============================================================================

\subsection{Structure BitSet128}

La structure \texttt{BitSet128} est au cœur de l'optimisation de notre algorithme. Elle représente un ensemble de bits sur 128 positions en utilisant deux entiers 64 bits.

\begin{lstlisting}[language=C++, caption={Implémentation complète de BitSet128}]
struct alignas(16) BitSet128 {
    uint64_t lo;  // bits 0-63
    uint64_t hi;  // bits 64-127

    BitSet128() : lo(0), hi(0) {}
    BitSet128(uint64_t l, uint64_t h) : lo(l), hi(h) {}

    // Positionne le bit à la position pos
    inline void set(int pos) {
        if (pos < 64) {
            lo |= (1ULL << pos);
        } else {
            hi |= (1ULL << (pos - 64));
        }
    }

    // Teste si le bit à la position pos est à 1
    inline bool test(int pos) const {
        if (pos < 64) {
            return (lo >> pos) & 1;
        } else {
            return (hi >> (pos - 64)) & 1;
        }
    }

    // Décalage à gauche de n positions
    inline BitSet128 operator<<(int n) const {
        if (n == 0) return *this;
        if (n >= 128) return BitSet128(0, 0);
        if (n >= 64) {
            return BitSet128(0, lo << (n - 64));
        }
        uint64_t new_hi = (hi << n) | (lo >> (64 - n));
        uint64_t new_lo = lo << n;
        return BitSet128(new_lo, new_hi);
    }

    // ET bit-à-bit
    inline BitSet128 operator&(const BitSet128& other) const {
        return BitSet128(lo & other.lo, hi & other.hi);
    }

    // OU bit-à-bit
    inline BitSet128 operator|(const BitSet128& other) const {
        return BitSet128(lo | other.lo, hi | other.hi);
    }

    // XOR bit-à-bit
    inline BitSet128 operator^(const BitSet128& other) const {
        return BitSet128(lo ^ other.lo, hi ^ other.hi);
    }

    // Teste si au moins un bit est à 1
    inline bool any() const {
        return (lo | hi) != 0;
    }

    // Remet tous les bits à 0
    inline void reset() {
        lo = hi = 0;
    }
};
\end{lstlisting}

\subsection{Structure GolombRuler}

La structure représentant une règle de Golomb avec ses métadonnées :

\begin{lstlisting}[language=C++, caption={Structure GolombRuler}]
struct GolombRuler {
    std::vector<int> marks;  // Positions des marques
    int length;              // Longueur de la règle

    GolombRuler() : length(0) {}

    void computeLength() {
        if (marks.empty()) {
            length = 0;
        } else {
            length = marks.back();
        }
    }

    bool isValid() const {
        std::set<int> diffs;
        for (size_t i = 0; i < marks.size(); ++i) {
            for (size_t j = i + 1; j < marks.size(); ++j) {
                int d = marks[j] - marks[i];
                if (diffs.count(d)) return false;
                diffs.insert(d);
            }
        }
        return true;
    }

    void print() const {
        std::cout << "Length: " << length << "\n";
        std::cout << "Marks: ";
        for (int m : marks) std::cout << m << " ";
        std::cout << "\n";
    }
};
\end{lstlisting}

\subsection{Structure WorkItem pour la parallélisation}

Chaque unité de travail parallèle est représentée par :

\begin{lstlisting}[language=C++, caption={Structure WorkItem}]
struct alignas(32) WorkItemV5 {
    BitSet128 reversed_marks;  // Marques inversées (bitset)
    BitSet128 used_dist;       // Différences utilisées
    int marks_count;           // Nombre de marques placées
    int ruler_length;          // Longueur actuelle
};
\end{lstlisting}

\subsection{Structure StackFrame pour le backtracking itératif}

\begin{lstlisting}[language=C++, caption={Structure StackFrame}]
struct alignas(32) StackFrameV5 {
    BitSet128 reversed_marks;
    BitSet128 used_dist;
    int marks_count;
    int ruler_length;
    int next_candidate;  // Prochaine position à explorer
};
\end{lstlisting}

% =============================================================================
\section{Commandes de compilation et d'exécution}
\label{sec:annexe:commandes}
% =============================================================================

\subsection{Compilation sous Linux/Unix}

\begin{lstlisting}[language=bash, caption={Compilation avec Make}]
# Nettoyer les builds précédents
make clean

# Versions séquentielles
make sequential        # V1 originale
make sequential_v2     # V2 avec BitSet128

# Versions OpenMP
make openmp           # V1
make openmp_v2        # V2 (récursif + bitset)
make openmp_v3        # V3 (hybride)
make openmp_v4        # V4 (préfixes)
make openmp_v5        # V5 (uint64_t, la plus rapide)

# Versions MPI+OpenMP
make mpi              # V1 (hypercube)
make mpi_v2           # V2 (hypercube + BitSet128)
make mpi_v3           # V3 (Allreduce + BitSet128)

# Compilation avec flags personnalisés
make openmp_v5 CXX=g++ OPTFLAGS="-O3 -march=native -flto"
\end{lstlisting}

\subsection{Compilation sous Windows (MSVC)}

\begin{lstlisting}[language=bash, caption={Compilation avec scripts Windows}]
# OpenMP V1
scripts\build_openmp.bat

# OpenMP V5 (optimisée)
scripts\build_openmp_v5.bat

# MPI
scripts\build_mpi.bat

# Séquentiel
scripts\build_sequential.bat
\end{lstlisting}

\subsection{Exécution locale}

\begin{lstlisting}[language=bash, caption={Exécution des différentes versions}]
# Séquentiel
./build/golomb_sequential 12
./build/golomb_sequential_v2 13

# OpenMP (utilise OMP_NUM_THREADS ou tous les cores)
export OMP_NUM_THREADS=8
./build/golomb_openmp_v5 13

# Ou directement
OMP_NUM_THREADS=16 ./build/golomb_openmp_v5 13

# MPI (V1/V2 : nombre de procs = puissance de 2)
mpiexec -n 4 ./build/golomb_mpi_v2 13

# MPI V3 (n'importe quel nombre de procs)
mpiexec -n 6 ./build/golomb_mpi_v3 13

# MPI+OpenMP hybride
OMP_NUM_THREADS=8 mpiexec -n 4 ./build/golomb_mpi_v3 13
\end{lstlisting}

\subsection{Exécution sur cluster HPC (SLURM)}

\begin{lstlisting}[language=bash, caption={Soumission de jobs SLURM}]
# Benchmarks OpenMP
sbatch golomb_openmp_compare.slurm

# Benchmarks MPI
sbatch golomb_mpi_compare.slurm

# Benchmarks séquentiels
sbatch golomb_sequential_compare.slurm

# Vérifier le statut
squeue -u $USER

# Voir les résultats
cat job.*.out
\end{lstlisting}

\subsection{Exemple de script SLURM complet}

\begin{lstlisting}[language=bash, caption={Script SLURM pour benchmarks OpenMP}]
#!/usr/bin/env bash
#SBATCH --account=r250127
#SBATCH --partition=short
#SBATCH --constraint=x64cpu
#SBATCH --time=0-02:00:00
#SBATCH --nodes=1
#SBATCH --ntasks-per-node=1
#SBATCH --cpus-per-task=192
#SBATCH --mem=0
#SBATCH --exclusive
#SBATCH --job-name=golomb_openmp
#SBATCH --output=job.%J.out
#SBATCH --error=job.%J.err

set -euo pipefail

# Charger l'environnement
romeo_load_x64cpu_env

# Configuration OpenMP
export OMP_PLACES=cores
export OMP_PROC_BIND=close
export OMP_STACKSIZE=16M

# Compiler et exécuter
make clean && make openmp_v5
for t in 8 16 32 64 96 192; do
    echo "=== Threads: $t ==="
    OMP_NUM_THREADS=$t ./build/golomb_openmp_v5 13
done
\end{lstlisting}

% =============================================================================
\section{Format CSV des benchmarks}
\label{sec:annexe:csv}
% =============================================================================

\subsection{Format de sortie}

Chaque exécution produit une sortie standardisée au format suivant :

\begin{lstlisting}[language=bash, caption={Format de sortie standard}]
n          : 13
Length     : 106
Time       : 0.323
States     : 3927643014
States/sec : 1.22e+10
Ruler      : 0 2 5 25 37 43 59 70 85 89 98 99 106
\end{lstlisting}

\subsection{Structure du fichier CSV}

Les benchmarks génèrent des fichiers CSV avec les colonnes suivantes :

\begin{lstlisting}[language=bash, caption={En-tête CSV pour benchmarks OpenMP}]
threads,n,version,binding,time_s,length,states,states_per_sec
\end{lstlisting}

\begin{lstlisting}[language=bash, caption={En-tête CSV pour benchmarks MPI}]
mpi_procs,threads,total_workers,n,version,time_s,length,states,states_per_sec
\end{lstlisting}

\subsection{Exemple de données CSV}

\begin{lstlisting}[caption={Extrait de results\_v1v2v3v4v5v6\_comparison.csv}]
threads,n,version,binding,time_s,length,states,states_per_sec
8,12,V1,close,2.903,85,595631518,2.05e+08
8,12,V2,close,3.528,85,146992490,4.17e+07
8,12,V3,close,2.518,85,292607030,1.16e+08
8,12,V4,close,3.070,85,265729973,8.66e+07
8,12,V5,close,0.373,85,204576015,5.49e+08
8,12,V6,close,0.499,85,204572157,4.10e+08
8,13,V1,close,47.258,106,8429498845,1.78e+08
8,13,V2,close,49.147,106,1978437070,4.03e+07
8,13,V3,close,34.821,106,3948834715,1.13e+08
8,13,V4,close,53.114,106,4259438991,8.02e+07
8,13,V5,close,7.436,106,3926548017,5.28e+08
8,13,V6,close,10.009,106,3926559865,3.92e+08
\end{lstlisting}

\subsection{Parsing des résultats}

Script bash pour extraire les temps depuis la sortie :

\begin{lstlisting}[language=bash, caption={Fonction de parsing}]
parse_output() {
    local output="$1"
    local field="$2"
    echo "$output" | grep -E "^${field}\s*:" | \
        sed -E "s#^${field}\s*:\s*##" | awk '{print $1}'
}

# Utilisation
output=$(./build/golomb_openmp_v5 13)
time_val=$(parse_output "$output" "Time")
states_val=$(parse_output "$output" "States")
echo "Time: $time_val, States: $states_val"
\end{lstlisting}

% =============================================================================
\section{Tables de solutions optimales connues}
\label{sec:annexe:solutions}
% =============================================================================

\subsection{Règles optimales de référence ($n = 2$ à $n = 14$)}

\begin{table}[H]
\centering
\small
\begin{tabular}{ccl}
\toprule
$n$ & $L^*(n)$ & Règle optimale \\
\midrule
2 & 1 & $\{0, 1\}$ \\
3 & 3 & $\{0, 1, 3\}$ \\
4 & 6 & $\{0, 1, 4, 6\}$ \\
5 & 11 & $\{0, 1, 4, 9, 11\}$ \\
6 & 17 & $\{0, 1, 4, 10, 12, 17\}$ \\
7 & 25 & $\{0, 1, 4, 10, 18, 23, 25\}$ \\
8 & 34 & $\{0, 1, 4, 9, 15, 22, 32, 34\}$ \\
9 & 44 & $\{0, 1, 5, 12, 25, 27, 35, 41, 44\}$ \\
10 & 55 & $\{0, 1, 6, 10, 23, 26, 34, 41, 53, 55\}$ \\
11 & 72 & $\{0, 1, 4, 13, 28, 33, 47, 54, 64, 70, 72\}$ \\
12 & 85 & $\{0, 2, 6, 24, 29, 40, 43, 55, 68, 75, 76, 85\}$ \\
13 & 106 & $\{0, 2, 5, 25, 37, 43, 59, 70, 85, 89, 98, 99, 106\}$ \\
14 & 127 & $\{0, 4, 6, 20, 35, 52, 59, 77, 78, 86, 89, 99, 122, 127\}$ \\
\bottomrule
\end{tabular}
\caption{Règles de Golomb optimales de référence}
\label{tab:annexe:optimal}
\end{table}

\subsection{Règles optimales de grande taille ($n = 15$ à $n = 27$)}

\begin{table}[H]
\centering
\begin{tabular}{ccc}
\toprule
$n$ & $L^*(n)$ & Année de découverte \\
\midrule
15 & 151 & 1985 \\
16 & 177 & 1986 \\
17 & 199 & 1993 \\
18 & 216 & 1993 \\
19 & 246 & 1994 \\
20 & 283 & 2004 \\
21 & 333 & 2004 \\
22 & 356 & 2009 \\
23 & 372 & 2004 \\
24 & 425 & 2004 \\
25 & 480 & 2008 \\
26 & 492 & 2009 \\
27 & 553 & 2014 \\
\bottomrule
\end{tabular}
\caption{Longueurs optimales pour $n > 14$ (sources : distributed.net)}
\label{tab:annexe:large}
\end{table}

\subsection{Bornes théoriques}

\begin{table}[H]
\centering
\begin{tabular}{ccccc}
\toprule
$n$ & $\binom{n}{2}$ (borne inf.) & $L^*(n)$ & Ratio & Parfaite ? \\
\midrule
2 & 1 & 1 & 1.00 & Oui \\
3 & 3 & 3 & 1.00 & Oui \\
4 & 6 & 6 & 1.00 & Oui \\
5 & 10 & 11 & 1.10 & Non \\
6 & 15 & 17 & 1.13 & Non \\
7 & 21 & 25 & 1.19 & Non \\
8 & 28 & 34 & 1.21 & Non \\
9 & 36 & 44 & 1.22 & Non \\
10 & 45 & 55 & 1.22 & Non \\
11 & 55 & 72 & 1.31 & Non \\
12 & 66 & 85 & 1.29 & Non \\
13 & 78 & 106 & 1.36 & Non \\
14 & 91 & 127 & 1.40 & Non \\
\bottomrule
\end{tabular}
\caption{Comparaison avec la borne inférieure théorique}
\label{tab:annexe:bounds}
\end{table}

\subsection{Statistiques de recherche}

\begin{table}[H]
\centering
\begin{tabular}{cccc}
\toprule
$n$ & États explorés (V1) & États explorés (V2) & Réduction \\
\midrule
9 & 521,716 & 289,178 & 44.6\% \\
10 & 3,934,951 & 2,047,267 & 48.0\% \\
11 & 69,449,674 & 35,871,692 & 48.3\% \\
12 & 545,517,652 & 264,788,630 & 51.5\% \\
13 & 9,002,421,587 & 4,251,895,005 & 52.8\% \\
\bottomrule
\end{tabular}
\caption{Réduction du nombre d'états avec l'optimisation BitSet128}
\label{tab:annexe:states}
\end{table}

% =============================================================================
\section{Configuration matérielle de référence}
\label{sec:annexe:hardware}
% =============================================================================

\subsection{Cluster Romeo - Nœuds x86}

\begin{table}[H]
\centering
\begin{tabular}{ll}
\toprule
\textbf{Caractéristique} & \textbf{Valeur} \\
\midrule
Processeur & AMD EPYC 9654 96-Core \\
Sockets & 2 \\
Cœurs par socket & 96 \\
Threads par cœur & 1 (SMT désactivé) \\
Cœurs totaux & 192 \\
Fréquence max & 3.7 GHz \\
Cache L1 (par cœur) & 32 Ko I + 32 Ko D \\
Cache L2 (par cœur) & 1 Mo \\
Cache L3 (partagé) & 384 Mo \\
Domaines NUMA & 8 \\
Mémoire & 768 Go DDR5 \\
\bottomrule
\end{tabular}
\caption{Configuration des nœuds Romeo x86 (romeo-c)}
\label{tab:annexe:romeo}
\end{table}

\subsection{Variables d'environnement OpenMP recommandées}

\begin{lstlisting}[language=bash, caption={Configuration OpenMP optimale}]
export OMP_NUM_THREADS=192
export OMP_PLACES=cores
export OMP_PROC_BIND=close    # ou spread pour multi-NUMA
export OMP_STACKSIZE=16M
export OMP_SCHEDULE="dynamic,1"
\end{lstlisting}

\subsection{Options de compilation recommandées}

\begin{lstlisting}[language=bash, caption={Flags de compilation pour performance maximale}]
# GCC
CXXFLAGS="-std=c++20 -O3 -march=native -mtune=native \
          -funroll-loops -fomit-frame-pointer -flto \
          -fopenmp -DNDEBUG"

# Clang
CXXFLAGS="-std=c++20 -O3 -march=native \
          -funroll-loops -flto \
          -fopenmp -DNDEBUG"

# MSVC
CXXFLAGS="/std:c++20 /O2 /openmp /DNDEBUG"
\end{lstlisting}


% =============================================================================
% BIBLIOGRAPHIE
% =============================================================================
\printbibliography[heading=bibintoc, title={Références bibliographiques}]

\vfill

\begin{center}
\textit{Fin du rapport}
\end{center}

\end{document}